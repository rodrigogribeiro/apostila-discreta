\section{L\'ogica Proposicional}

\subsection{2.2.1 Exerc\'icios}

	\begin{enumerate}
		\item 
			\begin{enumerate}
				\item
				 \[\begin{array}{ll}    
				 \text{Jo\~ao \'e pol\'itico} & \text{Conjun\c{c}\~ao (mas) }\\
				 \text{Jo\~ao \'e honesto} & \\
				 \end{array}
				 \]
				 
				 \item
				 \[\begin{array}{ll}    
 				 \text{Jo\~ao \'e honesto} & \text{Conjun\c{c}\~ao (mas), Nega\c{c}\~ao (n\~ao) }\\
 				 \text{O irm\~ao de Jo\~ao \'e honesto} & \\
 				 \end{array}
 				 \]
 				 
 				 \item
 				 \[\begin{array}{ll}    
  				 \text{Jo\~ao vir\'a a festa} & \text{Disjun\c{c}\~ao (ou), Conjun\c{c}\~ao (al\'em) }\\
  				 \text{A irm\~a de Jo\~ao vir\'a a festa.} & \\
  				 \text{A m\~ae de Jo\~ao vir\'a a festa.} & \\
  				 \end{array}
  				 \]
				 
				 \item
  				 \[\begin{array}{ll}    
   				 \text{A estrela do espet\'aculo canta.} & \text{Nega\c{c}\~ao (n\~ao) , Conjun\c{c}\~ao + Nega\c{c}\~ao (nem)} \\
   				 \text{A estrela do espet\'aculo dan\c{c}a.} & \\
   				 \text{A estrela do espet\'aculo representa.} & \\
   				 \end{array}
   				 \]
				 
				 \item
   				 \[\begin{array}{ll}    
  				 \text{O trem apita.} & \text{Condicional (Sempre que \ldots) } \\
  				 \text{Jo\~ao sai correndo.} & \\
  				 \end{array}
  				 \]
				 
				 \item
    			 \[\begin{array}{ll}    
   				 \text{Jo\~ao perde dinheiro no jogo.} & \text{Condicional (Caso \ldots), Nega\c{c}\~ao (n\~ao) } \\
   				 \text{Jo\~ao vai a festa.} & \\
   				 \end{array}
   				 \]
				  
				 \item
     			 \[\begin{array}{ll}    
  				 \text{Jo\~ao vai ser multado.} & \text{Condicional (a menos que \ldots) }  \\
  				 \text{Jo\~ao diminui a velocidade.} & \text{Disjun\c{c}\~ao (ou), Nega\c{c}\~ao (n\~ao)} \\
  				 \text{A rodovia tem radar.} & \\
  				 \end{array}
  				 \]
				 				  
				\item
   			    \[\begin{array}{ll}    
				\text{Um n\'umero natural \'e primo.} & \text{Condicional (Uma condi\c{c}\~ao suficiente \ldots)} \\
				\text{Um n\'umero natural \'e \'impar.} & \\
				\end{array}
				\]
				 
				\item
			    \[\begin{array}{ll}    
				\text{Jo\~ao vai ao teatro.} & \text{Condicional (somente se \ldots)} \\
				\text{Uma com\'edia est\'a em cartaz.} & \\
				\end{array}
				\]
				  
				\item
			    \[\begin{array}{ll}    
				\text{Voc\^e \'e Brasileiro.} & \text{Condicional   (Se\ldots)} \\
				\text{Voc\^e gosta de futebol.} & \text{Bicondicional + Nega\c{c}\~ao (a menos \ldots)} \\
				\text{Voc\^e tor\c{c}e para o Tabajara.} & \\
				\text{Voc\^e tor\c{c}e para o \'Ibis.} & \\
				\end{array}
				\]  
				
				\item
			    \[\begin{array}{ll}    
				\text{A propina ser\'a paga.} & \text{Bicondicional (exatamente nas situa\c{c}\~oes} \\
				\text{O deputado vota como instru\'ido por Jo\~ao.} & \text{ em que \ldots)} \\
				\end{array}
				\]
	 
			\end{enumerate}

	\end{enumerate}


\subsection{2.2.3 Exerc\'icios}
	
		\begin{enumerate}
			\item
			\begin{enumerate}
							
			\item  $(A \lor B) \to C$
			\[\begin{array}{c|l} 		
			\text{Vari\'avel} & \text{Proposi\c{c}\~ao Simples} \\ \hline
   			$A$        & \text{Jane vence}\\ 
   			$B$        & \text{Jane perde}\\
   			$C$        & \text{Jane fica cansada} \\									
			\end{array}
			\]			
										
			\item  $A \lor B$
			\[\begin{array}{c|l} 		
			\text{Vari\'avel} & \text{Proposi\c{c}\~ao Simples} \\ \hline
   			$A$        & \text{Rosas s\~ao vermelhas}\\ 
   			$B$        & \text{Violetas s\~ao azuis}\\											
			\end{array}
			\]
		
			\item  $A \to B$
			\[\begin{array}{c|l} 
			\text{Vari\'avel} & \text{Proposi\c{c}\~ao Simples} \\ \hline
			$A$        & \text{Elefantes podem subir em \'arvores}\\ 
			$B$        & \text{3 \'e um n\'umero irracional}\\
			\end{array}
			\]
					
			\item $A \lor B$
			\[\begin{array}{c|l} 
			\text{Vari\'avel} & \text{Proposi\c{c}\~ao Simples} \\ \hline
			$A$        & \text{\'E proibido fumar cigarros}\\ 
			$B$        & \text{3 \'e um n\'umero irracional}\\
			\end{array}
			\]												
			
			\item $ \neg \,(A \to B )$
			\[\begin{array}{c|l} 
			\text{Vari\'avel} & \text{Proposi\c{c}\~ao Simples} \\ \hline
			$A$        & \pi > 0\\ 
			$B$        & \pi > 1\\
			\end{array}
			\]
			
			\item $ A \to B $
			\[\begin{array}{c|l} 
			\text{Vari\'avel} & \text{Proposi\c{c}\~ao Simples} \\ \hline
			$A$        & \text{As laranjas s\~ao amarelas}\\ 
			$B$        & \text{Os morangos s\~ao vermelhos}\\
			\end{array}
			\]			
						
			\item $ \neg ( A \to B) $
			\[\begin{array}{c|l} 
			\text{Vari\'avel} & \text{Proposi\c{c}\~ao Simples} \\ \hline
			$A$        & \text{Montreal \'e a capital do Canad\'a}\\ 
			$B$        & \text{A pr\'oxima copa ser\'a realizada no Brasil}\\
			\end{array}
			\]
	
			\end{enumerate}
	

	
			\item
				\begin{enumerate}
				
				\item $ A \land B $
				\[\begin{array}{c|l} 
				\text{Vari\'avel} & \text{Proposi\c{c}\~ao Simples} \\ \hline
				$A$        & \text{Jo\~ao \'e pol\'itico}\\ 
				$B$        & \text{Jo\~ao \'e honesto}\\
				\end{array}
				\]
					
				\item $ A \land \neg B $
				\[\begin{array}{c|l} 
				\text{Vari\'avel} & \text{Proposi\c{c}\~ao Simples} \\ \hline
				$A$        & \text{Jo\~ao \'e honesto}\\ 
				$B$        & \text{O irm\~ao de jo\~ao \'e honesto}\\
				\end{array}
				\]

				\item $ (A \lor B ) \land C $
				\[\begin{array}{c|l} 
				\text{Vari\'avel} & \text{Proposi\c{c}\~ao Simples} \\ \hline
				$A$        & \text{Jo\~ao vir\'a a festa}\\ 
				$B$        & \text{A irm\~a de Jo\~ao vir\'a a festa}\\
				$C$ 	   & \text{A m\~ae de Jo\~ao vir\'a a festa} \\
				\end{array}
				\]
				
				\item $ \neg A \land \neg B \land \neg C $
				\[\begin{array}{c|l} 
				\text{Vari\'avel} & \text{Proposi\c{c}\~ao Simples} \\ \hline
				$A$        & \text{A estrela do espet\'aculo canta}\\ 
				$B$        & \text{A estrela do espet\'aculo dan\c{c}a}\\
				$C$ 	   & \text{A estrela do espet\'aculo representa} \\
				\end{array}
				\]
				
				\item $ A \to B $
				\[\begin{array}{c|l} 
				\text{Vari\'avel} & \text{Proposi\c{c}\~ao Simples} \\ \hline
				$A$        & \text{O trem apita}\\ 
				$B$        & \text{Jo\~ao sai correndo}\\
				\end{array}
				\]
 
				\item $ \neg A \to B $
				\[\begin{array}{c|l} 
				\text{Vari\'avel} & \text{Proposi\c{c}\~ao Simples} \\ \hline
				$A$        & \text{Jo\~ao perde dinheiro no jogo}\\ 
				$B$        & \text{Jo\~ao vai a festa}\\
				\end{array}
				\] 
				
				\item $ A \to \neg (B \lor \neg C) $
				\[\begin{array}{c|l} 
				\text{Vari\'avel} & \text{Proposi\c{c}\~ao Simples} \\ \hline
				$A$        & \text{Jo\~ao vai ser multado}\\ 
				$B$        & \text{Jo\~ao diminui a velocidade}\\
				$C$        & \text{A rodovia tem radar}\\
				\end{array}
				\]
				
				\item $ A \to B $
				\[\begin{array}{c|l} 
				\text{Vari\'avel} & \text{Proposi\c{c}\~ao Simples} \\ \hline
				$A$        & \text{Um n\'umero natural \'e primo}\\ 
				$B$        & \text{Um n\'umero natural \'e \'impar}\\
				\end{array}
				\]		
						
				\item $ A \to B $
				\[\begin{array}{c|l} 
				\text{Vari\'avel} & \text{Proposi\c{c}\~ao Simples} \\ \hline
				$A$        & \text{Jo\~ao vai ao teatro}\\ 
				$B$        & \text{Uma com\'edia est\'a em cartaz}\\
				\end{array}
				\]		
									
				\item $ (A \to B) \to \neg (C \lor D) $
				\[\begin{array}{c|l} 
				\text{Vari\'avel} & \text{Proposi\c{c}\~ao Simples} \\ \hline
				$A$        & \text{Voc\^e \'e Brasileiro}\\ 
				$B$        & \text{Voc\^e gosta de futebol}\\
				$C$        & \text{Voc\^e tor\c{c}e para o Tabajara}\\ 
				$D$        & \text{Voc\^e tor\c{c}e para o \'Ibis}\\
				\end{array}
				\]		
								
				\item $ A \leftrightarrow B $
				\[\begin{array}{c|l} 
				\text{Vari\'avel} & \text{Proposi\c{c}\~ao Simples} \\ \hline
				$A$        & \text{A propina ser\'a paga}\\ 
				$B$        & \text{O deputado vota como instru\'ido por Jo\~ao}\\
				\end{array}
				\]		
							
				\end{enumerate}
	
		\end{enumerate}

\subsection{2.3.1 Exerc\'icios}

	\begin{enumerate}
	
			\item 
			\begin{enumerate}		
					\item 
					Pela regra 2 temos que as vari\'aveis $A$, $B$ e $C$ s\~ao f\'ormulas da l\'ogica, com isso pela regra 3\,-a temos que $\neg A$ tamb\'em faz parte do conjunto de f\'ormulas. Pela regra 3\,-b temos $\neg A \land B$. Novamente por 3\,-b podemos concluir $\neg A \land B \to C $. 
					
					\item
					Pela regra 2 temos que as vari\'aveis $A$, $B$ e $C$ s\~ao f\'ormulas da l\'ogica. Por 3\,-b temos $A \to B$ e $A \lor B$, com isso, novamente por 3\,-b conclu\'imos $A \lor B \to C$. Pela regra 3\,-a temos $\neg(A \lor B \to C)$ e por fim, pela regra 3\,b chegamos a $(A \to B) \land \neg(A \lor B \to C)$.
					
					\item
					Pelas regras 1 e 2 temos que as vari\'aveis $A$, $B$, $C$ e a constante $ \bot $ pertence ao conjunto de f\'ormulas da l\'ogica. Tomando como base a regra 3\,-b temos $B \to C$ que por sua vez podemos concluir $A \to B \to C $ e novamente por 3\,-b conclu\'imos  $A \to B \to C \leftrightarrow \bot $.
					
					\item
					Pela regra 2 temos que as vari\'aveis $A$, $B$ s\~ao f\'ormulas da l\'ogica, com isso pela regra 3\,-a temos que $\neg A$ tamb\'em faz parte do conjunto de f\'ormulas. Por 3\,-b pode-se construir $\neg A \to B $, novamente por 3\,-b $ A \land \neg A \to B $.
					
					\item
					Pela regra 2 temos que as vari\'aveis $A$, $B$ e $C$ s\~ao f\'ormulas da l\'ogica. Por 3\,-b temos $B \land C$ e com isso conclu\'imos $A \lor B \land C$.			
			\end{enumerate}
			
			\item
			\begin{enumerate}
				\item $(\neg A \land B) \to C$
				\item $(A \to B) \land \neg((A \lor B) \to C)$
				\item $(A \to (B \to C)) \leftrightarrow \bot$
				\item $(A \land \neg A) \to B$
				\item $A \lor (B \land C)$
			\end{enumerate}
			
			
			\item
			\begin{enumerate}
				\item $(A \lor B) \lor (C \lor D)$
				\item $A \to B \to (A \land B)$
				\item $\neg (A \lor B \land C)$
				\item $\neg (A \land (B \lor C))$
			\end{enumerate}
	
	\end{enumerate}

\subsection{2.4.10 Exerc\'icios}


	\begin{enumerate}
	
	
			\item
			\begin{enumerate}
			
			%LETRA A -------------------------------------------->
			\item Conting\^encia	
			\[\begin{array}{|c|c|c|c|c|c|}
			\hline
			 A & B & A \to B & A \lor B & \neg(A \lor B)& (A \to B) \leftrightarrow \neg(A \lor B) \\ \hline
			T & T & T & T & F & F \\
			T & F & F & T & F & F \\
			F & T & T & T & F & F \\
			F & F & T & F & T & T \\
			\hline
			\end{array}
			\]	
			
			
			%LETRA B -------------------------------------------->
			\item Conting\^encia		
			\[\begin{array}{|c|c|c|c|c|c|}
			\hline
			A & B & C & A \land B & (A \land B) \lor C & B \lor C \\ \hline
			T & T & T & T & T & T \\
			T & T & F & T & T & T \\
			T & F & T & F & T & T \\
			T & F & F & F & F & F \\
			F & T & T & F & T & T \\
			F & T & F & F & F & T \\
			F & F & T & F & T & T \\
			F & F & F & F & F & F \\
			\hline
			\end{array}
			\]
			
			\[\begin{array}{|c|c|}
			\hline
			A \land (B \lor C) & (A \land B) \lor C \to A \land (B \lor C)  \\ \hline
			T & T \\
			T & T \\
			T & T \\
			F & T \\
			F & F \\
			F & T \\
			F & F \\
			F & T \\
			\hline
			\end{array}
			\]
			
			%------------------------------------------------------>
			
		
			%LETRA C -------------------------------------------->
			\item Conting\^encia		
			\[\begin{array}{|c|c|c|c|c|}
			\hline
			A & B & \neg A & \neg B & \neg A \lor \neg B \\ \hline
			T & T & F & F & F \\
			T & F & F & T & T \\
			F & T & T & F & T \\
			F & F & T & T & T \\
			\hline
			\end{array}
			\]
			
			
			\[\begin{array}{|c|c|}
			\hline
			\neg(\neg A \lor \neg B) & A \land \neg(\neg A \lor \neg B) \\ \hline
			T & T \\
			F & F \\
			F & F \\
			F & F \\
			\hline
			\end{array}
			\]
			%------------------------------------------------------>
						
					
			%LETRA D -------------------------------------------->
			\item Conting\^encia
			\[\begin{array}{|c|c|c|c|c|}
			\hline
			A & B & A \land B & \neg A & A \land B \to \neg A \\ \hline
			T & T & T & F & F\\
			T & F & F & F & T\\
			F & T & F & T & T\\
			F & F & F & T & T\\
			\hline
			\end{array}
			\]
			
			
			%LETRA E -------------------------------------------->
			\item Tautologia
			\[\begin{array}{|c|c|c|c|c|c|}
			\hline
			A & B & C & A \to B & A \lor C & B \lor C \\ \hline
			T & T & T & T & T & T \\
			T & T & F & T & T & T \\
			T & F & T & F & T & T \\
			T & F & F & F & T & F \\
			F & T & T & T & T & T \\
			F & T & F & T & F & T \\
			F & F & T & T & T & T \\
			F & F & F & T & F & F \\
			\hline
			\end{array}
			\]
			
			\[\begin{array}{|c|c|}
			\hline
			(A \lor C) \to (B \lor C) & (A \to B) \to [(A \lor C) \to (B \lor C)] \\ \hline
			T & T \\
			T & T \\
			T & T \\
			F & T \\
			T & T \\
			T & T \\
			T & T \\
			T & T \\
			\hline
			\end{array}
			\]
			%------------------------------------------------------>
			
			%LETRA F -------------------------------------------->
			\item Tautologia
			\[\begin{array}{|c|c|c|c|}
			\hline
			A & B & B \to A & A \to (B \to A) \\ \hline
			T & T & T & T \\
			T & F & T & T \\
			F & T & F & T \\
			F & F & F & T \\
			\hline
			\end{array}
			\]
			
			%LETRA G -------------------------------------------->
			\item Contradi\c{c}\~ao
			\[\begin{array}{|c|c|c|c|c|}
			\hline
			A & B & A \land B & \neg B & \neg A \\ \hline
			T & T & T & F & F \\
			T & F & F & T & F \\
			F & T & F & F & T \\
			F & F & F & T & T \\
			\hline
			\end{array}
			\]
			
			\[\begin{array}{|c|c|}
			\hline
			(\neg B \lor \neg A) & (A \land B) \leftrightarrow (\neg B \lor \neg A) \\ \hline
			F & F \\
			T & F \\
			T & F \\
			T & F \\
			\hline
			\end{array}
			\]
			
			%------------------------------------------------------>			
			\end{enumerate}
			
			\item
			\begin{enumerate}
				
				%2a
				\item 
				\[\begin{array}{|c|c|c|c|}
				\hline
				P & Q & P \leftrightarrow Q & P \to Q \\ \hline
				T & T & T & T \\
				T & F & F & F \\
				F & T & F & T \\
				F & F & T & T \\
				\hline
				\end{array}
				\]
				
				\[\begin{array}{|c|c|c|c|}
				\hline
				\neg P & \neg Q & (\neg P \to \neg Q) & (P \to Q) \land (\neg P \to \neg Q) \\ \hline
				F & F & T & T \\
				F & T & T & T \\
				T & F & F & F \\
				T & T & T & T \\
				\hline
				\end{array}
				\]
				
				Pela an\'alise das tabelas anteriores podemos concluir que  $P \leftrightarrow Q$ e $(P \to Q) \land (\neg P \to \neg Q)$ n\~ao s\~ao f\'ormulas equivalentes da l\'ogica proposicional, pois apresentam resultados diferentes na solu\c{c}\~ao da tabela verdade.
				
				\item
				\[\begin{array}{|c|c|c|c|c|c|c|}
				\hline
				P & Q & \neg P & \neg Q & P \land \neg Q & \neg P \land Q & (P \land \neg Q) \lor (\neg P \land Q) \\ \hline
				T & T & F & F & F & F & F \\
				T & F & F & T & T & F & T \\
				F & T & T & F & F & T & T \\
				F & F & T & T & F & F & F \\
				
				
				\hline
				\end{array}
				\]
				
				\[\begin{array}{|c|c|c|c|}
				\hline
				P \lor Q & P \land Q & \neg (P \land Q)& (P \lor Q) \land \neg (P \land Q) \\ \hline
				T & T & F & F \\
				T & F & T & T \\
				T & F & T & T \\
				F & F & T & F \\
				\hline
				\end{array}
				\]
				
				$(P \land \neg Q) \lor (\neg P \land Q)$ e $(P \lor Q) \land \neg (P \land Q)$ s\~ao f\'ormulas equivalentes da l\'ogica proposicional, j\'a que apresentam resultados iguais pela tabela verdade.		
			\end{enumerate}
			
			\item 
			\begin{enumerate}
				
				\item Quando dizemos que uma f\'ormula e satisfaz\'ivel isso indica que pelo menos um resultado do conjunto de valores poss\'iveis daquela f\'ormula \'e verdadeiro. Supondo que a f\'ormula a ser analisada seja uma tautologia, a nega\c{c}\~ao da mesma nos dar\'a uma contradi\c{c}\~ao. Portanto se passarmos a nega\c{c}\~ao da f\'ormula para o algoritmo e este retornar falso conclu\'imos que a nega\c{c}\~ao n\~ao \'e satisfaz\'ivel e portanto uma contradi\c{c}\~ao, o que confirma a nossa hip\'otese de que a f\'ormula \'e uma tautologia.
				
				\item Se passarmos a f\'ormula para o algoritmo e este nos retornar falso indica que a f\'ormula n\~ao \'e satisfaz\'ivel e portanto n\~ao possui nenhum resultado no conjunto de valores poss\'iveis com o valor verdadeiro, com isso temos a defini\c{c}\~ao de contradi\c{c}\~ao.
				
			\end{enumerate}
			
			
	\end{enumerate}
	
\subsection{2.5.7 Exerc\'icios}
\begin{enumerate}
	\item
	\begin{enumerate}
	\item $\{(P\land Q)\land R,\, S\land T\}\vdash\,Q\land S$
	
	\[
      \infer[\andI]
               {Q\land S}
               {
                   \infer[\andED]
                            {Q}
                            {
                            \infer[\andEE]
                            		{P\land Q}
                            		{
                            			\infer[\Id]
                            				{(P\land Q)\land R}
                            				{}
                            		}
                            }
                    &  
                   \infer[\andEE]
                            {S}
                            {
	                            \infer[\Id]
	                                   {S\land T}{}
                            }
               }
 	 \]
 	 
 	 
 	 \item $\{(P\land Q)\land R\}\,\vdash\,(P\land R)\lor Z$
 	 
 	 \[
 	 	\infer[\orIE]
 	 		{(P\land R)\lor Z}
 	 		{
 	 			\infer[\andI]
 	 				{P \land R}
 	 				{
 	 					\infer[\andEE]
 	 						{P}
 	 						{
 	 							\infer[\andEE]
 	 								{P\land Q}
 	 								{\infer[\Id]{(P\land Q)\land R}{}}
 	 						}
 	 					&
 	 					\infer[\andED]
 	 						{R}
 	 						{\infer[\Id]{(P\land Q)\land R}{}}
 	 				}
 	 		}
 	 \]
 	 
 	 \item  $\{Q\rightarrow (P\rightarrow R),\, \neg R,\, Q\,\} \vdash\,\neg P$
		
	\[
		\infer[\Id]
			{\neg P}
			{
				\infer[\impI^1]
					{P \to \bot}
					{
						\infer[\impE]
							{\bot}
							{
								\infer[\Id]
									{R \to \bot}{}
							&
								\infer[\impE]
									{R}
									{
										\infer[\impI]
											{P \to R}
											{
												\infer[\Id]
													{Q \to (P \to R)}
													{}
												&
												\infer[\Id]
													{Q}
													{}
											}
										&
										\infer[\Id]
											{P^1}
											{}
									}
							}
					}
			}	
	\]
	
	\item $\{P\}\,\vdash\, Q\rightarrow(P\land Q)$
	
	\[
		\infer[\impI^1]
			{Q\rightarrow(P\land Q)}
			{
				\infer[\andI]
					{P\land Q}
					{
						\infer[\Id]
							{P}
							{}
						&
						\infer[\Id]
							{Q^1}
							{}
					}
			}
	\]
	
	\item $\{(P\rightarrow R)\land (Q\rightarrow R),\, P\land Q\}\,\vdash\, Q\land R$
	
	\[
		\infer[\andI]
			{Q\land R}
			{
				\infer[\andED]
					{Q}
					{
						\infer[Id]
							{P \land Q}
							{}
					}
				&
				\infer[\impE]
					{R}
					{
						\infer[\andEE]
							{P \to R}
							{
								\infer[Id]
									{(P \to R)\land(Q \to R)}
									{}
							}
						&
						\infer[\andEE]
							{P}
							{
								\infer[Id]
									{P \land Q}
									{}
							}
					}
			}
	\]
	
	\item $\{P\rightarrow Q, R\rightarrow S\}\vdash (P\lor R)\rightarrow (Q\lor S)$
	
	\[
		\infer[\impI^1]
			{(P \lor R) \to (Q \lor S)}
			{
				\infer[\orE]
					{Q \lor S}
					{
						\infer[Id]
							{P \lor R}
							{} &
						\infer[\orIE]
							{Q \lor S }
							{
								\infer[\impE]
									{Q}
									{
										\infer[Id]
											{P \to Q}
											{}
										&
										P^1
									}
							} &
						\infer[\orID]
							{Q \lor S }
							{
								\infer[\impE]
									{S}
									{
										\infer[Id]
											{R \to S}
											{}
										&
										R^1
									}
							}
					}
			}
	\]
	\item $\{Q\rightarrow R\}\vdash (P\rightarrow Q)\rightarrow(P\rightarrow R)$
	\[
		\infer[\impI^1]
			{(P \to Q) \to (P \to R)}
			{
				\infer[\impI^2]
					{P \to Q}
					{
						\infer[\impE]
						{Q}
						{
							\infer[\Id]
								{Q \to R}
								{}
							&
							\infer[\impE]
								{Q}
								{
									\infer[\Id^1]
										{P \to Q}
										{}
									&
									\infer[\Id^2]
										{P}
										{}
								}
						}
					}
			}	
	\]
	
	\item $\{(P\land Q)\lor(P\land R)\}\vdash P\land(Q \lor R)$ 
	\[
     	\infer[\orE^1]
              {P \land (Q \lor R)}
              {
                (P\land Q)\lor(P\land R) 
                &
                \infer[\andI]
                         {P\land (Q \lor R)}
                         {
                           \infer[\andEE]
                                    {P}
                                    {P \land Q^1}
                            &
                            \infer[\orIE]
                                     {Q \lor R}
                                     {
                                         \infer[\andED]
                                                  {Q}
                                                  {P \land Q^1}
                                     } 
                         }
                 &
                \infer[\andI]
                         {P \land (Q \lor R)}
                         {
                           \infer[\andED]
                                    {P}
                                    {P \land R^1}
                            &
                            \infer[\orIE]
                                     {Q \lor R}
                                     {
                                         \infer[\andED]
                                                  {R}
                                                  {P \land R^1}
                                     }
                         }
              }
     \]

	\item $\{\neg(A \lor B)\}\vdash \neg A \land \neg B$
	
	\[
		\infer[\andI]
			{\neg A \land \neg B}
			{
			 \infer[\Id]
			 	{\neg A}
			 	{
			 		\infer[\impI^1]
			 			{A \to \bot}
			 			{
			 				\infer[\impE]
			 					{\bot}
			 					{
			 						\infer[\Id]
			 							{(A \lor B) \to \bot}
			 							{\neg(A \lor B)}
			 					&
			 						\infer[\orIE]
			 							{A \lor B}
			 							{A^1}
			 					}
			 			}
			 	}
				&
			 \infer[\Id]
			 	{\neg B}
			 	{
			 		\infer[\impI^2]
			 			{B \to \bot}
			 			{
			 				\infer[\Id]
			 					{(A \lor B) \to \bot}
			 					{\neg(A \lor B)}
			 			&
			 				\infer[\orID]
			 					{A \lor B}
			 					{B^2}
			 			}
			 	}
			}
	\]
	
	\item $\{\neg A \land \neg B\}\vdash \neg (A\lor B)$
	
	\[
		\infer[\Id]
			{\neg (A\lor B)}
			{
				\infer[\impI^1]
					{(A \lor B) \to \bot}
					{
						\infer[\impE]
							{\bot}
							{
								\infer[\Id]
									{\neg(A \lor B)}
									{\neg A \land \neg B}
								&
								(A \lor B)^1	
							}
					}
			}
	\]
	
	\item $\{\neg(A \land B)\}\vdash \neg A \lor \neg B$ 
	\paragraph*{Obs.:}Note que n\~ao \'e poss\'ivel resolver este exerc\'icio sem o uso da t\'ecnica de Redu\c{c}\~ao do Absurdo.
	\[
		\infer[\raa^1]
			{\neg A \lor \neg B}
			{
				\infer[\impE]
					{\bot}
					{
					\infer[\Id]
						{\neg(A \land B)}
						{}
					 &	
					 \infer[\andI]
					 	{A \land B}
					 	{
					 		\infer[\raa^2]
					 			{A}
					 			{
					 				\infer[\impE]
					 					{\bot}
					 					{
					 						\neg(\neg A \lor \neg B)^1
					 						&
					 						\infer[\orIE]
					 							{\neg A \lor \neg B}
					 							{
					 							 \neg A^2
					 							}
					 					}
					 			}
					 		&
					 		\infer[\raa^3]
					 			{B}
					 			{
					 				\infer[\impE]
					 					{\bot}
					 					{			 						\neg(\neg A \lor \neg B)^1
											&   						\infer[\orIE]						{\neg A \lor \neg B}												 										 																		 {\neg B^3}	
						 			}
					 			}	
					 	}
					}
			}
	\]
	
	

	\item $\{\neg A \lor \neg B\}\vdash \neg (A\land B)$
	
	\[
		\infer[\Id]
			{\neg (A \land B)}
			{
				\infer[\impI^1]
					{(A \land B) \to \bot}	
					{
						\infer[\orE^2]
							{\bot}
							{
								\neg A \lor \neg B
								&
									\infer[\impE]
										{\bot}
										{
											\infer[\Id]
												{A \to \bot}
												{\neg A^2}
										&
											\infer[\andEE]
												{A}
												{(A \land B)^1}
										}
								&
									\infer[\impE]
										{\bot}
										{
											\infer[\Id]
												{B \to \bot}
												{\neg B^2}
										&
											\infer[\andED]
												{B}
												{(A \land B)^1}
										}
							}
					}
			}
	\]
	\end{enumerate}
	
	\item
	\begin{enumerate}
		\item $\{A \to B\}\vdash \neg A \lor B$
			\[
				\infer[\raa^1]
					{\neg A \lor B}
					{
						\infer[\impE]
							{\bot}
							{
								\infer[\andED]
									{\neg B}
									{A \land \neg B^1}
								&
								\infer[\impE]
									{B}
									{
										\infer[\Id]
											{A \to B}
											{}
										&
										\infer[\andEE]
											{A}
											{A \land \neg B^1}
									}	
							}	
					}		
			\]
		\item $\vdash (\neg B \to \neg A)\to (A \to B)$
			\[
				\infer[\impE^1]
					{(\neg B \to \neg A)\to (A \to B)}
					{
						\infer[\impE^2]
							{A \to B}
							{
								\infer[\raa^3]
									{B}
									{
										\infer[\impE]
											{\bot}
											{
												\infer[\Id]
													{A \to \bot}
													{
														\infer[\impE]
															{\neg A}
															{
																(\neg B \to \neg A)^1
																&
																\neg B^3	
															}
													}
												&
												A^2
											}
									}
							}
					}
			\]
		
		\item Em andamento
		\item Em andamento

	\end{enumerate}
	
\end{enumerate}

\subsection{2.6.6 Exerc\'icios}

\begin{enumerate}
	\item
	\begin{enumerate}
		
			\item $(A\lor B)\land B\equiv B$
			\[
				\begin{array}{lcl}
					(A \lor B)\land B & = & \{\lor-\text{identidade}\} \\
					(A \lor B) \land (B \lor F) & = & \{\lor-\text{comutativo}\}\\
					(B \lor A) \land (B \lor F) & = & \{\lor-\text{distribui}-\land\}\\
					B \lor (A \land F) & = & \{\lor-\text{null}\}\\
					B \lor F & = & \{\lor-\text{identidade}\}\\
					B & &	
				\end{array}
			\]
			\item $(\neg A\land B)\lor (A\land\neg B)\equiv (A\lor B)\land \neg (A\land B)$
			\[
				\begin{array}{lcl}
					(\neg A\land B)\lor (A\land\neg B) & = & \{\lor-\text{distribui}-\land\}\\
					((\neg A \land B)\lor A) \land ((\neg A \land B) \lor \neg B) & = & \{\lor-\text{distribui}-\land\}\\
					((\neg A \lor A)\land(B \lor A)) \land ((\neg A \lor \neg B) \land (B \lor \neg B)) & = & \{\text{complemento}-\lor\}\\
					\top \land (B \lor A) \land (\neg A \lor \neg B) \land \top & = & \{\text{DeMorgan}-\land\}\\
					(B \lor A) \land \neg(A \land B) & = & \{\lor-\text{comutativo}\}\\
					(A \lor B) \land \neg(A \land B) & &
					
				\end{array}
			\]
			
			\item $((A\rightarrow B)\rightarrow A)\rightarrow A\equiv T$
			\[
				\begin{array}{lcl}
					((A \to B) \to A) \to A & = & \{\text{implica\c{c}\~ao}\}\\
					\neg((A \to B) \to A) \lor A & = & \{\text{implica\c{c}\~ao}\}\\
					\neg(\neg(A \to B) \lor A) \lor A & = & \{\text{implica\c{c}\~ao}\}\\
					\neg(\neg(\neg A \lor B)\lor A)\lor A & = & \{\text{DeMorgan}-\lor\}\\
					\neg((A \land \neg B) \lor A) \lor A & = & \{\lor-\text{distribui}-\land\}\\
					\neg((A \lor A) \land (\neg B \lor A)) \lor A & = &\{\lor-\text{idempotente}\}\\
					\neg(A \land (\neg B \lor A)) \lor A & = & \{\text{DeMorgan}-\land\}\\
					(\neg A \lor \neg(\neg B \lor A))\lor A & = & \{\text{DeMorgan}-\lor\} \\
					\neg A \lor (B \land \neg A) \lor A & = &\{\lor-\text{comutativo}\}\\
					(B \land \neg A) \lor A \lor \neg A & = & \{\text{complemento}-\lor\} \\
					(B \land \neg A) \lor \top & = & \{\lor-\text{null}\}\\
					\top
				\end{array}
			\]	
	
	\end{enumerate}
	
	\item $\{\neg,\land\}$
	\begin{enumerate}
		\item A constante $\bot$ pode ser representada como $\alpha\land\neg\alpha$, pela regra $\{\text{complemento}-\land\}$
		
		\item O conectivo de disjunção pode ser representado por $\neg$ e $\land$ da seguinte maneira, em que $\alpha$ e $\beta$ s\~ao f\'ormulas quaisquer:
			\[
				\begin{array}{lc}
					\alpha \lor \beta & = \\
					\neg \neg \alpha \lor \neg \neg \beta & = \\
					\neg(\neg \alpha \land \neg \beta) & = \\
				\end{array}
			\]
		\item O conectivo de implicação pode ser representado da seguinte maneira, em que $\alpha$ e $\beta$ s\~ao f\'ormulas quaisquer:
			\[
				\begin{array}{lc}
					\alpha \to \beta & = \\
					\neg\alpha \lor \beta & = \\
				\end{array}
			\]
		Como deduzimos anteriormente que $A \lor B \equiv \neg(\neg A \land \neg B)$, e considerando $A = \neg\alpha$ e $B = \beta$, temos:
		\[
			\begin{array}{lc}
				\neg(\neg(\neg\alpha) \land \neg \beta) & = \\
				\neg( \alpha \land \neg\beta)
			\end{array}
		\]
		\item Já o conectivo do bicondicional pode ser representado como demonstrado a seguir, em que $\alpha$ e $\beta$ s\~ao f\'ormulas quaisquer:
		\[
			\begin{array}{lc}
				\alpha \leftrightarrow \beta & = \\
				(\alpha \to \beta) \land (\beta \to \alpha) & = \\
				(\neg \alpha \lor \beta) \land ( \neg\beta \lor \alpha) & = \\
			\end{array}
		\]
		Pela defini\c{c}\~ao anterior do conectivo $\lor$, temos:
		\[
			\begin{array}{lc}
				(\neg(\neg (\neg\alpha) \land \neg\beta)) \land (\neg(\neg (\neg\beta) \land \neg\alpha)) & = \\
				(\neg(\alpha \land \neg\beta)) \land (\neg(\beta \land \neg\alpha))
			\end{array}
		\]
	\end{enumerate}	
	
	\item $\{\neg,\to\}$
		\begin{enumerate}
			\item A constante $\bot$ pode ser representada da seguinte forma:
				\[
					\begin{array}{lcl}
						\alpha \to \alpha & \equiv & \top \, \{\text{nega\c{c}\~ao}-\top\}\\
						\neg(\alpha \to \alpha)  & \equiv & \bot \\
						\neg(\alpha \to \alpha)  & & \\
					\end{array}
				\]
			\item O conectivo $\lor$ \'e representado usando a lei da \{implica\c{c}\~ao\}:
				\[
					\begin{array}{lcl}
						\alpha \to \beta & \equiv & \neg \alpha \lor \beta \\
						\neg \alpha \to \beta & \equiv & \neg \neg \alpha \lor \beta\\
						\neg \alpha \to \beta & \equiv & \alpha \lor \beta \\
						\neg \alpha \to \beta  &  & \\
					\end{array}
				\]
			\item Com o conectivo $\lor$ definido, podemos ent\~ao us\'a-lo para construir a f\'ormula equivalente ao conectivo $\land$:
				\[
					\begin{array}{lcl}
						\neg \alpha \to \beta & \equiv & \alpha \lor \beta \\
						\neg(\neg \alpha \to \beta) & \equiv & \neg \alpha \land \neg \beta \\
						\neg(\neg \neg \alpha \to \neg \beta) & \equiv & \alpha \land \beta \\
						\neg( \alpha \to \neg \beta) & & \\
					\end{array}
				\]
			\item Enfim, constru\'imos o \'ultimo conectivo usando a lei do bicondicional:
			\[
				\begin{array}{lcl}
					\alpha \leftrightarrow \beta & \equiv & \{\text{bicondicional}\} \\
					(\alpha \to \beta) \land (\beta \to \alpha) & \equiv & \{\text{pela letra c) deste exerc\'icio}\}\\
					\neg[(\alpha \to \beta) \to \neg(\beta \to \alpha) ] & & \\
				\end{array}
			\]
			 
		\end{enumerate}
	\item em andamento
	\item
		\begin{enumerate}
			\item 
			\[\begin{array}{|c|c|c|}
				\hline
				\alpha & \beta & \alpha\downarrow\beta  \\ \hline
				T & T & F  \\
				T & F & T  \\
				F & T & T  \\
				F & F & T  \\
				\hline
			\end{array}
			\]
			\item em andamento
		\end{enumerate}

\end{enumerate}

\subsection{2.7.3 Exerc\'icios}
	\begin{enumerate}
		\item
			\begin{enumerate}
				\item $(A\land B)\lor C\rightarrow A\land(B\lor C)$
					\[
						\begin{array}{lcl}
						\text{Forma Conjuntiva} & & \\
						&&\\
						(A\land B)\lor C\rightarrow A\land(B\lor C) & \equiv & \text{passo 2}\\
						
						\neg ((A \land B)\lor C) \lor (A \land (B \lor C))& \equiv & \text{passo 3}\\
						
						
						(\neg (A \land B) \land \neg C) \lor (A \land (B \lor C)) & \equiv & \ldots \\
						
						
						((\neg A \lor \neg B) \land \neg C) \lor (A \land (B \lor C)) & \equiv & \text{passo 5}\\
						
						((\neg A \lor \neg B) \land \neg C) \lor A) \land (((\neg A \lor \neg B) \land \neg C) \lor (B \lor C)) & \equiv & \ldots \\
						
						((\neg A \lor \neg B) \lor A) \land (\neg C \lor A) \land ((\neg A \lor \neg B) \lor (B \lor C)) \land (\neg C \lor (B \lor C)) & \equiv & \\
						
						(\neg B \lor (\neg A \lor A)) \land (\neg C \lor A) \land ((\neg A \lor (\neg B \lor B) \lor C) \land (\neg C \lor C)\lor B) & \equiv &  \\
						
						\neg B \land (\neg C \lor A) \land (\neg A \lor C) \land B & \equiv &  \\
						
						(\neg C \lor A) \land (\neg A \lor C) \land (\neg B \land B) & \equiv &  \\
						
						(\neg C \lor A) \land (\neg A \lor C) &  & \\
						
						& & \\
						
						%----------------------------------------------------
						
						\text{Forma Disjuntiva} & & \\
						& & \\
						
						
						(A\land B)\lor C\rightarrow A\land(B\lor C) & \equiv & \text{passo 2}\\
												
						\neg ((A \land B)\lor C) \lor (A \land (B \lor C))& \equiv & \text{passo 3}\\
						
						
						(\neg (A \land B) \land \neg C) \lor (A \land (B \lor C)) & \equiv & \ldots \\
						
						
						((\neg A \lor \neg B) \land \neg C) \lor (A \land (B \lor C)) & \equiv & \text{passo 5}\\
						
						((\neg A \land \neg C) \lor (\neg B \land \neg C))\lor ((A \land B) \lor (A \land C)) & \equiv & \\
						
						(\neg A \land \neg C) \lor (\neg B \land \neg C)\lor (A \land B) \lor (A \land C) &  & \\

						\end{array}
					\]
				
				\item $A\land\neg (\neg A\lor \neg B)$
					\[
						\begin{array}{lcl}
							\text{Forma Conjuntiva} &  & \\
							& & \\
							A\land \neg (\neg A \lor \neg B) & \equiv & \text{passo 3} \\
							A \land (\neg \neg A \land B) & \equiv & \text{passo 4} \\
							A \land (A \land B) &  & \\
						\end{array}
					\]
				
				
				\item $A\land B\rightarrow\neg A$
						\[
							\begin{array}{lcl}
								\text{Forma Disjuntiva} & & \\
								& & \\
								A\land B\rightarrow\neg A & \equiv & \text{passo 2}\\
								\neg(A \land B) \lor \neg A & \equiv & \text{passo 5} \\
								(\neg A \lor B) \lor \neg A
							\end{array}
						\]
				
				\item $(A\rightarrow B)\rightarrow[(A\lor C)\rightarrow (B\lor C)]$ 
					\[
						\begin{array}{lcl}
							\text{Forma Conjuntiva} &  & \\
							& & \\
							(A\rightarrow B)\rightarrow[(A\lor C)\rightarrow (B\lor C)] & \equiv & \\
							
							(A\rightarrow B)\rightarrow[\neg(A\lor C)\lor (B\lor C)] & \equiv & \\
							
							
							\neg(A\rightarrow B)\lor[\neg(A\lor C)\lor (B\lor C)] & \equiv & \\
							
							\neg(\neg A\lor B)\lor[\neg(A\lor C)\lor (B\lor C)] & \equiv & \\
							
							(\neg \neg A \land \neg B) \lor [(\neg A \land \neg C) \lor (B \lor C)] & \equiv & \\
							
							(A \land \neg B) \lor [(\neg A \land \neg C) \lor (B \lor C)] & \equiv & \\
							
							(A \land \neg B) \lor [(\neg A \lor (B \lor C)) \land (\neg C \lor (B \lor C))] & \equiv & \\
							(A \land \neg B) \lor [(\neg A \lor (B \lor C)) \land B] & \equiv & \\
							A \lor [( \neg A \lor (B \lor C)) \land B] \land  \neg B \lor [(\neg A \lor (B \lor C)) \land B] & \equiv & \\
							
						    [(A \lor( \neg A \lor (B \lor C))) \land (A \lor B)] \land [(\neg B \lor (\neg A \lor (B \lor C))) \land (\neg B \lor B)] & \equiv & \\
								
							[(A \lor \neg A \lor B \lor C) \land (A \lor B)] \land [(\neg B \lor \neg A \lor B \lor C) \land (\neg B \lor B)] & \equiv & \\
							
							[( B \lor C) \land (A \lor B)] \land [ \neg A  \lor C] & \equiv & \\

						\end{array}
					\]
				
				
				\item $A\rightarrow(B\rightarrow A)$ 
					\[
						\begin{array}{lcl}
							
							A\rightarrow(B\rightarrow A) & \equiv & \text{passo 2} \\
							A \to (\neg B \lor A) & \equiv & \text{passo 2}\\
							\neg A \lor (\neg B \lor A) & \equiv & \\
							\neg B \lor (\neg A \lor A) & \equiv & \\
							\neg B & & \\
							
						\end{array}
					\]
				\item $(A\land B)\leftrightarrow(\neg B\lor \neg A)$
					\[
						\begin{array}{lcl}
							\text{Forma Conjuntiva} & & \\
							& & \\
							(A\land B)\leftrightarrow(\neg B\lor \neg A) &\equiv & \\
							((A \land B) \to (\neg B \lor \neg A)) \land ((\neg B \lor \neg A) \to (A \land B)) & \equiv& \\
							(\neg (A \land B) \lor (\neg B \lor \neg A)) \land (\neg (\neg B \lor \neg A) \lor (A \land B)) &\equiv & \\
							((\neg A \lor \neg B) \lor (\neg B \lor \neg A)) \land ((\neg \neg B \land \neg \neg A) \lor (A \land B)) & \equiv& \\
							((\neg A \lor \neg B) \lor (\neg B \lor \neg A)) \land ((B \land A) \lor (A \land B)) & \equiv& \\
							((\neg A \lor \neg B) \lor (\neg B \lor \neg A)) \land ((B \lor (A \land B)) \land (A \lor (A \land B))) &\equiv & \\
							((\neg A \lor \neg B) \lor (\neg B \lor \neg A)) \land 
							((B \lor A) \land (B \lor B) \land (A \lor A) \land (A \lor B)) &\equiv & \\
							((\neg A \lor \neg B) \lor (\neg B \lor \neg A)) \land 
							((B \lor A) \land B  \land A \land (A \lor B)) &\equiv & \\
							(\neg A \lor \neg B) \land ((A \lor B) \land B  \land A) & \equiv & \\
							& & \\
							\text{Forma Disjuntiva} & & \\
							& & \\
							(A\land B)\leftrightarrow(\neg B\lor \neg A) &\equiv & \\
							((A \land B) \to (\neg B \lor \neg A)) \land ((\neg B \lor \neg A) \to (A \land B)) & \equiv& \\
							(\neg (A \land B) \lor (\neg B \lor \neg A)) \land (\neg (\neg B \lor \neg A) \lor (A \land B)) &\equiv & \\
							((\neg A \lor \neg B) \lor (\neg B \lor \neg A)) \land ((\neg \neg B \land \neg \neg A) \lor (A \land B)) & \equiv& \\
							((\neg A \lor \neg B) \lor (\neg B \lor \neg A)) \land ((B \land A) \lor (A \land B)) & \equiv& \\
							((\neg A \lor \neg B) \land ((B \land A) \lor (A \land B))) \lor ((\neg B \lor \neg A) \land ((B \land A) \lor (A \land B))) & & \\
							((\neg A \lor \neg B) \land (A \land B)) \lor ((\neg B \lor \neg A) \land (A \land B)) & & \\
							(((\neg A \land (A \land B)) \lor (\neg B \land (A \land B))) \lor (((\neg B \land (A \land B)) \lor (\neg A \land (A \land B))) & & \\
							(B \lor A) \lor (A \lor B) & & \\
						
							
						\end{array}
					\]
			\end{enumerate}
		
	\end{enumerate}






