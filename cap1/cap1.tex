\chapter{L\'ogica Proposicional}

A l\'ogica prov\^e um ferramental para o racioc\'inio sobre matem\'atica, algoritmos
e circuitos digitais. A aplicabilidade da l\'ogica permeia diversas \'areas da computa\c{c}\~ao,
citaremos aqui apenas alguns exemplos:

\begin{itemize}
  \item \textbf{Engenharia de Software}: considera-se uma boa pr\'atica especificar um sistema antes 
        de iniciar a sua codifica\c{c}\~ao. Uma das v\'arias t\'ecnicas de especifica\c{c}\~ao de 
        software \'e o uso de l\'ogica.
  \item \textbf{Aplica\c{c}\~oes de Miss\~ao Cr\'itica}: dizemos que uma determinada aplica\c{c}\~ao 
        \'e de miss\~ao cr\'itica se essa est\'a relacionada a algum risco (de vida, elevados preju\'izos financeiros, etc.).
        Em tais aplica\c{c}\~oes, a utiliza\c{c}\~ao de testes para garantir o funcionamento adequado 
        n\~ao \'e suficiente. O que espera-se \'e uma prova de corretude do programa em quest\~ao, isso \'e uma
        demonstra\c{c}\~ao de que o programa comporta-se de acordo com sua especifica\c{c}\~ao em todas as
        situa\c{c}\~oes poss\'iveis. A l\'ogica \'e a fundamenta\c{c}\~ao matem\'atica de demonstra\c{c}\~oes de
        corre\c{c}\~ao de programas.
   \item \textbf{Recupera\c{c}\~ao de informa\c{c}\~ao}: em m\'aquinas de busca para Web, utiliza-se l\'ogica para
         especificar como propriedades que classificam uma determinada p\'agina como relevante ou n\~ao com base
         em seu conte\'udo.
   \item \textbf{Circuitos Digitais e Arquitetura de Computadores:} l\'ogica \'e a linguagem utilizada para descrever
         sinais produzidos e recebidos como entrada por componentes eletr\^onicos. Um problema comum no projeto de
         circuitos eletr\^onicos \'e determinar uma vers\~ao equivalente, por\'em mais eficiente, de um circuito l\'ogico.
         T\'ecnicas para solu\c{c}\~ao desse problema s\~ao baseadas em algoritmos eficientes para o processamento de
         f\'ormulas da l\'ogica.
   \item \textbf{Bancos de dados}: um recurso fundamental de qualquer sistema gerenciador de bancos de dados \'e uma linguagem
         simples e expressiva para recuperar informa\c{c}\~oes nele armazenadas. A utiliza\c{c}\~ao de recursos baseados em 
         l\'ogica \'e a chave para a expressividade de linguagens para consultas a bancos de dados.       
\end{itemize}

Al\'em das \'areas citadas anteriormente, a l\'ogica \'e fundamental no estudo e no projeto de linguagens de programa\c{c}\~ao
e da teoria de computabilidade.

Neste cap\'itulo, discutiremos as dificuldades presentes na utiliza\c{c}\~ao do Portugu\^es para expressar racioc\'inio 
l\'ogico e como contorn\'a-las utilizando l\'ogica formal. Existem diversos tipos de l\'ogicas formais, cada uma com 
uma aplica\c{c}\~ao espec\'ifica. Inicialmente consideraremos uma l\'ogica bem simples chamada de l\'ogica proposicional.
Primeiramente, iremos analisar a sintaxe da linguagem da l\'ogica proposicional e depois consideraremos tr\^es sistemas 
matem\'aticos para racioc\'inio sobre f\'ormulas da l\'ogica proposicional: tabelas verdade, dedu\c{c}\~ao natural e \'algebra 
Booleana.

\emph{Tabelas verdade} definem o significado dos conectivos l\'ogicos e como eles podem ser utilizados para calcular os 
valores de express\~oes e provar que duas proposi\c{c}\~oes s\~ao logicamente equivalentes. Como tabelas verdade
expressam diretamente o siginificado de prpoposi\c{c}\~oes, dizemos que essas s\~ao uma abordagem baseada em sem\^antica para l\'ogica.]
Tabelas verdade s\~ao de simples entendimento, por\'em n\~ao possuem s\~ao \'uteis na solu\c{c}\~ao de problemas reais devido ao seu tamanho.

\emph{Dedu\c{c}\~ao Natural} \'e uma formaliza\c{c}\~ao de princ\'ipios b\'asicos de racioc\'inio l\'ogico utilizado no cotidiano.
A dedu\c{c}\~ao natural prov\^e um conjunto de regras de infer\^encia que especificam exatamente quais fatos podem ser deduzidos
a partir de um conjunto de fatos dados. Em dedu\c{c}\~ao natural n\~ao h\'a a no\c{c}\~ao de `valor l\'ogico` de proposi\c{c}\~oes;
j\'a que tudo no sistema est\'a encapsulado em suas regras de infer\^encia. Conforme veremos posteriormente, essas regras s\~ao baseadas
na estrutura das proposi\c{c}\~oes envolvidas, a dedu\c{c}\~ao natural \'e uma abordagem puramente sint\'atica para a l\'ogica. Diversas
t\'ecnicas utilizadas em pesquisas na \'areas de linguagens de programa\c{c}\~ao s\~ao baseadas em sistemas l\'ogicos que s\~ao de alguma
maneira relacionados \`a dedu\c{c}\~ao natural.

\emph{\'Algebra Booleana} \'e uma abordagem para formaliza\c{c}\~ao da l\'ogica baseada em um conjunto de equa\c{c}\~oes --- as leis
da \'algebra Booleana --- para especificar que certas proposi\c{c}\~oes s\~ao iguais a outras. A \'algebra Booleana \'e uma abordagem
axiom\'atica, similar a \'algebra elementar e geometria, pois prov\^e um conjunto de leis para manipular proposi\c{c}\~oes. T\'ecnicas
alg\'ebricas para a l\'ogica s\~ao fundamentais para o projeto de circuitos digitais.

\section{Introdu\c{c}\~ao \`a L\'ogica Formal}

A l\'ogica formal foi inicialmente concebida na gr\'ecia antiga onde fil\'osofos desejavam ser capazes de analisar argumentos 
em linguagem natural. Os gregos eram fascinados pela id\'eia de que alguns argumentos eram sempre verdadeiros e outros sempre
falsos. Por\'em, eles rapidamente perceberam que o racioc\'inio l\'ogico \'e dif\'icil de ser analisado usando linguagens naturais
como o Grego (ou o Portugu\^es!). Isso se deve principalmente devido \`as \emph{ambiguidades} inerentes \`as linguagens naturais.
Uma das maneiras de se evitar essas dificuldades \'e o uso de vari\'aveis que denominaremos \emph{vari\'aveis proposicionais}. 

Suponha que um conhecido lhe diga `O dia est\'a ensolarado e estou feliz`. Aparentemente essa frase possui interpreta\c{c}\~ao
\'obvia, mas ao observ\'a-la com cuidado percebe-se que o significado dessa n\~ao \'e t\~ao evidente. Talvez essa pessoa goste
de dias ensolarados e fica contente quando esse fato ocorre. Note que existe uma conex\~ao entre as duas partes da senten\c{c}a,
dessa forma, a palavra `e` presente na frase `O dia est\'a ensolarado e estou feliz` significa `e, portanto`. Por\'em, essa an\'alise
depende de nossa experi\^encia em relacionar o clima com a felicidade das pessoas. Considere agora o seguinte exemplo: 
`Gatos s\~ao peludos e elefantes pesados`. Essa senten\c{c}a possui a mesma estrutura do exemplo anterior, mas ningu\'em ir\'a tentar
relacionar o peso de elefantes com a quantidade de pelos de gatos. Neste caso, a palavra `e` significa `e, tamb\'em`. Pode-se perceber que a
palavra `e` possui diversos significados sutis, e escolhemos o significado apropriado usando nosso conhecimento do mundo \`a nossa volta.
Perceba que as duas simples frases de exemplo consideradas ilustram as dificuldades de interpreta\c{c}\~ao que podem surgir ao se utilizar
uma linguagem natural. As dificuldades em se dar um significado preciso a frases em linguagem natural n\~ao se restringem a somente como 
intepretar a palavra `e`. O estudo preciso da sem\^antica de senten\c{c}as expressas em linguagem natural \'e objeto de estudo da lingu\'istica
e da filosofia.

Ao inv\'es de tentarmos o imposs\'ivel --- expressar, de maneira precisa,  racioc\'inio l\'ogico em linguagem natural  --- n\'os iremos separar
a estrutura l\'ogica de um argumento de todas as conota\c{c}\~oes da l\'ingua portuguesa. Faremos isso utilizando \textbf{proposi\c{c}\~oes}, que
s\~ao definidas a seguir.

\begin{Definition}[Proposi\c{c}\~ao]
  Definimos por proposi\c{c}\~ao qualquer senten\c{c}a pass\'ivel de possuir um dos valores l\'ogicos: verdadeiro ou falso.
\end{Definition}

Sempre que poss\'ivel, ap\'os uma defini\c{c}\~ao, apresentaremos alguns exemplos para ilustr\'a-la.

\begin{Example}
  Quais das seguintes senten\c{c}as podem ser consideradas proposi\c{c}\~oes?
  \begin{enumerate}
    \item Hoje \'e segunda-feira.
    \item $10 < 7$
    \item $x + 1 = 3$
    \item Como est\'a voc\^e?
    \item Ela \'e muito talentosa
    \item Existe vida em outros planetas.
  \end{enumerate}
  Neste exemplo, temos que a senten\c{c}a $1$ \'e uma proposi\c{c}\~ao, pois o dia de hoje pode ser ou n\~ao segunda-feira tornando essa frase
  verdadeira ou falsa. A senten\c{c}a $2$ \'e uma proposi\c{c}\~ao, pois temos que $10$ n\~ao \'e menor do que $7$. Logo, o valor l\'ogico dessa
  senten\c{c}a \'e igual a falso. A senten\c{c}a $3$ n\~ao \'e uma proposi\c{c}\~ao pois seu valor l\'ogico depende do valor atribu\'ido a 
  vari\'avel $x$. Se $x = 2$, temos que a senten\c{c}a $3$ \'e verdadeira. A mesma senten\c{c}a $3$ \'e falsa para qualquer outro valor de $x$. 
  Logo, como n\~ao \'e poss\'ivel determinar de maneira \'unica o valor l\'ogico da senten\c{c}a $3$, essa n\~ao \'e considerada uma 
  proposi\c{c}\~ao. 
  A senten\c{c}a $4$ n\~ao \'e uma proposi\c{c}\~ao pois n\~ao \'e poss\'ivel atribuir um valor verdadeiro ou falso para uma pergunta.
  A senten\c{c}a $5$ n\~ao \'e uma proposi\c{c}\~ao pois ``ela'' n\~ao est\'a especificada. Portanto, o fato de ``ela'' ser talentosa ou n\~ao
  depende de quem \'e ``ela''. Logo, essa senten\c{c}a n\~ao \'e uma proposi\c{c}\~ao.
  A senten\c{c}a $6$ \'e uma proposi\c{c}\~ao pois o fato de existir vida em outros planetas pode ser verdadeiro ou falso.
\end{Example}

Visando eliminar detalhes desnecess\'arios representaremos proposi\c{c}\~oes