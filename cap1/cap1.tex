\chapter{Conceitos Preliminares}

O objetivo deste cap\'itulo \'e apresentar alguns conceitos que ser\~ao utilizados durante o todo texto. Primeiramente,
apresentaremos os conceitos de sintaxe e sem\^antica que s\~ao fundamentais na Ci\^encia da Computa\c{c}\~ao. Logo ap\'os,
apresentamos uma breve introdu\c{c}\~ao ao assistente de provas Coq, que ser\'a utilizado durante esta apostila como uma
forma de colocar a teoria em um contexto pr\'atico usando uma ferramenta computacional.

\section{Linguagens Formais}

Matematicamente, uma linguagem formal \'e um conjunto (finito ou infinito) de termos estruturados. Esses
termos s\~ao de tamanho finito e s\~ao definidos sobre um conjunto finito de s\'imbolos denominado \textit{alfabeto}.
A defini\c{c}\~ao de uma linguagem formal apenas descreve a estrutura de seus elementos. Por\'em, somente a especifica\c{c}\~ao
da sintaxe de uma linguagem n\~ao \'e de grande utilidade se estes n\~ao possu\'irem sem\^antica, isto \'e, uma maneira de
interpretar estes elementos de maneira a lhes atribuir um significado. Na se\c{c}\~ao \ref{cap1:syn}, nosso objetivo ser\'a apresentar
algumas defini\c{c}\~oes de sintaxe de linguagens simples e, posteriormente na se\c{c}\~ao \ref{cap1:sem} ser\'a apresentado como
atribuir significado aos elementos destas linguagens.

\subsection{Sintaxe}\label{cap1:syn}

Linguagens finitas podem (em princ\'ipio) ser especificadas pela enumera\c{c}\~ao de termos da linguagem; j\'a, linguagens
infinitas s\~ao usualmente descritas por um conjunto finito de regras para construir o conjunto de termos inclu\'idos nesta linguagem.
O estudo de linguagens formais possui uma extensiva literatura e \'e o objeto de estudo da disciplina Fundamentos Te\'oricos da Computa\c{c}\~ao.
Neste cap\'itulo, iremos apenas apresentar alguns exemplos que visam ilustrar como definir conjuntos em termos de um conjunto de regras.

\begin{Definition}[Sintaxe do Conjunto de Booleanos]
  O conjunto de valores Booleanos (l\'ogicos), $\mathcal{B}$, \'e um conjunto finito, que pode ser definido pelas seguintes regras:
  \[
      \begin{array}{lcl}
        T & \in & \mathcal{B}\\
        F & \in & \mathcal{B}
      \end{array}
  \]
\end{Definition}

Em conjunto, essas regras podem ser interpretadas como ``o conjunto de valores booleanos, $\mathcal{B}$, \'e formado apenas pelos valores 
$T$ e $F$''.
O conjunto de valores Booleanos \'e um conjunto finito e sua defini\c{c}\~ao utilizando regras \'e imediata. A pr\'oxima defini\c{c}\~ao,
apresenta a sintaxe de termos equivalentes ao conjunto dos n\'umeros naturais.

\begin{Definition}[Sintaxe do Conjunto de N\'umeros Naturais]
O conjunto de termos equivalentes aos n\'umeros naturais, $\mathcal{N}$, \'e um conjunto infinito que pode ser definido pelas seguintes regras:
\[
   \begin{array}{l}
     zero \in \mathcal{N}\\
     \text{se }n \in \mathcal{N} \text{ ent\~ao }suc\,n\in\mathcal{N}
   \end{array}
\]
\end{Definition}

As regras anteriores podem ser entendidas como:
\begin{itemize}
  \item O termo $zero$ pertence ao conjunto $\mathcal{N}$;
  \item Se $n$ \'e um termo pertencente ao conjunto $\mathcal{N}$, temos que o termo $suc\,n$ tamb\'em pertence a $\mathcal{N}$.
\end{itemize}
Desta forma, o conjunto de termos $\mathcal{N}$ \'e $\mathcal{N}=\{zero,\,suc\,zero,\,suc\,(suc\,zero)\,...\}$. A equival\^encia entre o 
conjunto $\mathcal{N}$ e o conjunto de n\'umeros naturais, $\mathbb{N} = \{0,1,2,...\}$, ser\'a apresentada na se\c{c}\~ao \ref{cap1:sem}.
Antes de iniciarmos a discuss\~ao sobre como atribuir sem\^antica a termos de uma linguagem, iremos apresentar mais dois exemplos de 
defini\c{c}\~oes de sintaxe.

\begin{Definition}[Sintaxe do Conjuntos de Listas]
  O conjunto de listas de elementos de um conjunto $\mathcal{T}$, $\textit{List }\mathcal{T}$, \'e definido pelas seguintes regras:
  \[
  \begin{array}{l}
    [\,] \in \textit{List }\mathcal{T}\\
    \text{se }t \in \mathcal{T} \text{ e } ts \in \textit{List }\mathcal{T}\text{ ent\~ao } t :: ts \in \textit{List }\mathcal{T}.
  \end{array}
  \]
\end{Definition}
Listas s\~ao definidas de maneira independente do conjunto de elementos que formam essas listas, isto \'e, a defini\c{c}\~ao de listas
\'e polim\'orfica em rela\c{c}\~ao ao conjunto de seus elementos.
Por exemplo, o conjunto de listas sobre 
o conjunto de valores Booleanos \'e $\textit{List }\mathcal{B}=\{[\,],\,F :: [\,],\,T :: [\,],\, F :: (T :: [\,]),\, ...\}$.

\begin{Definition}[Sintaxe do Conjunto de Express\~oes Aritm\'eticas]\label{def:arithexp}
  O conjunto de termos equivalentes a express\~oes aritm\'eticas, $\mathcal{E}$, \'e definido pelas seguintes regras:
  \[
  \begin{array}{l}
    \text{se }n\in\mathcal{N}\text{, ent\~ao } \textit{const }n\in\mathcal{E}\\
    \text{se }e_1 \in \mathcal{E} \text{ e } e_2 \in \mathcal{E}\text{ ent\~ao }\textit{plus }e_1\,e_2\in\mathcal{E}\\
    \text{se }e_1 \in \mathcal{E} \text{ e } e_2 \in \mathcal{E}\text{ ent\~ao }\textit{times }e_1\,e_2\in\mathcal{E}
  \end{array}
  \]
\end{Definition}
Nas defini\c{c}\~oes anteriores apresentamos a estrutura sint\'atica de quatro conjuntos de termos e, informalmente, explicitamos a equival\^encia
destes com outros conjuntos j\'a conhecidos. A pr\'oxima se\c{c}\~ao descrever\'a como atribuir significado a essas defini\c{c}\~oes sint\'aticas.

\subsection{Sem\^antica}\label{cap1:sem}

Defini\c{c}\~oes sem\^anticas associam significado a sintaxe. Formalmente, a sem\^antica de uma linguagem \'e descrita como
uma fun\c{c}\~ao que associa termos da linguagem em quest\~ao a elementos de um conjunto cujo significado \'e definido
matematicamente, como por exemplo, o conjunto dos n\'umeros naturais, $\mathbb{N}$. Idealmente, a sem\^antica de uma linguagem formal
\'e definida em termos de sua estrutura sint\'atica.

Antes de apresentarmos uma fun\c{c}\~ao sem\^antica, elementos de uma linguagem s\~ao apenas uma sequ\^encia estruturada de s\'imbolos
sem significado. Como qualquer fun\c{c}\~ao, defini\c{c}\~oes sem\^anticas devem ser especificadas em termos de seu dom\'inio e 
contra-dom\'inio\footnote{Neste ponto, assumimos que os conceitos de dom\'inio e contra-dom\'inio de fun\c{c}\~oes \'e familiar ao leitor.
Estes conceitos ser\~ao apresentados formalmente no cap\'itulo \ref{}}. As pr\'oximas defini\c{c}\~oes ilustram poss\'iveis fun\c{c}\~oes 
sem\^anticas para as linguagens descritas na se\c{c}\~ao \ref{cap1:syn}.

\begin{Definition}[Sem\^antica do Conjunto de Booleanos]
Uma poss\'ivel sem\^antica de termos do conjunto $\mathcal{B}$ \'e dada pela seguinte fun\c{c}\~ao:
\[
\begin{array}{lcl}
\llbracket T \rrbracket & = & 1\\
\llbracket F \rrbracket & = & 0\\
\end{array}
\]
Note que o dom\'inio desta fun\c{c}\~ao \'e $\mathcal{B}$ e o contra-dom\'inio o conjunto $\{0,1\}$.
\end{Definition}
Evidentemente, a fun\c{c}\~ao anterior n\~ao \'e a \'unica poss\'ivel maneira de interpretarmos termos de $\mathcal{B}$.
Outra poss\'ivel defini\c{c}\~ao seria:
\[
\begin{array}{lcl}
\llbracket T \rrbracket & = & \{k\in\mathbb{N}\,|\,k\neq 0\}\\
\llbracket F \rrbracket & = & \{0\}\\
\end{array}
\]
Em que o termo $T$ \'e associado com o conjunto de todos os n\'umeros naturais diferentes de $0$ e $F$ com o conjunto contendo 
o n\'umero $0$. A fun\c{c}\~ao anterior atribui a sem\^antica para Booleanos similar \`a utilizada pelas linguagens de programa\c{c}\~ao
C/C++, em que o valor verdadeiro \'e associado a qualquer inteiro n\~ao zero e falso a zero.

A fun\c{c}\~ao sem\^antica para n\'umeros naturais \'e mostrada na defini\c{c}\~ao seguinte.

\begin{Definition}[Sem\^antica do Conjunto de N\'umeros Naturais]
De maneira simplista, uma forma de atribuir significado aos elementos de $\mathcal{N}$ \'e associar o valor $0$ ao termo $zero\in\mathcal{N}$
e o valor $k\in\mathbb{N}$ ao termo contendo $k$ ocorr\^encias da constante $suc$. Isto pode ser definido recursivamente da seguinte maneira:
\[
\begin{array}{lcl}
\llbracket zero \rrbracket & = & 0\\
\llbracket suc\,\,n\rrbracket & = & \llbracket n \rrbracket + 1, \text{ para }n\in\mathcal{N}
\end{array}
\]
\end{Definition}

A defini\c{c}\~ao anterior \'e um exemplo de uma fun\c{c}\~ao recursiva sobre a
estrutura da sintaxe. Como a sintaxe do conjunto $\mathcal{N}$ \'e definida recursivamente, a
fun\c{c}\~ao que lhe atribui significado \'e tamb\'em recursiva. Esperamos que o leitor deste texto
seja familiar com o conceito de recurs\~ao.

Para garantir que defini\c{c}\~oes recursivas sejam consideradas fun\c{c}\~oes, estas devem obedecer dois crit\'erios:
totalidade e termina\c{c}\~ao. A totalidade especifica que a fun\c{c}\~ao deve associar todo elemento de seu dom\'inio
a um elemento no contradom\'inio. Uma maneira de se garantir a totalidade \'e especificar uma equa\c{c}\~ao para cada uma
das regras de forma\c{c}\~ao da sintaxe. Na defini\c{c}\~ao anterior, temos que a fun\c{c}\~ao que atribui sem\^antica a
elementos do conjunto $\mathcal{N}$ \'e total, pois esta \'e definida para todas as regras de forma\c{c}\~ao da sintaxe de 
$\mathcal{N}$. A termina\c{c}\~ao pode ser garantida permitindo que chamadas recursivas sejam feitas apenas a sub-termos.
A fun\c{c}\~ao sem\^antica de $\mathcal{N}$ possui a propriedade de termina\c{c}\~ao, pois, a cada chamada recursiva, o 
n\'umero de ocorr\^encias da constante $suc$ \'e decrescido de $1$. Como todo termo de $\mathcal{N}$ \'e finito, temos que
isso \'e suficiente para garantir a termina\c{c}\~ao desta defini\c{c}\~ao recursiva.

Para listas e express\~oes aritm\'eticas n\~ao estamos interessados em interpret\'a-las como algum objeto matem\'atico conhecido
e sim em definir fun\c{c}\~oes sobre elementos destes conjuntos. Como tanto listas quanto express\~oes s\~ao definidos recursivamente,
fun\c{c}\~oes sobre estes elementos tamb\'em ser\~ao definidas por recurs\~ao sobre a sua estrutura. Apresentaremos, como exemplo, 
defini\c{c}\~oes de duas fun\c{c}\~oes sobre listas: uma para calcular o n\'umero de elementos da lista e outra para concatenar duas listas.

\begin{Definition}[Calculando o n\'umero de elementos de uma lista]
Considere a seguinte fun\c{c}\~ao, $length$, que a partir de uma lista de elementos produz como resultado um valor $n\in\mathbb{N}$ que
corresponde ao n\'umero de elementos da lista. Temos que a fun\c{c}\~ao $length$ possuir\'a como dom\'inio o conjunto $\textit{List }\mathcal{T}$
e como contra-dom\'inio o conjunto $\mathbb{N}$.
\[
\begin{array}{lclr}
  length\,\,[\,] & = & 0 & (1)\\
  length\,\,t :: ts & = & 1 + length\,\, ts & (2)
\end{array}
\]
A defini\c{c}\~ao de $length$ constitui uma fun\c{c}\~ao pois: 1) $length$ \'e total, pois \'e definida para cada uma das regras que formam
a sintaxe de listas e; 3) termina sempre, pois, a cada passo da execu\c{c}\~ao da fun\c{c}\~ao $length$ o primeiro elemento da lista \'e
``descartado'' na chamada recursiva.
\end{Definition}

\begin{Example}
Visando exemplificar a defini\c{c}\~ao anterior, considere a tarefa de calcular o n\'umero de elementos da seguinte lista de valores
booleanos: $T :: (F :: (T :: [\,]))$. A execu\c{c}\~ao de $length\,\,T :: (F :: (T :: [\,]))$ \'e apresentada abaixo:
\[
\begin{array}{lcl}
length\,\,T :: (F :: (T :: [\,])) & = & \\
1 + length\,\,(F :: (T :: [\,]))  & = & \{\text{pela equa\c{c}\~ao }(2)\}\\
1 + (1 + length\,\,(T :: [\,]))  & = & \{\text{pela equa\c{c}\~ao }(2)\}\\
1 + (1 + (1 + length\,\,[\,]))  & = & \{\text{pela equa\c{c}\~ao }(2)\}\\
1 + (1 + (1 + 0))  & = & \{\text{pela equa\c{c}\~ao }(1)\}\\
3                  &   & 
\end{array}
\]
\end{Example}

Note que a execu\c{c}\~ao simplesmente reescreve a express\~ao de acordo com as equa\c{c}\~oes que definem a fun\c{c}\~ao $length$, ou seja, por
exemplo, o resultado de executar $length\,\,T :: [\,]$  \'e $1 + length [\,]$, de acordo com a equa\c{c}\~ao $(2)$ de $length$.

A opera\c{c}\~ao de concatenar duas listas consiste em formar uma nova lista que consiste da segunda justaposta ao final da primeira. Por exemplo,
o resultado de concatenar a lista $T :: F :: [\,]$ com a lista $F :: F :: [\,]$ \'e a lista $T :: F :: F :: F :: [\,]$.

\begin{Definition}[Concatena\c{c}\~ao de listas]
 A defini\c{c}\~ao recursiva seguinte calcula a concatena\c{c}\~ao de duas listas fornecidas como par\^ametro. 
 \[
  \begin{array}{lclr}
    [\,] \text{ ++ } ys & = & ys & (1)\\
    (x :: xs) \text{ ++ } ys & = & x :: (xs \text{ ++ } ys) & (2)
  \end{array}
  \]
Intutitivamente, a concatena\c{c}\~ao \'e definida sobre a estrutura sint\'atica da lista fornecida como primeiro par\^ametro. A equa\c{c}\~ao 
$(1)$ especifica que se a primeira lista \'e igual a $[\,]$, ent\~ao o resultado da concatena\c{c}\~ao \'e a segunda lista. Por sua vez,
a equa\c{c}\~ao $(2)$ diz que caso a primeira lista n\~ao seja vazia, ent\~ao o resultado \'e inserir o primeiro elemento no in\'icio da lista
resultante de se concatenar a cauda da primeira lista com a segunda.
\end{Definition}

\begin{Example}
Apresentaremos, passo a passo, a execu\c{c}\~ao da fun\c{c}\~ao de concatena\c{c}\~ao para as listas $T :: F :: [\,]$ e $F :: F :: [\,]$.
\[
\begin{array}{lcl}
(T :: F :: [\,]) \text{ ++ } (F :: F ::[\,]) & = & \\
T :: ((F :: [\,]) \text{ ++ } (F :: F ::[\,])) & = & \text{pela equa\c{c}\~ao }(2)\\
T :: (F :: ([\,] \text{ ++ } (F :: F ::[\,])) & = & \text{pela equa\c{c}\~ao }(2)\\
T :: (F :: (F :: F ::[\,])) & = & \text{pela equa\c{c}\~ao }(1)\\
\end{array}
\]
\end{Example}

As defini\c{c}\~oes apresentadas nesta se\c{c}\~ao s\~ao todas pass\'iveis de implementa\c{c}\~ao em qualquer linguagem de programa\c{c}\~ao
funcional. Neste texto, optaremos pelo assistente de provas Coq para este fim. A se\c{c}\~ao \ref{cap1:coq} apresenta, de maneira suscinta, 
os conceitos necess\'arios de Coq para descri\c{c}\~ao de sintaxe, sem\^antica e fun\c{c}\~oes recursivas sobre a estrutura sint\'atica de termos.

\subsection{Exerc\'icios}

\begin{enumerate}
  \item Apresente uma defini\c{c}\~ao recursiva que calcula o valor de uma express\~ao aritm\'etica (defini\c{c}\~ao \ref{def:arithexp}).
        Sua solu\c{c}\~ao deve possuir como dom\'inio o conjunto $\mathcal{E}$ e como contra-dom\'inio o conjunto $\mathbb{N}$. Para isso, 
        interprete as constantes \textit{plus} e \textit{times} como as opera\c{c}\~oes de adi\c{c}\~ao e multiplica\c{c}\~ao, respectivamente; e
        a constante \textit{const n} deve ser interpretada como um valor num\'erico pertencente ao conjunto dos n\'umeros naturais, $\mathbb{N}$.
  \item A defini\c{c}\~ao apresentada por voc\^e no item anterior constitui uma fun\c{c}\~ao? Justifique em termos dos conceitos de totalidade
        e termina\c{c}\~ao, apresentados na se\c{c}\~ao \ref{cap1:sem}.
\end{enumerate}

\section{Introdu\c{c}\~ao ao Assistente de Provas Coq}\label{cap1:coq}

\section{Notas Bibliogr\'aficas}