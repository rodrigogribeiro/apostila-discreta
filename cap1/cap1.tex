\chapter{L\'ogica Proposicional}

A l\'ogica prov\^e um ferramental para o racioc\'inio sobre matem\'atica, algoritmos
e circuitos digitais. A aplicabilidade da l\'ogica permeia diversas \'areas da computa\c{c}\~ao,
citaremos aqui apenas alguns exemplos:

\begin{itemize}
  \item \textbf{Engenharia de Software}: considera-se uma boa pr\'atica especificar um sistema antes 
        de iniciar a sua codifica\c{c}\~ao. Uma das v\'arias t\'ecnicas de especifica\c{c}\~ao de 
        software \'e o uso de l\'ogica.
  \item \textbf{Aplica\c{c}\~oes de Miss\~ao Cr\'itica}: dizemos que uma determinada aplica\c{c}\~ao 
        \'e de miss\~ao cr\'itica se essa est\'a relacionada a algum risco (de vida, elevados preju\'izos financeiros, etc.).
        Em tais aplica\c{c}\~oes, a utiliza\c{c}\~ao de testes para garantir o funcionamento adequado 
        n\~ao \'e suficiente. O que espera-se \'e uma prova de corretude do programa em quest\~ao, isso \'e uma
        demonstra\c{c}\~ao de que o programa comporta-se de acordo com sua especifica\c{c}\~ao em todas as
        situa\c{c}\~oes poss\'iveis. A l\'ogica \'e a fundamenta\c{c}\~ao matem\'atica de demonstra\c{c}\~oes de
        corre\c{c}\~ao de programas.
   \item \textbf{Recupera\c{c}\~ao de informa\c{c}\~ao}: em m\'aquinas de busca para Web, utiliza-se l\'ogica para
         especificar como propriedades que classificam uma determinada p\'agina como relevante ou n\~ao com base
         em seu conte\'udo.
   \item \textbf{Circuitos Digitais e Arquitetura de Computadores:} l\'ogica \'e a linguagem utilizada para descrever
         sinais produzidos e recebidos como entrada por componentes eletr\^onicos. Um problema comum no projeto de
         circuitos eletr\^onicos \'e determinar uma vers\~ao equivalente, por\'em mais eficiente, de um circuito l\'ogico.
         T\'ecnicas para solu\c{c}\~ao desse problema s\~ao baseadas em algoritmos eficientes para o processamento de
         f\'ormulas da l\'ogica.
   \item \textbf{Bancos de dados}: um recurso fundamental de qualquer sistema gerenciador de bancos de dados \'e uma linguagem
         simples e expressiva para recuperar informa\c{c}\~oes nele armazenadas. A utiliza\c{c}\~ao de recursos baseados em 
         l\'ogica \'e a chave para a expressividade de linguagens para consultas a bancos de dados.       
\end{itemize}

Al\'em das \'areas citadas anteriormente, a l\'ogica \'e fundamental no estudo e no projeto de linguagens de programa\c{c}\~ao
e da teoria de computabilidade.

Neste cap\'itulo, discutiremos as dificuldades presentes na utiliza\c{c}\~ao do Portugu\^es

In this chapter, we discuss the difficulties with informal logical reasoning in
English, and we show how to avoid those difficulties with formal logic. There
are several different kinds of formal logic, and for now we will consider just the
simplest one, called propositional logic. After looking at the language of propo-
sitional logic, we will consider in detail three completely different mathematical
systems for reasoning formally about propositions: truth tables, natural deduc-
tion, and Boolean algebra.

