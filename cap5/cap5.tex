\chapter{Teoria de Conjuntos}\label{cap5}

\epigraph{Ninguém deveria nos expulsar do paraíso que Cantor criou.}{David Hilbert, Matemático Alemão sobre a Teoria de
  Conjuntos criada por Georg Cantor.}

\section{Motivação}

De maneira simplista, pode-se dizer que o alicerce fundamental da
matemática é a teoria de conjuntos. Isto se torna mais e mais evidente
a medida que você avança por cursos mais avançados de matemática, já
que a teoria de conjuntos é uma linguagem projetada para descrever e
explicar todos os tipos de estruturas matemáticas.

Em se tratando de computação, a teoria de conjuntos possui um papel
importante no projeto de estruturas de dados e bancos de
dados. Primeiramente, diversas estruturas
eficientes são implementações de um tipo abstrato de dados que
define operações sobre conjuntos. Por sua vez, toda a teoria de bancos
de dados relacionais é baseada em operações básicas sobre conjuntos.

O objetivo deste capítulo é apresentar a teoria de conjuntos e como
esta pode ser utilizada para descrever propriedades de objetos
matemáticos.

\section{Introdução aos Conjuntos}

Não apresentaremos uma definição formal do que é um conjunto. Isto se
deve ao fato de que  a teoria de conjuntos foi concebida com o intuito
de ser a fundamentação teórica de toda a matemática. Isto é, em
princípio, todos os objetos matemáticos são definidos em termos de
conjuntos.

Conjuntos nada mais são que uma coleção de objetos denominados
\emph{elementos}. Porém, existem algumas restrições para considerarmos
uma coleção de objetos um conjunto. A primeira diz respeito a
\emph{ordem}. Em um conjunto a ordem dos elementos é irrelevante. A
segunda é sobre a \emph{multiplicidade}. Esta especifica que em um
conjunto qualquer há somente uma ocorrência de um certo valor, isto é,
não é permitido que um elemento apareça mais de uma vez em um mesmo
conjunto.

Uma vez que elementos podem ocorrer uma única vez em um conjunto,
podemos dizer que a operação de determinar se um elemento está ou não
em um conjunto possui um valor lógico (isto é, verdadeiro ou
falso). Se $A$ é um conjunto e $x$ um elemento, representamos por $x
\in A$ o fato de $x$ ser um elemento do conjunto $A$. Representamos que $x$ não
é um elemento de $A$ por $x \not\in A$. Note que a seguinte
equivalência é verdadeira: $x\not\in A \equiv \neg (x \in A)$.

Denominamos por \emph{cardinalidade} ou tamanho o número de elementos
de um conjunto. Se $A$ é um conjunto, representamos por $|A|$ o número
de elementos de $A$.

Existe um único conjunto $A$ tal que $|A| = 0$. Este é conhecido como
conjunto vazio e é representado como $\{\}$ ou $\emptyset$.

Adotaremos, como convenção, que conjuntos
serão sempre representados por letras maiúsculas e elementos por
letras minúsculas.


\section{Descrevendo Conjuntos}

Existem diversas maneiras para se descrever conjuntos. Apresentaremos,
de maneira suscinta três maneiras: enumeração, \emph{set
  comprehension}\footnote{Infelizmente, não conheço uma tradução para
  este termo. Por isso, mantive o nome original.} e por recursão.

\subsection{Enumeração}

Definimos um conjunto por enumeração simplesmente listando seus
elementos. Este é um método conveniente para conjuntos finitos que
possuam poucos elementos.

O exemplo a seguir mostra alguns conjuntos
definidos por enumeração.

\begin{Example}
Abaixo apresentamos alguns conjuntos definidos por enumeração:
\[
\begin{array}{lcl}
   V & = & \{a,e,i,o,u\} \\
   P & = & \{\text{arara, pelicano, pardal}\}\\
   X & = & \{2,4,6,8\} \\
   J & = & \{\}\\
  L & = & \{1, \{1\}\} \\
\end{array}
\]
A cardinalidade de cada um deles é:

\[
\begin{array}{lcl}
   |V| & = & 5 \\
   |P| & = & 3\\
   |X| & = & 4 \\
   |J| & = & 0\\
  |L| & = &2 \\
\end{array}
\]
\end{Example}

\subsection{Set Comprehension}

A notação de set comprehension\footnote{Manteremos o nome sem tradução
    por não conhecer um termo em língua portuguesa para este tipo de
    notação matemática.} permite-nos especificar um conjunto em
termos de uma propriedade que descreve quais são os elementos deste.
De maneira simples, temos que um set comprehension é
representado da seguinte maneira:
\[\{x  \in X\,|\,p(x)\}\]
em que $x$ é uma variável  (ou uma expressão), $X$ um conjunto e $p(x)$ é uma fórmula
da lógica de predicados. Esta maneira de descrever conjuntos é útil
para descrever conjuntos com muitos elementos ou infinitos.

Podemos caracterizar a pertinência a um conjunto definido usando set
comprehension utilizando a seguinte equivalência:

\[y\in\{x\in X \,|\, p(x)\} \equiv y\in X \land p(y)\]

O leitor atento deve ter percebido que a equivalência anterior nada
mais é que a aplicação da substituição $[x\mapsto y]$ a fórmula
especificada no set comprehension.

\begin{Example}
Considere a tarefa de representar os conjuntos de todos os números naturais
pares e de todos os números naturais múltiplos de 3. Poderíamos
representar estes conjuntos da seguinte maneira:
\[
\begin{array}{lcr}
P & = &\{0,2,4,6,...\}\\
T & = & \{0,3,6,9,...\}\\
\end{array}
\]
Apesar da estrutura parecer óbvia, o uso de ``...'' deve ser evitado
por este permitir ambiguidades na interpretação de um conjunto. Sem
saber que o conjunto $T$ representa os números naturais múltiplos de
3, como saber se $173$ pertence ou não a este conjunto?

Para evitar este tipo de ambiguidades, podemos utilizar set
comprehensions para definir conjuntos infinitos de maneira precisa. Os
conjuntos anteriores podem ser representados da seguinte maneira:
\[
\begin{array}{lcr}
P & = &\{x\in\mathbb{N}\,|\,\exists y. y\in \mathbb{N}\land x =
2y\}\\
T & = & \{x\in\mathbb{N}\,|\ \exists y. y\in\mathbb{N}\land x =
3y\}\\
\end{array}
\]
Note que como utilizarmos uma fórmula da lógica de predicados para
descrever elementos de um conjunto não há margem para interpretações
ambíguas.
\end{Example}

Devemos sempre especificar qual o conjunto de
``origem''\footnote{Considera-se como conjunto ``origem'' do set
  comprehension $\{x\in X\,|\,p(x)\}$ o conjunto $X$.} dos elementos
para os quais estamos definindo um conjunto utilizando set comprehension.
A não especificação do conjunto de origem permite a formulação de
paradoxos, como o conhecido paradoxo de Russell, descoberto por
Bertrand Russell no início do século XX.

\subsubsection{O Paradoxo de Russell}

Antes de apresentar o paradoxo de Russell formalmente, é útil
analisá-lo em um contexto mais simples, porém, equivalente.

\begin{Example}
Considere o seguinte problema:

\begin{quote}
``Considere uma cidade em que existe apenas um barbeiro e que este faz a
barba de todos que não fazem a própria barba. O barbeiro faz sua
própria barba?''
\end{quote}

Após refletir uma pouco sobre esta sentença, percebemos que esta é um
paradoxo, pois:
\begin{itemize}
  \item Se o barbeiro não faz a própria barba, ele deveria fazê-la, já
    que ele faz a barba apenas de quem não faz a própria barba.
  \item Porém se ele faz a própria barba, pela definição, ele não
    deveria fazê-la.
\end{itemize}

Ou seja, a sentença sobre o barbeiro desta cidade é um paradoxo.
\end{Example}

Russell percebeu que a definição usando set comprehension poderia
gerar um paradoxo similar ao apresentado no exemplo
anterior. A demonstração deste paradoxo é apresentada a seguir.

Seja $\mathcal{S}$ o conjunto de todos os conjuntos que não
são elementos de si próprios, isto é:
\[\mathcal{S} =\{X\,|\,X\not\in X\}\]
Evidentemente, temos que $\mathcal{S} \in \mathcal{S}$ ou $\mathcal{S}
\not\in \mathcal{S}$. Considere os seguintes casos:
\begin{itemize}
  \item Caso $\mathcal{S} \in \mathcal{S}$: Se $\mathcal{S} \in
    \mathcal{S}$, pela definição de $\mathcal{S}$, temos que
    $\mathcal{S} \not\in \mathcal{S}$, o que constitui uma contradição.
  \item Caso $\mathcal{S} \not\in \mathcal{S}$: Logo, pela definição
    de $\mathcal{S}$, temos que $\mathcal{S} \in \mathcal{S}$, o que
    constitui uma contradição.
\end{itemize}
