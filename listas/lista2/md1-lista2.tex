\documentclass[11pt,a4paper]{report}

\usepackage[brazil]{babel}
\usepackage[latin1]{inputenc}
\usepackage{amsmath}
\usepackage{amsfonts}
\usepackage{fullpage}
%\usepackage[linesnumbered, vlined]{algorithm2e}

\newcounter{conta}
 
\begin{document}
 
 \hfill DECSI - UFOP \\
{\it Matem\'atica Discreta}
 \hfill $\mbox{1}^{\mbox{\underline{o}}}$ semestre de 2014 \\
Professor: \parbox[t]{14cm}{Rodrigo Geraldo Ribeiro \\
                     e-mail: rodrigo@decsi.ufop.br}
 
\noindent {\bf Lista de Exerc\'icios 2} \hfill {\bf Tema: L\'ogica de Predicados}
 
\vspace*{3mm}
\begin{enumerate}
  	\item Expresse as seguintes frases utilizando l\'ogica de predicados. 
	      Para isso, crie predicados, fun\c{c}\~oes e constantes do dom\'inio
	      de interpreta\c{c}\~ao que julgar adequados.
	\begin{enumerate}
		\item Quem faz exerc\'icios tem melhor qualidade de vida.
		\item Alunos n\~ao gostam de fazer provas.
		\item Nem tudo que reluz \'e ouro.
		\item Quem conhece Godofredo o adora.
		\item N\~ao conhe\c{c}o quem n\~ao odeie as brincadeiras de Eud\'esio.
		\item Ningu\'em visita Hermengarda, a menos que ela esteja af\^onica.
	\end{enumerate}
	\item Prove os seguintes sequentes usando dedu\c{c}\~ao natural:
	\begin{enumerate}
		\item $\{\forall x. (p(x)\rightarrow q(x))\}\vdash(\forall x.\neg q(x))\rightarrow(\forall x.\neg p(x))$
		\item $\{\forall x. (p(x)\rightarrow \neg q(x))\}\vdash\neg (\exists x. (p(x)\land q(x)))$
		\item $\{\forall x.(a(x)\rightarrow (b(x)\lor c(x))),\forall x.\neg b(x)\}\vdash(\forall x. a(x))\rightarrow 
		       (\forall x. c(x))$
		\item $\{\exists x. (p(x)\land q(x)), \forall x. (p(x)\rightarrow r(x))\}\vdash\exists (r(x)\land q(x))$
		\item $\{\forall x. p(a,x,x), \forall x.\forall y.\forall z. p(x,y,z)\rightarrow p(f(x),y,f(z))\}\vdash p(f(a),a,f(a))$
	\end{enumerate}
   \end{enumerate}
\end{document}
