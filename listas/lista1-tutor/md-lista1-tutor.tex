\documentclass[11pt,a4paper]{report}

\usepackage[brazil]{babel}
\usepackage[latin1]{inputenc}
\usepackage{amsmath}
\usepackage{amsfonts}
\usepackage{fullpage}

\newcounter{conta}
 
\begin{document}
 
 \hfill DECSI - UFOP \\
{\it Matem\'atica Discreta}
 \hfill $\mbox{1}^{\mbox{\underline{o}}}$ semestre de 2014 \\
Professor: \parbox[t]{14cm}{Rodrigo Geraldo Ribeiro \\
                     e-mail: rodrigo@decsi.ufop.br}
 
\noindent {\bf Lista de Exerc\'icios} \hfill {\bf Tema: L\'ogica de
  Primeira Ordem}
 
\vspace*{3mm}
\begin{enumerate}
   \item Sejam $A,B,C,D,E,F$ as seguintes proposi\c{c}\~oes l\'ogicas:
   \[
      \begin{array}{ccl}
        A & = & \text{Todos os homens s\~ao mortais}\\
        B & = & \text{Os pol\'iticos s\~ao honestos}\\
        C & = & \text{A verdade \'e bela}\\
        D & = & \text{A beleza \'e verdadeira}\\
        E & = & \text{Existem infinitos n\'umeros primos}\\
        F & = & \text{A Terra \'e plana}\\
      \end{array}
   \]
   Represente as seguintes sente\c{c}as como f\'ormulas da l\'ogica proposicional.
   \begin{enumerate}
      \item Se existem infinitos n\'umeros primos ent\~ao a verdade \'e bela.
      \item Se os pol\'iticos s\~ao honestos ent\~ao a Terra \'e plana e n\~ao \'e 
            verdade que todos os homens s\~ao mortais.
      \item A verdade n\~ao \'e bela somente se os pol\'iticos n\~ao s\~ao honestos.
   \end{enumerate}
   \item Construa tabelas verdade para as f\'ormulas seguintes e classifique cada uma delas
         como sendo tautologia, contradi\c{c}\~ao ou contingente.
   \begin{enumerate}
      \item $P\lor Q \rightarrow Q \lor P$
      \item $((P\land Q)\lor (P\land R))\leftrightarrow(P\land (Q\lor R))$
      \item $(P\rightarrow Q) \land P\land \neg Q$
      \item $(P \rightarrow Q)\land \neg P \rightarrow Q$
   \end{enumerate}
   \item No projeto de circuitos eletr\^onicos, conectivos l\'ogicos s\~ao representados por componentes denominados portas l\'ogicas.
         Existem portas l\'ogicas que representam os conectivos $\neg, \lor, \land$. Uma porta l\'ogica muito utilizada em circuitos
         \'e a chamada porta ``ou-exclusivo'', representada algebricamente por $A \oplus B$. A f\'ormula $A\oplus B$ \'e verdadeira
         sempre que $A$ e $B$ possuem valores l\'ogicos diferentes.
         Com base no apresentado, responda:
   \begin{enumerate}
     \item Apresente a tabela verdade para $A\oplus B$.
     \item Dizemos que duas f\'ormulas da l\'ogica proposicional $f_1$ e $f_2$ s\~ao equivalentes se $f_1 \leftrightarrow f_2$ \'e uma
           tautologia. Mostre que $A \oplus B$ e $\neg (A \leftrightarrow B)$ s\~ao equivalentes.
     \item Apresente uma f\'ormula na forma normal conjuntiva equivalente a $A\oplus B$ e mostre que estas s\~ao realmente equivalentes.
   \end{enumerate}
   \item Considere a tarefa de modelar um conectivo l\'ogico que possua um
     comportamento similar ao comando ``if'' presente em linguagens de
     programa\c{c}\~ao. A f\'ormula $a\,?\,b\,:\,c$ deve possuir o valor
     l\'ogico $b$ caso $a = T$ e $c$, se $a = F$.
   \begin{enumerate}
     \item Apresente a tabela verdade para esse conectivo.
     \item Encontre uma f\'ormula na forma normal conjuntiva equivalente a $a\,?\,b\,:\,c$.
   \end{enumerate}
   \item Mostre que as seguintes equival\^encias s\~ao verdadeiras utilizando \'algebra booleana.
   \begin{enumerate}
     \item $A \lor (B \land \neg A) \equiv A \lor B$
     \item $(A \lor B)\land (A \lor \neg B) \equiv A$
     \item $\neg (A \rightarrow \neg B) \equiv A \land B$
     \item $A \lor \neg (A \land \neg B) \equiv T$
   \end{enumerate}
   \item Prove os seguintes sequentes utilizando dedu\c{c}\~ao natural.
   \begin{enumerate}
     \item $P,Q,R\vdash P \land (Q \land R)$
     \item $A\land \neg A\vdash F$
     \item $A, A\rightarrow B, B\rightarrow C, C \rightarrow D \vdash D$
     \item $A\rightarrow B, \neg B \vdash \neg A$
     \item $A \lor (B \land C) \vdash (A \lor B) \land (A \lor C)$
     \item $A \rightarrow B \vdash \neg B\rightarrow \neg A$
     \item $B\lor \neg B, A\rightarrow B\vdash \neg A \lor B$
     \item $A \to (B \to C)\,\vdash\,(A \to B) \to (A \to C)$
     \item $\vdash A \lor B \to (\neg A \to B)$
     \item $\vdash A \land B \to \neg (A \to \neg B)$
   \end{enumerate}
\item Traduza as seguintes sente\c{c}as para f\'ormulas da l\'ogica de predicados utilizando os seguintes s\'imbolos
         para representar predicados:
   \[
      \begin{array}{ccl}
        h(x) & = & x \text{ \'e um homem.}\\
        l(x) & = & x \text{ \'e um livro.}\\
        p(x) & = & x \text{ pensa}\\
        g(x,y) & = & x \text{ gosta de }y\\
      \end{array}
   \]
   \begin{enumerate}
     \item Todos os homens pensam.
     \item Todos os homens que pensam, gostam de livros.
     \item Todos os homens gostam de livros, menos Di\'ogenes.
     \item Nenhum livro pensa.
     \item Pelo menos um homem n\~ao pensa.
     \item Alguns homens n\~ao pensam, mas gostam de todos os livros.
   \end{enumerate}
   \item Para cada uma das f\'ormulas a seguir, indique se ela \'e verdadeira ou falsa, 
         quando o universo de discurso \'e cada um dos seguintes conjuntos: $\mathbb{N}$: conjunto
         dos n\'umeros naturais, $\mathbb{Z}$: conjunto dos n\'umeros inteiros e $\mathbb{R}$ conjunto
         dos n\'umeros reais.
         \begin{center}
         \begin{tabular}{|l|c|c|c|}
             \hline
             F\'ormula & $\mathbb{N}$ & $\mathbb{Z}$ & $\mathbb{R}$ \\ \hline
             $\exists x. x^2 = 2$ & & &\\ \hline
             $\forall x. \exists y. x^2 = y$ & & & \\ \hline
             $\forall x. x\neq 0 \rightarrow \exists y. xy = 1$ & & & \\ \hline
             $\exists x. \exists y. (x + 2y^2 = 2) \land (2x + 4y = 5)$ & & & \\
             \hline
         \end{tabular}
         \end{center}
         \item Prove os seguintes sequentes utilizando dedu\c{c}\~ao natural.
         \begin{enumerate}
           \item $\forall x. p(x) \rightarrow q(x) \vdash \forall x. p(x) \rightarrow \forall x. q(x)$
           \item $\exists x. \neg p(x) \vdash \neg \forall x. p(x)$
           \item $\forall x. a(x)\rightarrow b(x) \lor c(x), \forall x. \neg b(x)\vdash \forall x. a(x)\rightarrow\forall x.b(x)\lor c(x)$
         \end{enumerate}
         \item Prove as seguintes equival\^encias utilizando regras alg\'ebricas para l\'ogica de predicados.
         \begin{enumerate}
            \item $\forall x. p(x) \rightarrow \neg q(x) \equiv \neg \exists x. p(x) \land q(x)$
            \item $\neg \forall x.\exists y. r(x,y)\land \neg p(x,y)\equiv\exists x.\forall y.r(x,y)\rightarrow p(x,y)$
         \end{enumerate}
   \end{enumerate}
\end{document}
