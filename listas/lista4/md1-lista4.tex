\documentclass[11pt,a4paper]{report}

\usepackage[brazil]{babel}
\usepackage[latin1]{inputenc}
\usepackage{amsmath}
\usepackage{amsthm}
\usepackage{amsfonts}
\usepackage{fullpage}
%\usepackage[linesnumbered, vlined]{algorithm2e}

\newcounter{conta}
\theoremstyle{definition}
\newtheorem{Definition}{Defini\c{c}\~ao}

\begin{document}
 
 \hfill DECEA - UFOP \\
{\it Matem\'atica Discreta}
 \hfill $\mbox{1}^{\mbox{\underline{o}}}$ semestre de 2014 \\
Professor: \parbox[t]{14cm}{Rodrigo Geraldo Ribeiro \\
                     e-mail: rodrigo@decsi.ufop.br}
 
\noindent {\bf Lista de Exerc\'icios 4} \hfill {\bf Tema: Demonstra\c{c}\~ao
  de Teoremas}
\vspace*{3mm}
\rule{\textwidth}{0.1mm}
\begin{center}
   \textbf{Defini\c{c}\~oes}
\end{center}
\begin{Definition}
   Seja $n\in\mathbb{Z}$. Dizemos que $n$ \'e \textbf{par} se existe
   $k\in\mathbb{Z}$ tal que $n = 2k$; $n$ \'e \textbf{\'impar} se existe
      $k\in\mathbb{Z}$ tal que $n = 2k + 1$.
\end{Definition}
\begin{Definition}
  Sejam $n,d\in\mathbb{Z}$. Dizemos que $n$ \'e \textbf{divis\'ivel} por $d$
  (representado como $d\,\mid\,n$) se existe $k\in\mathbb{Z}$ tal que
  $n = kd$.
\end{Definition}
\begin{Definition}
  Um n\'umero $n\in\mathbb{Z}$ \'e um \textbf{quadrado perfeito} se existe
  $k\in\mathbb{Z}$ tal que $n = k^2$.
\end{Definition}
\begin{Definition}
  Seja $\mathcal{F}$ uma fam\'ilia de conjuntos. Define-se o conjunto $\bigcup
  ! \mathcal{F}$ por \[\bigcup ! \mathcal{F} = \{x\,|\,\exists ! A. A\in
    \mathcal{F}\land x\in A\}\].
\end{Definition}
\rule{\textwidth}{0.1mm}
\begin{enumerate}
        \item Prove que se $n$ \'e \'impar, ent\~ao $3n +9$ \'e par.
        \item Considere o seguinte teorema: Se $n$ \'e par, ent\~ao $n^2$ \'e par.
        \begin{enumerate}
          \item Apresente uma prova direta para esse teorema.
          \item Apresente uma prova pela contrapositiva para esse teorema.
        \end{enumerate}
        \item Prove que se $n$ e $m$ s\~ao n\'umeros \'impares ent\~ao $n + m$
          \'e par.
        \item Prove que se $n$ \'e par e $m$ \'e \'impar ou 
              $n$ \'e \'impar e $m$ \'e par ent\~ao $n + m$ \'e \'impar.
        \item Prove que se a soma de dois n\'umeros inteiros \'e par, sua
          diferen\c{c}a tamb\'em o \'e.
        \item Prove que se $n$ e $m$ s\~ao dois quadrados prefeitos, $nm$
          tamb�m \'e um quadrado perfeito.
        \item Prove que se $n$ \'e um inteiro e $3n + 2$ \'e \'impar, ent\~ao
          $n$ \'e \'impar.
        \item Prove que o produto de dois n\'umeros inteiros consecutivos \'e
          par.
        \item Prove que a soma de dois n\'umeros inteiros consectivos \'e \'impar.
	\item Prove que, se $A\subseteq B$ e $B\subseteq C$, ent\~ao $A\subseteq C$.
	\item Prove que, se $\overline{A}\subseteq\overline{B}$ ent\~ao $B\subseteq{A}$.
	\item Prove que, se $A\subseteq B$ ent\~ao $\mathcal{P}(A)\subseteq\mathcal{P}(B)$.
	\item Prove que, $\mathcal{P}(A)\cap\mathcal{P}(B)=\mathcal{P}(A\cap B)$
	\item Prove que, $\mathcal{P}(A)\cup\mathcal{P}(B)\subseteq\mathcal{P}(A\cup B)$
	\item Prove que, se $(A - B)\cup (B - A) = A\cup B$ ent\~ao $A\cap B =\emptyset$ (isto \'e $A$ e $B$ s\~ao disjuntos).
	\item Prove que, se $A \cup B = A - B$ ent\~ao $B=\emptyset$.
	\item Prove que, se $A\cap B = A$ ent\~ao $A\subseteq B$.
	\item Suponha que $A - B \subseteq C\cap D$ e que $x\in A$. Prove que se $x\not\in D$ ent\~ao $x\in B$.
	\item Suponha que $A\subseteq C$ e que $B$ e $C$ s\~ao disjuntos. Prove que se $x\in A$ ent\~ao $x\not\in B$.
	\item Prove que se $A$ e $B - C$ s\~ao disjuntos, ent\~ao $A\cap B\subseteq C$.
	\item Prove que se $\mathcal{F}$ \'e uma fam\'ilia de conjuntos e $A\in\mathcal{F}$, ent\~ao 
	      $A\subseteq\bigcup\mathcal{F}$.
	\item Prove que $A=\bigcup\mathcal{P}(A)$, para qualquer conjunto
          $A$. 
       \item Prove que se $\emptyset\in\mathcal{F}$ ent\~ao
         $\bigcap\mathcal{F} = \emptyset$.
       \item Prove que para todo n\'umero $x\in\mathbb{R}$, existe um \'unico
         $y\in\mathbb{R}$ tal que $x^2y = x - y$.
       \item Prove que para qualquer fam\'ilia $\mathcal{F}$, $\bigcup
         !\mathcal{F}\subseteq \bigcup\mathcal{F}$.
       \item Seja $P(x)$ uma f\'ormula da l\'ogica de predicados em que $x$ \'e uma vari\'avel livre.
       \begin{enumerate}
         \item Encontre uma f\'ormula da l\'ogica de predicados que represente: ``existem exatamente dois valores de $x$ que fazem $P(x)$ 
               ser verdadeira''.
         \item Baseado na resposta do item anterior, descreva uma estrat\'egia de prova para f\'ormulas ``existem exatamente dois valores de $x$
               que tornam $P(x)$ verdadeiro''.
         \item Utilizando a estrat\'egia de prova criada por voc\^e, mostre que a equa\c{c}\~ao $x^3 = x^2$ possui exatamente duas ra\'izes.
       \end{enumerate}
\end{enumerate}
\end{document}
