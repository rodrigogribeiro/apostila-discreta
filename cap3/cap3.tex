\chapter{L\'ogica de Predicados}

\section{Motivação}

No capítulo anterior, estudamos a lógica proposicional de um ponto de
vista sintático e semântico e, além disso, utilizamos a dedução
natural e álgebra Booleana para verificar consequências e
equivalências lógicas.

Apesar de possuir uma série de aplicações, a lógica proposicional possui
alguns inconvenientes. A seguir apresentamos um exemplo destes
inconvenientes.
\begin{Example}
Considere o seguinte argumento dedutivo:
\begin{quote}
Todo homem é mortal.\\ Sócrates é um homem. \\Logo, Sócrates é mortal.
\end{quote}
De acordo com nossa noção informal de dedução, este parece ser um
argumento válido. Sendo assim, este pode ser representado como um
sequente demonstrável utilizando dedução natural. Porém, ao
representarmos estas sentenças como fórmulas da lógica, podemos
perceber que nenhuma delas possui conectivos lógicos. Logo, todas
estas podem ser consideradas proposições simples, conforme mostramos
na tabela abaixo:

\begin{table}[h]
  \begin{tabular}{|l|c|}
    \hline
    Sentença & Fórmula \\ \hline
    Todo homem é mortal & $A$ \\
    Sócrates é um homem & $B$ \\
    Sócrates é mortal & $C$ \\\hline
  \end{tabular}
  \centering
\end{table}

Utilizando a modelagem apresentada na tabela acima, o sequente
\[
\{A,B\}\,\vdash\,C
\]
representa o argumento dedutivo em questão. Mas, como o leitor já deve
ter percebido, este não é provável utilizando o sistema de dedução
natural apresentado neste texto.
\end{Example}

Na seção \ref{soundcompleteprop} apresentamos que