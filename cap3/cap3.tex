\chapter{L\'ogica de Predicados}

\section{Motivação}

No capítulo anterior, estudamos a lógica proposicional de um ponto de
vista sintático e semântico e, além disso, utilizamos a dedução
natural e álgebra Booleana para verificar consequências e
equivalências lógicas.

Apesar de possuir uma série de aplicações, a lógica proposicional possui
limitações. A seguir apresentamos um exemplo que ilustra este problema.
\begin{Example}
Considere o seguinte argumento dedutivo:
\begin{quote}
Todo homem é mortal.\\ Sócrates é um homem. \\Logo, Sócrates é mortal.
\end{quote}
De acordo com nossa noção informal de dedução, este parece ser um
argumento válido. Sendo assim, este pode ser representado como um
sequente demonstrável utilizando dedução natural. Porém, quando tentamos
representar estas sentenças como fórmulas da lógica, podemos
perceber que nenhuma delas possui conectivos lógicos. Logo, todas
podem ser consideradas proposições simples, conforme mostramos
na tabela a seguir:

\begin{table}[h]
  \begin{tabular}{|l|c|}
    \hline
    Sentença & Fórmula \\ \hline
    Todo homem é mortal & $A$ \\
    Sócrates é um homem & $B$ \\
    Sócrates é mortal & $C$ \\\hline
  \end{tabular}
  \centering
\end{table}

Utilizando a modelagem apresentada na tabela acima, o sequente
\[
\{A,B\}\,\vdash\,C
\]
representa o argumento dedutivo em questão. Mas, como o leitor já deve
ter percebido, este não é provável utilizando o sistema de dedução
natural apresentado neste texto.
\end{Example}

Na seção \ref{soundcompleteprop} apresentamos que o sistema de dedução
natural é completo para a lógica proposicional, desta forma, toda
consequência lógica deve possuir um sequente provável
correspondente. De acordo com uma noção intuitiva de dedução lógica, o
argumento anterior é correto e portanto, deveríamos conseguir
representá-lo como um sequente demonstrável, o que, conforme
apresentado, não é possível.

O problema na modelagem formal deste argumento é que a lógica proposicional não possui expressividade para
representar sentenças que possuam as seguintes formas:
\begin{itemize}
  \item Todo $x$ possui a propriedade $p$.
  \item Algum $x$ possui a propriedade $p$.
\end{itemize}
Tais sentenças possuem, implicitamente, um conjunto sobre o qual a
frase em questão deve ser interpretada como verdadeira ou falsa. No
caso do exemplo anterior, temos que a frase:
\begin{center}
Todo homem é mortal.
\end{center}
implicitamente se refere ao conjunto de todos os seres humanos. Esta
mesma frase poderia ser re-escrita de maneira a tornar o conjunto  de
seres humanos (que está ``implícito'') explícito como:
\begin{center}
Todo elemento do conjunto de seres humanos possui a propriedade ``mortal''.
\end{center}
Para representar sentenças como ``Todo homem é moral'', precisamos de
estender a lógica proposicional de forma que sejamos capazes de
expressar propriedades sobre elementos de um certo conjunto. O
objetivo deste capítulo é estudarmos esta lógica, conhecida como
lógica de predicados ou lógica de primeira ordem.

\section{Introdução à lógica de predicados}

\subsection{Predicados}

\subsection{Quantificadores}

\subsection{Formalizando sentenças}

\section{Sintaxe da lógica de predicados}

\section{Semântica da lógica de predicados}
