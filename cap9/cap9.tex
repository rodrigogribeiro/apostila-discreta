\chapter{Indução Matemática}\label{cap9}

\epigraph{Induction makes you fell guilty for getting something out of
nothing, and it is artificial, but it is one of the greatest ideas of
civilization.}{Helbert S. Wilf, Matemático Norte-americano.}

\section{Motivação}

Na ciência e filosofia usamos, essencialmente, dois tipos de
raciocínio distintos: o dedutivo e indutivo. O raciocínio dedutivo é
governado por leis da lógica e foi tema de grande parte do curso de
matemática discreta. Se certo fato é deduzido usando lógica, este é
irrefutável, visto que, sistemas dedutivos para lógica são corretos e
completos. O raciocínio indutivo, por sua vez, é o que usamos quando
inferimos um padrão de comportamento futuro a partir de experiências
realizadas no passado. Este tipo de raciocínio, apesar de útil em
ciências experimentais, não é de interesse para os objetivos deste
texto. Em especial, estamos interessados na chamada indução
matemática, uma técnica de demonstração muito utilizada em diversas
áreas da computação.
O objetivo
deste capítulo é apresentar esta técnica de prova e como esta é usada
para demonstração de diversos fatos em matemática e em ciência da
computação.


\section{Introdução à indução matemática}

A técnica de indução matemática pode ser utilizada para demonstrar
conclusões que possuam a seguinte estrutura:
\[
\forall n. n\in \mathbb{N}\to P(n)
\]
Evidentemente, para demonstrar a fórmula anterior devemos provar que
a propriedade $P$ é verdadeira para cada um dos valores do conjunto
\[\mathbb{N} = \{0,1,2,3...\}\]
Como existem infinitos valores em $\mathbb{N}$, não podemos
simplesmente verificar se a propriedade em questão é verdadeira para
cada um destes valores. Então, como demonstrar tais propriedades? A
chave da indução matemática está na própria estrutura do conjunto de
números naturais. Note que para listar todos os valores de
$\mathbb{N}$, tudo o que temos que fazer é iniciar com $0 \in
\mathbb{N}$ e repetidamente somar $1$, produzindo assim, um novo
número natural. Assim, para mostrar que todos os números naturais
possuem uma certa propriedade $P$, basta:
\begin{itemize}
    \item Mostrar que $0$ possui a propriedade $P$, isto é, $P(0)$ é
      verdadeiro.
    \item Mostrar que sempre que um número natural $n$ possuir a
      propriedade $P$ então o sucessor de $n$, $n + 1$, também
      possuirá essa propriedade. Isto é, devemos provar que $\forall
      n. n\in\mathbb{N} \land P(n) \to P(n+1)$.
\end{itemize}
Exatamente esta observação motiva a seguinte estratégia de prova.
\begin{ProofStrategy}[Para provar uma conclusão da forma $\forall
  n. n\in\mathbb{N} \to P(n)$.]\label{simpleinduction}
Prove que a seguinte fórmula é verdadeira.
\[
P(0)\land\forall n.n\in\mathbb{N}\land P(n)\to P(n+1)
\]
\begin{flushleft}
\textbf{Texto}:
Caso Base: [Prova de $P(0)$].\\
Passo Indutivo:[Prova de $\forall n. n\in\mathbb{N}\land P(n)\to P(n+1)$].
\end{flushleft}
\end{ProofStrategy}
Porém, como somente a demonstração destes dois passos pode comprovar
que $\forall n. n\in\mathbb{N} \to P(n)$? A idéia é bastante
simples. Note que ao usarmos indução matemática, provamos as seguintes
fórmulas:
\begin{itemize}
    \item $P(0)$
    \item $\forall n. n\in \mathbb{N} \land P(n) \to P(n+1)$
\end{itemize}
Observe que temos que para qualquer número natural n, $P(n) \to
P(n+1)$. Como esta fórmula é verdadeira para todo $n\in\mathbb{N}$,
temos que esta é verdadeira também para $n = 0$. Eliminando o
quantificador universal substituindo $n$ por $0$, obtemos
\[
0 \in \mathbb{N} \land P(0) \to P(1)
\]
Porém, é óbvio que $0 \in \mathbb{N}$ e como provamos que $P(0)$, por
eliminação da implicação podemos concluir que $P(1)$ também é
verdadeiro. Repetindo este processo para $n = 1$, temos que a seguinte
implicação é obtida a partir da eliminação do quantificador universal:
\[
1 \in\mathbb{N}\land P(1) \to P(2)
\]
o que nos permite deduzir que $P(2)$ é verdadeiro. Note que podemos
repetir esse processo para concluir que $P$ é verdadeira para qualquer
$n \in\mathbb{N}$. A seguir, apresentaremos um exemplo simples de uma
propriedade provável por indução matemática.

\begin{Theorem}
Para todo $n \in \mathbb{N}$, temos que $\sum_{k = 0}^n2^k =
2^{n + 1} - 1$
\end{Theorem}
\begin{Commentary}
Note que este teorema pode ser expresso pela seguinte fórmula:
\[
\forall n. n\in\mathbb{N} \land \sum_{k = 0}^n2^k = 2^{n+1} - 1
\]
em que a propriedade $P(n)$ é
\[
\sum_{k = 0}^n2^k = 2^{n+1} - 1
\]
logo, de acordo com a estratégia de prova \ref{simpleinduction},
devemos provar as seguintes fórmulas:
\begin{itemize}
    \item $P(0)$
    \item $\forall n. n\in\mathbb{N}\land P(n) \to P(n+1)$
\end{itemize}
em que $P(0)$ é dada por
\[
\sum_{k = 0}^02^k = 2^{0+1} - 1
\]
que é facilmente demonstrada como verdadeira pela seguinte equação:
\[
\begin{array}{lcl}
\sum_{k = 0}^02^k & = \\
2^0                       & = & \{\text{pela def. da noção }\Sigma\}\\
1                          & = &\\
2 - 1                    & = & \\
2^{0 + 1} - 1
\end{array}
\]
Logo, $P(0) = \sum_{k = 0}^02^k = 2^{0+1} - 1$ é verdadeiro.

No próximo passo, temos que demonstrar o passo indutivo, que é dado
pela seguinte fórmula:
\[
\forall n. n\in\mathbb{N}\land \left(\sum_{k = 0}^n 2^k = 2^{n + 1} -
  1\right) \to \left(\sum_{k = 0}^{n + 1} 2^k = 2^{n + 2} -
  1\right)
\]
Para demonstrar o passo indutivo, supomos $n\in\mathbb{N}$ arbitrário
e que $\sum_{k = 0}^n 2^k = 2^{n + 1}  - 1$. Usualmente, damos o nome
de hipótese de indução a suposição de que a propriedade que desejamos
provar é verdadeira para um número $n$ qualquer, isto é que $P(n)$ é
verdadeira.

Para concluir a prova, resta mostrar que a seguinte igualdade é
verdadeira:
\[
\sum_{k = 0}^{n+1} 2^k = 2^{n + 2} - 1
\]
que é facilmente demonstrável usando a hipótese de indução e
álgebra. A demonstração desta equação é apresentada a seguir.
\[
\begin{array}{lcl}
\sum_{k = 0}^{n + 1}2^k & = & \\
\sum_{k = 0}^{n}2^k + 2^{n + 1} & = &\{\text{pela definição de
}\Sigma\}\\
2^{n + 1} - 1 + 2^{n+1} & = &\{\text{pela hipótese de indução}\}\\
2^{n+2} -1
\end{array}
\]
A seguir, apresentamos a versão final do texto desta demonstração.
\end{Commentary}
\begin{proof}
\verb| |\\
\begin{enumerate}
  \item[\ ]Caso Base: Para $n = 0$, temos:
\[
\begin{array}{lcl}
\sum_{k = 0}^02^k & = \\
2^0                       & = & \{\text{pela def. da notação }\Sigma\}\\
1                          & = &\\
2 - 1                    & = & \\
2^{0 + 1} - 1
\end{array}
\]
  \item[\ ]Passo indutivo: Suponha $n$ arbitrário. Suponha $n
    \in\mathbb{N}$ e que $\sum_{k = 0}^n 2^k = 2^{n + 1} - 1$. Temos:
\[
\begin{array}{lcl}
\sum_{k = 0}^{n + 1}2^k & = & \\
\sum_{k = 0}^{n}2^k + 2^{n + 1} & = &\{\text{pela definição de
}\Sigma\}\\
2^{n + 1} - 1 + 2^{n+1} & = &\{\text{pela hipótese de indução}\}\\
2^{n+2} -1
\end{array}
\]
\end{enumerate}
\end{proof}
Apresentaremos mais alguns exemplos de demonstração por indução.
\begin{Theorem}
Para todo $n\in\mathbb{N}$, $3\,|\,(n^3 - n)$.
\end{Theorem}
\begin{Commentary}
Novamente, utilizaremos indução matemática. Note que o enunciado do
teorema é dado por:
\[
\forall n. n\in\mathbb{N} \to 3\,|\,(n^3 - n)
\]
em que a propriedade a ser demonstrada é
\[
3\,|\,(n^3 - n)
\]
Note que $3\,|\,(n^3 - n)$ é na verdade uma fórmula envolvendo um
quantificador existencial.
\[
3\,|\,(n^3 - n) \equiv \exists k. k\in\mathbb{N}\land n^3 - n = 3k
\]
Logo, temos que mostrar a veracidade das seguintes fórmulas, para
demonstrar o teorema usando indução:
\begin{itemize}
  \item $3\,|\,(0^3 - 0)$
  \item $\forall n. n\in\mathbb{N}\land 3\,|\,(n^3 - n)\to
    3\,|\,((n+1)^3 - (n + 1))$
\end{itemize}
A demonstração de $3\,|\,(0^3 - 0)$ envolve provar a seguinte fórmula
\[
\exists k. k\in\mathbb{N}\land 3k = 0
\]
que é evidentemente verdadeira, basta fazer que $k = 0$.

Para o passo indutivo, devemos demonstrar que
\[\forall n. n\in\mathbb{N}\land 3\,|\,(n^3 - n)\to
    3\,|\,((n+1)^3 - (n + 1))\]
Iniciamos a demonstração supondo $n\in\mathbb{N}$ arbitrário e que
$3\,|\,(n^3 - n)$ e devemos provar que
\[
\exists k. k\in\mathbb{N}\land (n + 1)^3 - (n+1) = 3k
\]
Logo, devemos encontrar um valor de $k\in\mathbb{N}$ que torne a
fórmula anterior verdadeira. Como $3\,|\,(n^3 - n)$, existe $a \in
\mathbb{N}$ tal que $n^3 - n = 3a$. Para encontrar uma ``dica'' de
qual seria este valor de $k$, vamos desenvolver o polinômio $(n+1)^3 -
(n + 1)$:
\[
\begin{array}{lc}
(n+1)^3 - (n + 1) & =\\
n^3 + 3n^2 + 3n + 1 - (n + 1) & = \\
n^3 + 3n^2 + 3n + 1 - n - 1 & = \\
n^3 -n + 3n^2 + 3n
\end{array}
\]
Porém, pela hipótese de indução, temos que $n^3 - n = 3a$. Com isso,
temos:
\[
\begin{array}{lc}
n^3 -n + 3n^2 + 3n & =\\
3a + 3n^3 + 3n & = \\
3 (a + n^2 + n)
\end{array}
\]
que obviamente é divisível por $3$. Logo, para concluir a
demonstração, basta escolher $k = a + n^2 + n$. O texto final da prova
é apresentado a seguir.
\end{Commentary}
\begin{proof}
\verb| |\\
\begin{enumerate}
  \item[\ ]Caso base: Seja $k = 0$. Temos que $0^3 - 0 = 3.0$, conforme requerido.
  \item[\ ]Passo indutivo: Suponha $n\in\mathbb{N}$ arbitrário e que
    $3\,|\,n^3 - n$. Como $3\,|\,n^3 - n$, existe $a\in\mathbb{N}$ tal
    que $n^3 - n = 3a$. Seja $k = a + n^2 + n$. Temos:
    \[
 \begin{array}{lcl}
   3k & = & \\
   3 (a + n^2 + n) & = & \{\text{por }k = a + n^2 + n\}\\
   3a + 3n^2 + 3n & = & \\
   (n^3 - n) + 3n^2 + 3n & = &\{\text{pela hipótese de indução}\}\\
   n^3 -n + 3n^2 + 3n + 1 - 1 & = & \\
   (n^3 + 3n^2 + 3n + 1) - n -1 & = & \\
   (n + 1)^3 - (n +1)
 \end{array}
    \]
    Logo, $3\,|\,(n+1)^3-(n+1)$, conforme requerido.
\end{enumerate}
\end{proof}
Terminamos essa seção com um exemplo de demonstração por indução que
não envolve equações, mas sim inequações.
\begin{Theorem}
Para todo $n\in\mathbb{N},\:n\geq 5$ temos que $2^n > n^2$.
\end{Theorem}
\begin{proof}
\verb| |\\
\begin{enumerate}
  \item[\ ]Caso base: Para $n = 5$, temos que $2^5 > 25$, conforme
    requerido.
  \item[\ ]Passo indutivo: Suponha $n\in\mathbb{N}$ arbitrário e que
    $2^n > n^2$. Temos:
   \[
\begin{array}{lcl}
    2^{n+1} & = &\\
    2^n + 2^n & > &\\
   n^2 + n^2  & > &\{\text{pela hipótese de indução.}\}\\
   n^2 + 2n + 1 & = &\\
   (n+ 1)^2
\end{array}
  \]
  Logo, $2^{n+1} > (n + 1)^2$, conforme requerido.
\end{enumerate}
\end{proof}
\begin{Commentary}
A demonstração anterior utiliza praticamente a mesma estrutura das
anteriores, com duas diferenças:
\begin{enumerate}
  \item O caso base deve ser para $n = 5$, uma vez que o teorema é
    válido para valores $n \geq 5$.
  \item No passo indutivo, provamos a desigualdade desejada como uma
    sequência de igualdades / desigualdades. Este tipo de
    demonstração, comum em matemática, é válido para qualquer relação
    binária transitiva (note que tanto $>$, quanto $=$ são transitivas).
\end{enumerate}
\end{Commentary}

\subsection{Exercícios}

\begin{enumerate}
	\item Prove os seguintes teoremas.
	\begin{enumerate}
		\item Para todo $n\in\mathbb{N}$, $\sum_{i=0}^{n}3^i = \frac{3^{n + 1} - 1}{2}$
		\item Para todo $n\in\mathbb{N}$, $\sum_{i=0}^{n}i^{2} =
                  \frac{n(n+1)(2n + 1)}{6}$
                \item Para todo $n\in\mathbb{N}$, $\sum_{i=1}^{n}(8i - 5) =
                  4n^2 - n$
                \item Para todo $n\geq 1$, $5\,|\,(n^5-n)$
                \item Para todo $n\geq 1$, $6\,|\,(7^n-1)$
                \item Para todo $n\geq 1$, $6\,|\,(n^3+5n)$
		\item Para todo $n\in\mathbb{N}$, $2\,\mid\,(n^{2} + n)$
	\end{enumerate}
\end{enumerate}

\section{Indução Forte}

Em algumas situações, a suposição de que a propriedade a ser provada é
válida para um número $n \in \mathbb{N}$ não é forte o suficiente para
concluirmos a demonstração.  Para ilustrar esse problema, vamos
apresentar um teorema e tentar prová-lo usando a técnica de indução
vista na seção anterior. Na sequência, apresentaremos a técnica de indução forte
e a justificaremos de maneira informal, como feito para a indução
simples. Finalmente, concluiremos esta seção demonstrando o teorema
utilizado como motivador para a indução forte e mais alguns exemplos
desta técnica de prova.

Utilizaremos o chamado teorema fundamental da aritmética que afirma
que todo número inteiro $n > 1$ é primo ou produto de primos. O
enunciado formal deste teorema é apresentado a seguir:

\begin{Theorem}\label{strong-ind1}
Para todo $n\in\mathbb{N}$, se $n > 1$ então $n$ é primo ou é um
produto de números primos.
\end{Theorem}

Note que o enunciado deste teorema é dado pela seguinte fórmula:
\[
\forall n. n\in\mathbb{N} \to n > 1 \to n\text{ é primo} \lor n\text{
  é produto de primos}
\]
Como esta fórmula possui o formato exigido para demonstrações
usando indução, podemos tentar provar o caso base ($n = 2$) e o passo
indutivo. Porém, no passo indutivo chegamos em um possível ``beco-sem-saída'':
\begin{enumerate}
  \item[\ ]Caso base: Para $n = 2$, temos que $2$ é primo e portanto,
    $2$ é primo ou produto de primos.
  \item[\ ]Passo indutivo. Suponha $n\in\mathbb{N}$, $n > 1$
    arbitrário e que $n$ é primo ou produto de primos. Considere os
    seguintes casos:
    \begin{enumerate}
      \item[\ ] $n$ é primo: [Prova de $n + 1$ é primo ou produto de primos]
      \item[\ ] $n$ é produto de primos: [Prova de $n + 1$ é primo ou produto de primos]
    \end{enumerate}
\end{enumerate}
Note que no caso de $n$ ser primo, não podemos garantir que $n + 1$
será sempre primo ou produto de primos, visto que para $n = 2$, temos
que $n + 1 = 3$ é primo. Por sua vez, se supormos que $n$ é um produto
de primos, temos que $n + 1$ pode ser ou não primo. Então como
concluir esta demonstração?

Note que se um número $n$ não é primo, necessariamente existem $a$ e
$b$ tais que $1 < a,b < n$ e $n = a.b$, em que $a$ e $b$ podem ser ou
não números primos. Se ambos forem primos, temos que $n$ será um
produto de primos. Por sua vez, se  $a$ ou $b$ for um produto de
primos, temos que $n$ também o será. Apesar de correto, não podemos
utilizar este raciocínio em uma prova por indução, pois a hipótese de
indução supõe que a propriedade que desejamos provar é verdadeira para
um valor fixo $n$ qualquer e não para qualquer valor.

Nestas situações em que precisamos de uma hipótese de indução que seja
válida não apenas para um valor, devemos usar a indução forte que nos
permite supor que a propriedade a ser demonstrada é válida para todos
os números naturais menores que um número $n$. O princípio de indução
forte é apresentado na estratégia de prova seguinte.

\begin{ProofStrategy}[Para provar $\forall n. n\in\mathbb{N} \to
  P(n)$]
Prove que a seguinte fórmula é verdadeira:
\[
\forall n. n\in\mathbb{N} \to (\forall k. k\in\mathbb{N} \land k < n
\to P(k)) \to P(n)
\]
\end{ProofStrategy}

O ponto chave de uma demonstração usando indução está em sua hipótese
de indução:

\[\forall k. k\in\mathbb{N} \land k < n \to P(k)\]

que especifica que a propriedade $P$ é verdadeira não para apenas um
número qualquer, mas sim para todos valores $k < n$. Ao observarmos a
fórmula da estratégia de indução forte, notamos que esta não possui
``casos base''. Então, como esta pode ser equivalente a
$\forall n. n\in\mathbb{N} \to  P(n)$? Para a indução convencional,
mostramos que esta é equivalente a $\forall n. n\in\mathbb{N} \to
P(n)$ usando o caso base e uma sequência de eliminações do
quantificador universal e implicações\footnote{Se você não lembra
deste argumento, sugiro ler a seção anterior.}. Para mostrar que
indução forte é uma técnica de prova válida, utilizaremos uma
estratégia similar a usada para indução convencional. Para a fórmula
\[
\forall n. n\in\mathbb{N} \to (\forall k. k\in\mathbb{N} \land k < n
\to P(k)) \to P(n)
\]
ser verdadeira, esta deverá o ser para todos os valores de $n
\in\mathbb{N}$. Logo, para $n = 0$, esta também será
verdadeira. Substituindo $n$ por $0$ obtemos:
\[
0\in\mathbb{N}\to (\forall k. k\in \mathbb{N} \land l < 0 \to P(k))
\to P(0)
\]
Porém, como não existe $k \in \mathbb{N}$ tal que $k < 0$, temos que
$k< 0$ é equivalente a $\bot$. Logo, usando álgebra, temos:
\[
\begin{array}{lc}
0\in\mathbb{N}\to (\forall k. k\in \mathbb{N} \land k < 0 \to P(k))
\to P(0) & \equiv \\
0 \in \mathbb{N} \to (\forall k. k\in\mathbb{N} \land \bot \to
P(k))\to P(0) & \equiv \\
0 \in\mathbb{N} \to (\forall k. \bot \to P(k))\to P(0) & \equiv \\
0 \in\mathbb{N} \to (\top \to P(0)) & \equiv \\
0 \in \mathbb{N} \to (\neg \top \lor P(0)) & \equiv \\
0 \in \mathbb{N} \to (\bot \lor P(0)) & \equiv \\
0 \in \mathbb{N} \to P(0) &  \\
\end{array}
\]
que é equivalente a $P(0)$, por eliminação da implicação usando o fato
de que $0\in\mathbb{N}$. Logo, para $n = 0$, a fórmula da indução
forte é equivalente a $P(0)$. Seguindo o mesmo raciocínio, para $n =
1$ temos:
\[
1\in\mathbb{N}\to (\forall k. k\in \mathbb{N} \land l < 1 \to P(k))
\to P(1)
\]
mas, como o único valor de $k\in\mathbb{N}$ tal que $k < 1$ é $k = 0$,
temos que a fórmula anterior é equivalente a:
\[
\begin{array}{lc}
1\in\mathbb{N}\to (\forall k. k\in \mathbb{N} \land l < 1 \to P(k))
\to P(1) & \equiv \\
1\in\mathbb{N}\to (0 < 1 \to P(0))
\to P(1) & \equiv \\
1\in\mathbb{N}\to (\top \to P(0))
\to P(1) & \equiv \\
1\in\mathbb{N}\to (\neg \top \lor P(0))
\to P(1) & \equiv \\
1\in\mathbb{N}\to (\bot \lor P(0))
\to P(1) & \equiv \\
1\in\mathbb{N} \to P(0) \to P(1)
\end{array}
\]
que é equivalente a $P(0) \to P(1)$. Como temos que $P(0)$ (fórmula
equivalente para $n = 0$), podemos concluir $P(1)$, usando eliminação
da implicação. Repetindo esse mesmo raciocínio para $n = 2$, obtemos a
fórmula
\[
P(0) \land P(1) \to P(2)
\]
que pode ser usada para concluir $P(2)$. Repetindo esse processo,
temos que a propriedade $P$ será verdadeira para qualquer $n
\in\mathbb{N}$. Assim, temos que
\[
\forall n. n\in\mathbb{N} \to (\forall k. k\in\mathbb{N} \land k < n
\to P(k)) \to P(n)
\]
realmente é equivalente a $\forall n.n\in\mathbb{N}\to P(n)$.
Agora que justificamos a técnica de indução forte, a utilizaremos para
demonstrar o teorema \ref{strong-ind1}.

\begin{proof}
Suponha $n\in\mathbb{N}$ arbitrário e que para todo $k \in
\mathbb{N},\,k < n$, $k$ é primo ou produto de primos. Evidentemente,
se $n$ é primo o resultado é imediato. Portanto, suponha que $n$ não é
primo. Logo, existem $a,b\in\mathbb{N}$ tais que $1 < a,b < n$ e $n =
ab$. Como $a < n$ e $b < n$, pela hipótese de indução, temos que estes
são primos ou produto de primos. Considere os seguintes casos:
\begin{enumerate}
  \item $a$ e $b$ são primos. Logo, $n = a.b$ é um produto de primos.
  \item $a$ e $b$ não são primos\footnote{Isto é, pela hipótese de
      indução algum dos dois (ou ambos) são um produto de
      primos}. Logo, n = $a.b$ é um produto de primos.
\end{enumerate}
Como os casos cobrem todas as possibilidades, temos que $n$ é um
produto de primos.
\end{proof}
Como um segundo exemplo da técnica de indução forte, apresentaremos
outro resultado da teoria de números envolvendo o algoritmo de
divisão de dois números inteiros.
\begin{Theorem}
Para todos $n,m\in\mathbb{N}$, se $m > 0$ então existem $q$ e
$r$ tais que $n = mq + r$ e $r < m$.
\end{Theorem}
Note que $q$ e $r$ denotam o quociente e o resto da divisão,
respectivamente. O teorema pode ser expresso pela seguinte fórmula:
\[
\forall m. m\in\mathbb{N} \land m > 0\to \forall n. n \in\mathbb{N} \to m > 0
\to \exists q. \exists r. q \in\mathbb{N}\land r \in \mathbb{N} \land
n = m q + r \land r < m
\]
Iniciamos esta demonstração supondo $m \in \mathbb{N}$ arbitrário, $m
> 0$ e, na sequência, usamos indução forte para demonstrar
\[
\forall n. n \in\mathbb{N}
\to \exists q. \exists r. q \in\mathbb{N}\land r \in \mathbb{N} \land
n = m q + r \land r < m
\]
Em seguida, supomos $n \in\mathbb{N}$ arbitrário e que para todo $k < n$,
$\exists q. \exists r. q \in\mathbb{N}\land r \in \mathbb{N} \land
k = m q + r \land r < m$, o que, usando eliminação nos permite deduzir
as hipóteses de que $k = m q + r$ e $r < m$. Para finalizar a
demonstração, resta provar que
\[
\exists q. \exists r. q \in\mathbb{N}\land r \in \mathbb{N} \land
n = m q + r \land r < m
\]
Se $n < m$, então basta fazer $q = 0$ e $r = n$ e o resultado é
imediato. Para $n \geq m$, temos que encontrar valores de $q$ e $r$
tais que $n = mq + r$ e $r < m$. Note que como $n \geq m$, não podemos fazer que $r =
n$. É óbvio que neste caso deveremos usar a hipótese de indução, mas
para isso, devemos encontrar um valor de $k< n$ e  a partir deste
encontrar $q$ e $r$. Qual será esse valor de $k$? Se nos atentarmos ao
fato de que a divisão, $n \div m$, consiste em subtrair $m$ de $n$
sucessivamente, um possível valor para $k$ é $n - m$, que será menor
que $n$, visto que $m > 0$. Usando este valor o resultado desejado é
quase imediato, como pode ser visto na demonstração a seguir.

\begin{proof}
  Suponha $m \in \mathbb{N}$ arbitrário tal que $m > 0$. Suponha $n
  \in \mathbb{N}$ arbitrário e que para todo $k < n$ temos que existem
  $q'$ e $r'$ tais que $k = q'm + r'$ e $r' < m$. Considere os casos:
  \begin{enumerate}
    \item $n < m$. Seja $q = 0$ e $r = n$. Com isso, temos que $n
      = q.m + r$ e $r < m$, conforme requerido.
    \item $n \geq m$. Seja $k = n - m < n$. Como $n \geq m$, temos que
      $k$ é um número natural. Pela hipótese de indução, existem $q'$
      e $r'$ tais que $k = mq' + r'$ e $r' < m$. Então, $n - m = mq' +
      r'$ e, portanto, $n = mq' + r' + m = m (q'  +1) + r'$. Assim,
      sejam $q = q' + 1$ e $r = r'$. Então, temos que $n = mq + r$ e
      $r < m$, conforme requerido.
  \end{enumerate}
\end{proof}

Como um próximo exemplo, provaremos uma propriedade possuída por todo
subconjunto não vazio de números naturais.

\begin{Theorem}[Princípio da boa ordenação]\label{wellord}
Todo conjunto não vazio de números naturais possui um elemento mínimo.
\end{Theorem}
Representamos este teorema é representado pela seguinte fórmula
\[
\forall S. S\subseteq \mathbb{N} \to S \neq \emptyset \to S\text{
  possui um elemento mínimo}
\]
Aparentemente, este teorema não pode ser demonstrado por indução
forte, uma vez que este não possui a estrutura
\[
\forall n. n\in\mathbb{N} \to P(n)
\]
Porém, se supormos $S\subseteq \mathbb{N}$ arbitrário, e
representarmos
\[S \neq \emptyset \to S\text{ possui um elemento mínimo}\]
pela contrapositiva, temos a seguinte implicação:
\[S\text{ não possui um elemento mínimo} \to S = \emptyset\]
Supondo que $S$ não possui mínimo, temos que provar que $S =
\emptyset$, que é equivalente a dizer que $\forall n. n\in\mathbb{N}
\to n \not\in S$, o que é uma fórmula que pode ser demonstrada por
indução, conforme apresentado na demonstração abaixo.
\begin{proof}
Suponha $S\subseteq\mathbb{N}$ arbitrário. Suponha que $S$ não possui
um elemento mínimo. Suponha $n\in\mathbb{N}$ arbitrário e que para
todo $k < n$, $k \not\in S$. Se $n \in S$, temos que $S$ possui um
elemento mínimo, o que contraria a suposição de que $S$ não possui
mínimo. Logo, se $S \neq \emptyset$, $S$ possui um elemento mínimo.
\end{proof}
Agora, utilizaremos o princípio da boa ordenação para demonstrar mais
um fato sobre números.
\begin{Theorem}
$\sqrt{2}$ é um número irracional.
\end{Theorem}
Lembre-se que dizemos que um número $n$ é racional se existem $p$ e
$q$ tais que $n = \frac{p}{q}$. Para mostrar que $\sqrt{2}$ é
irracional, devemos supor que existem $p,q$ tais que $\sqrt{2} =
\frac{p}{q}$ e obter uma contradição a partir deste fato. É fácil ver
que a partir de $\sqrt{2} =\frac{p}{q}$ podemos deduzir que $p^2 =
2q^2$ e, portanto, $p^2$ é par. Porém, se $n^2$ é par, temos que $n$ é
par, logo temos que a fração $\frac{p}{q}$ poderia ser
simplificada. Aparentemente, este raciocínio não levaria a lugar
algum. Porém, ao usarmos esta idéia em conjunto com o princípio da boa
ordenação, obtemos a contradição desejada.
\begin{proof}
Suponha que $\sqrt{2}$ é número racional. Logo, existem $p,q$ tais que
$\sqrt{2}=\frac{p}{q}$.
Seja $Q$ o seguinte conjunto:
\[
Q =\{q\in\mathbb{N}^+\,|\,\exists p. p \in \mathbb{N}^+ \land
\frac{p}{q} = \sqrt{2}\}
\]
Porém se $\sqrt{2}$ é racional, $Q$ não é vazio e, portanto, pelo
princípio da boa ordenação (teorema \ref{wellord}), $Q$ possuirá um
elemento mínimo. Seja $q$ o mínimo de $Q$. Então, podemos escolher $p
\in\mathbb{N}^+$ tal que $\frac{p}{q} = \sqrt{2}$. Assim, temos que
$p^2$ e $p$ são pares. Logo, existe $x$ tal que $p = 2x$. Substituindo
$p = 2x$ em $p^2 = 2q^2$, temos que $4x^2 = 2q^2$ e, portanto, $q^2 =
2x^2$ e, portanto $q^2$ e $q$ são pares. Logo, $\sqrt{2} =
\frac{x}{y}$, em que $q = 2y$. Logo, $y\in Q$. Porém, como $y < q$,
temos que este fato contradiz a suposição de que $q$ é o mínimo de $Q$.
Logo, $\sqrt{2}$ é irracional.
\end{proof}

\subsection{Exercícios}

\begin{enumerate}
  \item Prove que $\sqrt{3}$ é irracional.
\end{enumerate}

\section{Paradoxos e Indução Matemática}

A técnica de indução matemática é muito útil para demonstrar
propriedades sobre números naturais. Porém, esta pode também ser usada
para ``provar'' paradoxos, como o seguinte teorema:
\begin{Theorem}
Todos os cavalos possuem a mesma cor.
\end{Theorem}
\begin{proof}
A prova será por indução sobre o número de cavalos.
\begin{enumerate}
  \item[\ ]Caso base: Para $n = 1$, temos que todos os cavalos do
    conjunto contendo $ n = 1$ cavalos possuem a mesma cor.
  \item[\ ]Passo indutivo: Suponha $n$ arbitrário e que todos os $n$
    cavalos, $C_1,...,C_n$ possuem a mesma cor. Para mostrar que todos
    os cavalos $C_1,...,C_{n+1}$ possuem a mesma cor, considere os
    seguintes conjuntos $A$ e $B$ ambos contendo $n$ cavalos:
    \[
    \begin{array}{l}
      A = \{C_1,...,C_n\} \\
      B = \{C_2,...,C_n\}\\
    \end{array}
    \]
    Como  $|A| = |B| = n$, temos que todos os cavalos de $A$ possuem a
    mesma cor $x$ e todos os cavalos de $B$ possuem a mesma cor
    $y$. Porém, como $C_2 \in A$ e $C_2 \in B$, temos que as cores $x$
    e $y$ são iguais. Portanto, todos os cavalos possuem a mesma cor.
\end{enumerate}
\end{proof}
Evidentemente, o teorema anterior possui uma falha, pois existem
cavalos das mais variadas cores. A falha deste teorema é que este não
é válido para conjuntos contendo $2$ cavalos. Note que se
considerarmos um conjunto possuindo apenas os cavalos $a$ e $b$,
quando dividirmos este em dois conjuntos $A$ e $B$, teremos $A =\{a\}$
e $B =\{b\}$, não possuindo, portanto, uma interseção.

Outro exemplo de uma falsa prova por indução é a seguinte:

\begin{Theorem}
Todo número natural é igual a $0$.
\end{Theorem}
\begin{proof}
Suponha $n\in\mathbb{N}$ arbitrário e que para todo $k < n$, $k =
0$. Considere os casos:
\begin{enumerate}
  \item[\ ] Caso $n = 0$. Neste caso, o resultado é imediato.
  \item[\ ] Caso $n \neq 0$. Logo, existe $m$ tal que $n = m +
    1$. Pela hipótese de indução, todo $k < n$ é igual a $0$, logo, $m
    = 0$ e $1 = 0$. Assim, temos que $n = m + 1 = 0 + 0 = 0$, conforme requerido.
\end{enumerate}
\end{proof}
Assim, como o exemplo anterior, este teorema é evidentemente
falso. Note que este falha para $n = 1$, pois o único valor de $k < 1$
é zero e, portanto, não podemos concluir que $1 = 0$ como usado na
``prova'' anterior.

\section{Notas Bibliográficas}

Indução matemática é tema de todo livro de matemática discreta. Os
exemplos e definições deste capítulo foram obtidos de \cite{Velleman06}.
