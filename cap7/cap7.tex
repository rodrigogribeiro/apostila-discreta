\chapter{Relações}\label{cap7}

\epigraph{O cliente pode ter um carro pintado com a cor que desejar,
  contanto que esta seja preto.}{Henry Ford, Pioneiro da indústria automobilística.}


\section{Motivação}

Existem diversos tipos de relações em nosso cotidiano. Algumas destas
descrevem como membros de uma família estão relacionados entre si:
pais, filhos, irmãos, irmãs, sobrinhos, etc. Outras especificam, por
exemplo, que certas cidades pertecem a um determinado país:
por exemplo, Londres está na Inglaterra, e Paris na França. Ou podemos
ter uma relação que descreve quais automóveis são montados por um
certo fabricante. Relações são utilizadas na matemática para descrever
como dois números se relacionam: por exemplo, dados dois números $x$ e
$y$ temos que $x \geq y$, $x < y$, em que $\geq$ e $<$ são relações
entre números.

Relações estão presentes em diversos ramos da computação, pois a
terminologia da teoria de relações permite descrever conceitos de
maneira precisa. Talvez, a aplicação mais famosa de relações em
ciência da computação são os bancos de dados relacionais. Porém,
relações formam a base teórica de muitas outras áreas como a semântica
de linguagens de programação, demonstração de terminação de
algoritmos, representação de informação armazenada em máquinas de
busca, teoria de grafos, etc. Uma vez que relações são ubíquas e
importantes, é útil definí-las como objetos matemáticos e descrever
suas propriedades. O objetivo deste capítulo é apresentar a teoria de
relações e a demonstração de alguns resultados importantes desta.

\begin{Remark}
Neste capítulo assumimos que o leitor já possui a maturidade para
compreender e demonstrar teoremas. Portanto, na maioria das
demonstrações o rascunho será completamente omitido. Porém,
recomenda-se que este seja ``reconstruído'' pelo leitor para um maior
entendimento do conteúdo.
\end{Remark}

\section{Pares Ordenados e Produto Cartesiano}

Em capítulos anteriores, lidamos com conjuntos em que cada elemento é
um ``componente'' deste. Porém, como você aprendeu em outros cursos,
existem conjuntos formados por pares de números que representam pontos
em um plano. Nesta seção, vamos introduzir formalmente o conceito de
par ordenado e como podemos construir conjuntos de pares utilizando
uma operação conhecida como produto cartesiano. As definições
seguintes apresentam estes conceitos.

\begin{Definition}[Par ordenado]
Sejam $A$ e $B$ conjuntos quaisquer em que $a \in A$ e $b \in
B$. Dizemos que $(a,b)$ é um par ordenado em que o primeiro elemento é
$a \in A$ e o segundo $b\in B$.
\end{Definition}

A operação sobre conjuntos que permite a criação de pares ordenados é
o chamado produto cartesiano, que é definido a seguir.

\begin{Definition}[Produto Cartesiano]
Sejam  $A$ e $B$ dois conjuntos quaisquer. O produto cartesiano de $A$
por $B$, $A\times B$, é definido como:
\[
A\times B = \{(a,b)\,|\,a \in A \land b\in B\}
\]
\end{Definition}

\begin{Example}
Sejam $A =\{1,2,3\}$ e $B = \{4,5,6\}$. Temos:
\[
A \times B = \{(1,4),(2,5),(3,6)\}
\]
Evidentemente, temos que $(1,4) \in A \times B$. Além disso, para o
par ordenado $(1,4)$ temos que $1$ é o primeiro elemento (um elemento
do conjunto $A$) e $4$ é o segundo (um elemento de $B$).
\end{Example}
Uma boa maneira de atestarmos a compreensão de um novo conceito
matemático é demonstrando teoremas sobre este.
\begin{Theorem}
Sejam $A, B$ e $C$ conjuntos arbitrários. Então $A \times (B\cap C) =
(A \times B) \cap (A \times C)$.
\end{Theorem}
\begin{proof}
Suponha que $A,B$ e $C$ são conjuntos arbitrários. Suponha $p$
arbitrário.
\begin{itemize}
   \item[$(\to)$] Suponha $p \in A \times (B \cap C)$. Pela definição
     de produto cartesiano, temos que $p = (a,b)$ em que $a \in A$ e
     $b \in B \cap C$. Já que $b\in B\cap C$, temos que $b \in B$ e $b
     \in C$. Como $a \in A$ e $b \in B$, temos que $(a,b)\in A \times
     B$. De maneira similar, já que $a \in A$ e $b \in C$, temos que
     $(a,b) \in A \times C$. Logo, $(a,b) \in (A\times B) \cap
     (B\times C)$. Portanto, se $p \in A\times (B\cap C)$ então
     $(A\times B)\cap (A \times C)$.
   \item[$(\leftarrow)$] Suponha $p\in (A\times B) \cap (A \times
     C)$. Assim, temos que $p \in A \times B$ e $p \in A \times C$.
     Pela definição de produto cartesiano, temos que $p = (a,b)$ em
     que $a \in A$, $b \in B$ e $b\in C$. Já que $b\in B$ e $b\in C$,
     temos que $b\in B\cap C$. Mas, como $a \in A$ e $b\in B\cap C$,
     temos que $(a,b)\in A\times (B\cap C)$. Logo, se $p \in (A\times
     B) \cap (A \times C)$ então $p \in A \times (B\cap C)$.
\end{itemize}
Como $p$ é arbitrário, temos que $A \times (B\cap C) = (A \times B)
\cap (A \times C)$. Portanto, para todos conjuntos $A,B$ e $C$ temos
que $A \times (B\cap C) = (A \times B)
\cap (A \times C)$.
\end{proof}

\begin{Commentary}
Um ponto crucial desta demonstração é a utilização das hipóteses que
um elemento $p$ que pertence ao produto cartesiano de dois conjuntos
deve ser um par em que o primeiro elemento pertence ao primeiro
conjunto e o segundo elemento, ao segundo conjunto.

No caso do teorema anterior, em um momento temos que $p \in A \times
(B\cap C)$, então $p = (a,b)$ em que $a \in A$ e $b \in B\cap C$.

Além deste
detalhe, toda a demonstração consiste apenas de uso de técnicas de
provas que já vimos no capítulos \ref{cap4} e \ref{cap5}.

Evidentemente, como o produto cartesiano de conjuntos é apenas um
conjunto de pares ordenados, notações da teoria de conjuntos
(apresentadas no capítulo \ref{cap5}).
\end{Commentary}

\section{Exercícios}

\begin{enumerate}
  \item Prove os seguintes teoremas:
  \begin{enumerate}
    \item Seja $A$ um conjunto qualquer. Então $A \times \emptyset =
      \emptyset$.
    \item Sejam $A$ e $B$ conjuntos quaisquer. Se $A \times B =
      B\times A$ se e somente se $A = \emptyset$ ou $B = \emptyset$ ou
      $A = B$.
  \end{enumerate}
\end{enumerate}

\section{Introdução às Relações}

Matematicamente, especificamos que dois objetos $a$ e $b$ estão
relacionados dizendo que o par $(a,b)$ pertence ao conjunto de pares
que descreve uma propriedade de interesse sobre estes objetos. Usamos
relações (conceito matemático) para expressar relacionamentos entre
objetos modelados matematicamente como conjuntos.

\begin{Definition}[Relação]
Suponha que $A$ e $B$ são conjuntos quaisquer. Denominamos o conjunto
$R \subseteq A \times B$ uma relação de $A$ em $B$.
\end{Definition}

A seguir apresentamos alguns exemplos de relações.

\begin{Example}
Considere os seguintes conjuntos $A = \{1,2,3\}$ e $B =
\{4,5,6\}$. Temos que $R = \{(1,5),(3,4)\}$ é uma relação de $A$ em
$B$, já que $R\subseteq A \times B$.

Outro exemplo de relação, agora envolvendo um conjunto infinito de
pares, é:
\[ G = \{(x,y)\in\mathbb{R}\times\mathbb{R}\,|\,x < y\}\]
esta relação representa, utilizando pares, o conceito de ``menor''
sobre números reais.
\end{Example}

Relações não necessariamente são formadas apenas por conjuntos
numéricos. O próximo exemplo mostra relações sobre conjuntos não
numéricos.

\begin{Example}
Considere os seguintes conjuntos que poderiam ser utilizados para
modelar um sistema de informação em uma universidade:
\begin{itemize}
  \item $S$ : conjunto de todos os estudantes da universidade.
  \item $C$: conjunto de todos os cursos de graduação da universidade.
  \item $D$: conjunto de todas as disciplinas oferecidas em cursos da
    universidade.
  \item $P$: conjunto de todos os professores que lecionam na universidade.
\end{itemize}
Utilizando estes conjuntos, temos as seguintes relações:
\begin{itemize}
  \item $R = \{(e,c)\in S \times C\,|\,\text{O estudante } e\text{
      está matriculado no curso }c\}$.
  \item $R_1 =\{(p,d)\in P \times D\,|\,\text{O professor }p\text{
      leciona a disciplina }d\}$.
\end{itemize}
Evidentemente, estas relações estariam representadas por mecanismos
apropriados de bancos de dados relacionais em um sistema de informação
de uma universidade. A primeira relação modela as informações sobre
qual é o curso em que um aluno está matriculado e a segunda, qual
disciplina um professor leciona.
\end{Example}

A seguir, apresentamos alguns conceitos provavelmente já conhecidos
pelo leitor, mas em um contexto de funções e não de
relações\footnote{Veremos, posteriormente, que funções são apenas um
  tipo especial de relações.}

\begin{Definition}[Domínio, Imagem, Inversa]
Suponha que $R$ é uma relação de $A$ em $B$. Então o domínio de $R$ é
o conjunto:
\[
dom(R) = \{a \in A \,|\, \exists b. b\in B \land (a,b) \in R\}
\]
A imagem\footnote{Normalmente, livros denotam o conjunto imagem usando
a abreviação $ran$ de \textit{range} de imagem em inglês.} de $R$ é definida pelo seguinte conjunto:
\[
ran(R) = \{b\in B \,|\, \exists a. a\in A \land (a,b) \in R\}
\]
Finalmente, a relação inversa de $R$, $R^{-1} \subseteq B\times A$, é:
\[
R^{-1} =\{(b,a)\,|\,\exists a. \exists b. a \in A \land b\in B \land
(a,b) \in R\}
\]
\end{Definition}

\begin{Example}
Considere:
\begin{itemize}
   \item $A = \{1,2,3,4\}$ e $B = \{6,7,8,9,0\}$.
   \item $R = \{(1,6),(3,0),(2,9)\}$.
\end{itemize}
Temos:
\begin{itemize}
   \item $dom(R) =\{1,2,3\}$
   \item $ran(R) =\{0,6,9\}$
   \item $R^{-1} = \{(6,1),(0,3),(9,2)\}$
\end{itemize}
\end{Example}

\begin{Definition}[Composição de Relações]
Sejam $R \subseteq A \times B$ e $S \subseteq B \times C$ duas
relações sobre conjuntos $A,B,C$ e $D$. A relação composta
de $S$ e $R$, $S \circ R$, como uma relação de $A$ em $C$ definida
como:
\[
S \circ R =\{(a,c)\,|\,\exists b. b \in B \land (a,b) \in R \land
(b,c) \in S\}
\]
\end{Definition}