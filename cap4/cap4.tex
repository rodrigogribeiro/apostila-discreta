\chapter{Demonstração de Teoremas}\label{cap4}

\epigraph{A matemática não é uma ciência dedutiva --- isto é um
  clichê. Quando você tenta provar um teorema, você não apenas lista
  as hipóteses, e começa a dedução. O que normalmente fazemos é fazer
  uso de experimentação e tentativa e erro.}{Paul Richard Halmos,
  Matemático.}

\section{Motivação}

Nos capítulos \ref{cap2} e \ref{cap3} foram apresentadas as lógicas
proposicional e de predicados. Para cada uma destas lógicas, estudamos
sua sintaxe, semântica e como verificar consequências lógicas
utilizando dedução natural. Neste capítulo, apresentaremos uma
importante aplicação de tudo que foi visto até o presente momento:
usar estas lógicas para demonstrar teoremas matemáticos.

Mas qual a importância do uso de demonstrações em computação? A única
tecnologia conhecida para garantir a ausência de erros em programas de
computador é provando que este não possui
erros. Evidentemente, isso requer a modelagem de programas em algum
formalismo matemático adequado para esta tarefa, o que está fora do
escopo deste texto. Porém, técnicas elementares de demonstração de
teoremas são a ``base'' para a formalização e verificação de sistemas
computacionais. Logo, é importante que todo estudante de computação
saiba construir e entender demonstrações formais.

\section{Introdução}

Damos o nome de \emph{teorema} a uma sentença matemática que é
verdadeira e pode ser verificada como tal. Teoremas são compostos por
um conjunto, possivelmente vazio, de sentenças, denominadas hipóteses
(ou premissas), que são assumidas como verdadeiras \emph{a priori} e
uma conclusão.
 Normalmente, teoremas
são expressos utilizando variáveis possivelmente livres. Damos o nome
de \emph{instância} de um teorema a uma particular atribuição de
valores às variáveis de um teorema.  A \emph{prova} ou
\emph{demonstração} de um teorema consiste de uma verificação que
mostra que o teorema em questão é verdadeiro, para todas as possíveis
instâncias deste. Note que um teorema só pode ser considerado como
válido se este o for para todas suas instâncias. Para mostrar que um
``teorema''\footnote{Note que uma sentença só pode ser
  considerada um teorema se esta for verdadeira. Afirmar que um
  teorema é falso é apenas um abuso de linguagem utilizado para
  facilitar a exposição deste conceito.} é inválido basta apresentar
uma instância que torna o enunciado deste falso. Damos o nome de
\emph{contra-exemplo} a uma instância que torna uma sentença falsa.

A seguir apresentamos um exemplo que ilustra os conceitos apresentados
no parágrafo anterior.

\begin{Example}
Considere a seguinte sentença matemática:
\begin{center}
\textit{Sejam $x,y$ dois números reais tais que $x > 3$ e $y < 2$. Então, $x^2
- 2y > 5$}.
\end{center}
Esta sentença é um teorema e sua prova será apresentada em um exemplo
posterior. Como esta é um teorema, ela deverá ser composta por um
conjunto de hipóteses e uma conclusão. Note que o enunciado deste
teorema assume que $x,y\in\mathbb{R}$ e que $x > 3$ e $y < 2$. Logo,
estas são as suas hipóteses. A conclusão deste teorema é que
a desigualdade $x^2 - 2y > 5$ deve ser verdadeira. Como um exemplo de
uma possível instância desse teorema são $x = 4$ e $y = 0$ que tornam a
desigualdade $16 - 2.0 > 5$ verdadeira. Evidentemente, caso $x = 3$ e
$y = 2$ (violando, assim, as hipóteses $x > 3$ e $y < 2$) temos que a
conclusão é falsa pois, $9 - 4 = 5 \not> 5$.

Agora, como exemplo de uma sentença inválida, considere:
\begin{center}
\textit{Sejam $x,y$ dois números reais tais que $x > 3$. Então, $x^2
- 2y > 5$.}
\end{center}
Esta sentença não pode ser considerada um teorema por este possuir um
contra-exemplo. Seja $x= 4$ e $y = 6$. Temos que $x = 4 > 3$, mas
$16 - 12 \not > 5$, o que torna falsa a sentença em questão.

É importante ter em mente que para demonstrar um teorema devemos
construir uma prova (dedução) de que este é correto para todas as suas
instâncias. Se quisermos mostrar que uma sentença é falsa, basta
apresentar um contra-exemplo.
\end{Example}

Você deve ter percebido que
teoremas possuem a mesma estrutura de sequentes da dedução
natural. Na verdade, todos os sequentes que demonstramos em capítulos
anteriores, são teoremas! Neste capítulo, utilizaremos a dedução
natural para demonstrar a validade de sentenças quaisquer da
matemática. Porém, ao invés de utilizarmos uma notação hierárquica (em
forma de uma árvore), como fizemos com a dedução natural, utilizaremos
uma notação \textit{estruturada}, no sentido que organizaremos
demonstrações em blocos, similares à blocos de comandos presentes na
maioria das linguagens existentes (como C/C++, Java, Python,
etc.).

Para facilitar a tarefa de construir demonstração corretas e similares
as encontradas em textos de matemática, vamos dividir a tarefa de
provar um teorema em duas partes, relacionadas: 1) construção de um
rascunho e 2) elaboração de um texto, a partir do rascunho\footnote{A
  técnica que adotaremos neste texto para construção de demonstrações
  é a apresentada no livro de Daniel Velleman \cite{Velleman06}.}.

O rascunho é utilizado para realizar as demonstrações. Normalmente
este é dividido em duas colunas que
