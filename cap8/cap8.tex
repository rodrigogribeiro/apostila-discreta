\chapter{Funções}\label{cap8}

\epigraph{Um matemático é uma função que a partir de café produz
  teoremas.}{Paul Ërdos --- Matemático Húngaro.}

\section{Motivação}

Funções são provavelmente um dos conceitos matemáticos de maior
importância para a Computação, já que funções podem ser entendidas
como um modelo abstrato de algoritmo: a partir de uma ou mais
entradas, uma função produz um único resultado. Além disso, este
resultado é completamente determinado pela entrada: se aplicarmos
repetidamente uma função ao mesmo argumento sempre obteremos o mesmo
resultado.

A característica mais importante do conceito de função para a
computação é que estas são um mecanismo de abstração. De maneira
simples, para usarmos uma função precisamos apenas saber sua interface
(o que ela recebe como parâmetros e o que ela retorna como resultado)
e não como esta é implementada internamente. Este conceito de
abstração é amplamente utilizado em computação.

Iniciaremos este capítulo definindo funções como um caso especial de
relações e então consideraremos uma maneira ``algorítmica'' de se
definir funções.

\section{Introdução às funções}

Como já dito na seção anterior, matematicamente, funções são apenas um
caso especial de relações.

\begin{Definition}[Função]
Sejam $A$ e $B$ dois conjuntos quaisquer e $f \subseteq A \times
B$. Dizemos que $f$ é uma função se:
\[
\forall x. x\in A \to \exists ! b. b\in b \land (a,b) \in f
\]
\end{Definition}

\section{Funções, algoritmicamente}

\section{Funções totais e parciais}

\section{Composição de funções}

\section{Propriedades de funções}

\subsection{Funções Injetoras}

\subsection{Funções Sobrejetoras}

\subsection{Funções Bijetoras}

\subsection{Função inversa}


\section{Notas Bibliográficas}