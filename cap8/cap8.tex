\chapter{Funções}\label{cap8}

\epigraph{Um matemático é uma função que a partir de café produz
  teoremas.}{Paul Ërdos --- Matemático Húngaro.}

\section{Motivação}

Funções são provavelmente um dos conceitos matemáticos de maior
importância para a Computação, já que funções podem ser entendidas
como um modelo abstrato de algoritmo: a partir de uma ou mais
entradas, uma função produz um único resultado. Além disso, este
resultado é completamente determinado pela entrada: se aplicarmos
repetidamente uma função ao mesmo argumento sempre obteremos o mesmo
resultado.

A característica mais importante do conceito de função para a
computação é que estas são um mecanismo de abstração. De maneira
simples, para usarmos uma função precisamos apenas saber sua interface
(o que ela recebe como parâmetros e o que ela retorna como resultado)
e não como esta é implementada internamente. Este conceito de
abstração é amplamente utilizado em computação.

Iniciaremos este capítulo definindo funções como um caso especial de
relações e então consideraremos uma maneira ``algorítmica'' de se
definir funções.

\section{Introdução às funções}

Como já dito na seção anterior, matematicamente, funções são apenas um
caso especial de relações.

\begin{Definition}[Função]
Sejam $A$ e $B$ dois conjuntos quaisquer e $f \subseteq A \times
B$. Dizemos que $f$ é uma função se:
\[
\forall x. x\in A \to \exists ! b. b\in B \land (a,b) \in f
\]
\end{Definition}

\begin{Notation}
Utilizaremos a seguinte notação para representar que a relação $f
\subseteq A \times B$ é uma função:
\[
f : A \to B
\]
e a notação $f(x) = y$ se $(x,y) \in f$ e $f$ é uma função.
\end{Notation}
Como uma função é um caso especial de relação, temos que todas as
definições de relações (domínio, imagem, contra-domínio) são aplicáveis.
\begin{Example}
Apresentaremos alguns exemplos de funções.
\begin{itemize}
  \item Sejam $A = \{1,2,3\}$ e $B =\{4,5,6\}$ e $f =
    \{(1,3),(2,6),(3,5)\}$. Temos que a relação $f$ é uma função de
    $A$ em $B$.
  \item Sejam $A = \{1,2,3\}$ e $B =\{4,5,6\}$ e $g =
    \{(1,4),(2,6),(1,5),(3,6)\}$. Temos que $g$ não é uma função, pois
     $(1,4) \in g$ e $(1,5) \in g$.
  \item Sejam $C$ o conjunto de todas as cidades e $P$ o conjunto de
    todos os países. A relação $d =\{(c,p)\in C \times P\,|\,\text{a
      cidade $c$ está no país $p$.}\}$ é uma função, pois toda cidade
    pertence a um único país.
  \item Seja $P$ o conjunto de todas as pessoas. A relação $h =
    \{(p,m)\in P \times \mathcal{P}(P)\,|\,\text{A pessoa $p$  é pai
      (ou mãe) das pessoas no conjunto $m$}\}$ é uma função, já que
    toda pessoa possui um único conjunto de filhos.
\end{itemize}
\end{Example}
Dizemos que duas funções $f : A \to B$ e $g : A \to B$ são iguais se
estas denotam o mesmo conjunto de pares. Este fato é expresso pelo
teorema seguinte.
\begin{Theorem}[Extensionalidade]
Suponha que $f : A \to B$, $g : A \to B$ e que $\forall x. x\in A \to
f(x) = g(x)$. Então, $f = g$.
\end{Theorem}
\begin{proof}
Suponha $(a,b)$ arbitrário.
\begin{enumerate}
  \item[$(\to)$]: Suponha $(a,b) \in f$ arbitrário.
                     Como $\forall x. x\in A \to f(x) = g(x)$ e $f(a)
                     = b$, temos que $g(a) = b$. Logo, se $f(a) = b$
                     então $g(a) = b$.
   \item[$(\leftarrow)$]: Suponha $(a,b) \in g$ arbitrário.
                     Como $\forall x. x\in A \to f(x) = g(x)$ e $g(a)
                     = b$, temos que $f(a) = b$. Logo, se $g(a) = b$
                     então $f(a) = b$.
\end{enumerate}
Como $(a,b)$ é arbitrário, temos que $f = g$.
\end{proof}

\section{Funções, algoritmicamente}

Matematicamente, funções são apenas um conjunto de pares
ordenados. Porém, de um ponto de vista computacional, o conceito de
função é distinto, pois este corresponde a noção de algoritmo. O
problema com a versão ``matemática'' do conceito de função é que este
não envolve variáveis cruciais em computação como consumo de memória e
tempo de execução.

O objetivo desta seção é apresentar classes de funções que
correspondem a algoritmos e mostrar que nem toda função pode ser
programada em uma linguagem de programação.

\subsection{Funções definidas recursivamente}

No decorrer deste texto, apresentamos algumas funções recursivas:
funções para determinar o conjunto de sub-fórmulas, variáveis livres
de fórmulas da lógica. Podemos classificar uma função como sendo
recursiva se esta está de acordo com a próxima definição.

\begin{Definition}[Funções de Recursivas Primitivas]\label{recprim}
Seja $A$ um conjunto qualquer. Dizemos que uma função $f : \mathbb{N}
\to A \to A$ é definida  recursivamente se a definição de
$f$ possui a seguinte estrutura, em que $g,h$ são funções
recursivas primitivas:
\[
\begin{array}{lcl}
f(0,x) & = & g\:x\\
f(n,x) & = & h(f (n - 1, x), n - 1, x)
\end{array}
\]
\end{Definition}
Funções definidas por recursão primitiva são garantidas de sempre
terminar, isto é, atendem o critério de terminação\footnote{Se você
  não se lembra deste critério, recomendo que releia o capítulo \ref{cap1}.}.
Não demonstraremos este fato por este envolver conceitos que estão
além dos propósitos desta disciplina. A seguir, apresentamos alguns
exemplos de funções recursivas primitivas.
\begin{Example}
A função que a partir de um número natural retorna o seu quadrado é
uma função recursiva primitiva, conforme apresentado a seguir.
\[
\begin{array}{lcl}
g(x) & = & x \times x
\end{array}
\]
\end{Example}
O leitor deve estar se perguntando: ``como esta função pode ser
considerada primitiva recursiva se esta não usa recursão?''. O fato é
que toda função não recursiva atende o critério de função recursiva
primitiva\footnote{Isto é fácil de se entender, informalmente. A idéia
é que toda função recursiva primitiva possui \textbf{todas} as
chamadas recursivas no formato da definição \ref{recprim}. Como
definições não recursivas não possuem chamadas recursivas, estas
atendem este critério por vacuidade.}. O próximo exemplo apresenta uma função recursiva primitiva
que usa recursividade.
\begin{Example}
A função que calcula o fatorial de um número natural pode ser escrita
da seguinte maneira, em que $\bot$ representa um valor semanticamente
inválido (uma exceção).
\[
\begin{array}{lcl}
fact(n) &= & f(n,\bot) \\
 & & \\
f(0,x) & = & 1 \\
f(n,x) & = & n \times f (n - 1,x)\\
\end{array}
\]
Esta função está de acordo com a definição \ref{recprim} pois:

\begin{itemize}
    \item A função $h$ da definição pode ser representada pela função
      de multiplicação, que é primitiva recursiva.
    \item A função $g$ da definição pode ser representada pela função
      constante que sempre retorna $1$.
\end{itemize}

Como exemplo, vamos considerar o cálculo de $fact(4)$, apresentado
abaixo.
\[
\begin{array}{lc}
fact(4) & = \\
f(4,\bot) & = \\
4 \times f(3,\bot) & = \\
4 \times 3 \times f(2,\bot) & = \\
4 \times 3 \times 2 \times f(1,\bot) & =\\
4 \times 3 \times 2 \times 1 \times f(0,\bot) & = \\
4 \times 3 \times 2 \times 1 \times 1 & =\\
24
\end{array}
\]
\end{Example}
\begin{Example}
A seguinte função não pode ser considerada como recursiva primitiva.
\[
\begin{array}{lcl}
f(x) & = & \left\{
                    \begin{array}{ll}
                      0 & \text{Se $x = 0$}\\
                      1 & \text{Se $x = 1$}\\
                      1 + f(x \div 2) & \text{Se $x$ é par}\\
                      f(x) & \text{Caso contrário}
                    \end{array}
                \right.
\end{array}
\]
A definição apresentada não pode ser considerada uma função, já que
esta não termina para números ímpares maiores que $1$.
\end{Example}

\subsection{Funções totais e parciais}

Normalmente, em matemática considera-se apenas funções totais, isto é
funções que são definidas para todos os valores de um domínio. Porém,
em computação, este conceito não é adequado. Neste sentido,
considera-se também a noção de função parcial, que definimos a seguir.

\begin{Definition}[Funções totais e parciais]
Seja $f : A \to B$ uma função. Dizemos que $f$ é uma função total se
$dom(f) = A$. Por sua vez, dizemos que $f$ é parcial se $dom(f)
\subset A$.
\end{Definition}

Se $f$ é uma função parcial e $x\in dom(f)$ então dizemos que $f(x)$ é
definida. Caso $x\not\in dom(f)$, dizemos que $f(x)$ é
indefinida. Em algumas situações, ao invés de simplesmente dizermos
que $f(x)$ é indefinida, associamos a $x$ um valor não pertencente ao
contradomínio da função $f$, que normalmente é representado como
$\bot$. Neste caso, dizemos que $f(x)$ é indefinida se $f(x) = \bot$.

\begin{Example}
Considere a seguinte função, $f : \mathbb{N} \to \mathcal{C}$, em que
$\mathcal{C}$ representa o conjunto de símbolos (letras) do alfabeto latino.
\[
\begin{array}{lcl}
f(i) & = & c \text{, em que $c$ é a $i$-ésima letra do alfabeto}
\end{array}
\]
Evidentemente, temos que esta função é parcial, visto que não é
definida para todo $n \in \mathbb{N}$. O domínio de $f$ é:
\[
dom(f) =\{x \in\mathbb{N}\,|\,1 \leq x \leq 26\}
\]
já que o alfabeto latino possui 26 letras. Com isso, temos que $f$ é
indefinida para todo elemento de $\{x\in\mathbb{N}\,|\,x \geq 27\}$.
\end{Example}
Em matemática, é simples descobrir para quais elementos uma função é
ou não definida. Porém, em computação, este nem sempre é o
caso. Como funções são modelos abstratos de programas, determinar o
domínio de uma função (isto é, determinar se esta vai ser aplicada
somente a valores considerados válidos) é tarefa de um compilador.
Uma forma limitada deste tipo de análise é feita considerando tipos de valores e
parâmetros de uma função. Em linguagens de programação modernas,
compiladores são capazes de validar que somente valores de tipos
adequados sejam argumentos para chamadas de funções. Normalmente,
erros devido a passagem inválida de parâmetros podem ocasionar erros
em tempo de execução\footnote{Programadores C/C++ estão habituados a
  este tipo de erro caracterizado pela mensagem:
  ``\texttt{segmentation fault.}''}. Porém, mesmo valores de tipo
adequados podem ocasionar erros em tempo de execução. Para isso, basta
que o valor, apesar de possuir o tipo correto especificado para o
parâmetro da função, não pertença ao domínio desta. Em computação,
valores indefinidos para um determinado valor do domínio, são
usualmente representados utilizando exceções que, se não tratadas,
causam a interrupção do código.

\section{Composição de funções}

Nesta seção, vamos reapresentar o conceito de composição (agora de
funções) e provar que esta operação é associativa.

\begin{Definition}[Composição de Funções]
Sejam $f : A \to B$ e $g : B \to C$. Definimos $g \circ f : A \to C$
como:
\[
g \circ f (x) = g (f(x))
\]
\end{Definition}

\begin{Theorem}[Associatividade da composição de funções]
Sejam $f : A \to B$, $g : B \to C$ e $h : C \to D$. Temos que $h \circ
(g \circ f) = (h \circ g) \circ f$.
\end{Theorem}
\begin{proof}
Suponha $x$ arbitrário. Suponha que $x \in A$. Temos:
\[
\begin{array}{lcl}
((h \circ g) \circ f)(x) & = \\
(h \circ g) (f(x)) & = & \{\text{pela def. de $\circ$}\}\\
h (g (f (x))) & = & \{\text{pela def. de $\circ$}\}\\
h ((g \circ f) (x)) & = & \{\text{pela def. de $\circ$}\}\\
(h \circ (g \circ f)) (x)
\end{array}
\]
\end{proof}

\section{Propriedades de funções}

Nesta seção descrevemos algumas propriedades de funções que
provavelmente já são conhecidas pelo leitor.

\subsection{Funções injetoras}

O requisito para uma relação ser considerada uma função é que esta
associe cada elemento do domínio a um único elemento do
contradomínio. Por sua vez, dizemos que uma função é injetiva se cada
elemento da imagem desta função é o resultado de aplicá-la a somente
um elemento do domínio. A definição seguinte formaliza este conceito.

\begin{Definition}[Função injetora]
Dizemos que $f : A \to B$ é uma função injetora se a seguinte fórmula
é verdadeira:
\[
\forall x. \forall y. x\in A \land y \in A \to x \neq y \to f (x) \neq f(y)
\]
\end{Definition}

\begin{Example}
Sejam $A = \{1,2\}$, $B =\{3,4,5\}$ e $f : A \to B$ definida como $f =
\{(1,3),(2,5)\}$. Como cada imagem possui um único valor associado no
domínio, temos que $f$ é uma função injetora.
\end{Example}
\begin{Example}
Seja $f : \mathbb{N} \to \mathbb{N}$ definida como $f(x) = x^2$. Temos
que esta função é injetora, já que esta associa a cada quadrado
perfeito de $\mathbb{N}$ a sua raiz.
\end{Example}

O leitor atento deve notar que esta função é injetora quando definida
sobre $\mathbb{N}$. Para os conjuntos $\mathbb{Z},\mathbb{R}$ esta não
o é, uma vez que, para quaisquer números $x$ e $-x$, o quadrado destes
sempre será $|x|^2$. De maneira mais precisa, temos que se $f :
\mathbb{Z} \to \mathbb{Z}$ ou $f : \mathbb{R}\to\mathbb{R}$, $f$ não é
uma função injetora.

A classe das funções injetoras são fechadas sobre a operação de
composição, isto é, a composição de duas funções injetoras é também
uma função injetora. Este resultado é formalizado no teorema seguinte.

\begin{Theorem}\label{inj-compose}
Sejam $f : A \to B$ e $g : B \to C$ duas funções injetoras. Então $g
\circ f : A \to C$ também é uma função injetora.
\end{Theorem}
\begin{proof}
Suponha que $f : A \to B$ e $g : B \to C$ são funções
injetoras. Suponha $x,y$ arbitrários. Suponha $x,y \in A$. Suponha que
$x \neq y$. Suponha, por contradição, que $(g \circ f)(x) = (g\circ
f)(y)$. Pela definição de composição de funções, temos que $g(f(x)) =
g (f(y))$, mas como $g$ é injetora, temos que $f(x) = f(y)$. Mas, como
$f$ também é injetora, temos que $x = y$,  o que contradiz a suposição
de que $x \neq y$. Logo, $(g \circ f)(x) \neq (g \circ f)(y)$. Assim,
se $x\neq y$ então $(g \circ f)(x) \neq (g \circ f)(y)$. Logo, se $x,y
\in A$ e $x\neq y$ então $(g \circ f)(x) \neq (g \circ f)(y)$. Como
$x,y$ são arbitrários, temos que $g \circ f : A \to C$ é uma função
injetora. Portanto, se $f : A \to B$ e $g : B \to C$ são funções
injetoras, então temos que $g \circ f : A \to C$ é uma função injetora.
\end{proof}

Utilizando o conceito de funções injetoras, podemos expressar de
maneira mais elegante o chamado princípio da casa dos pombos.

\begin{Theorem}[Princípio da casa dos pombos]
Sejam $A$ e $B$ dois conjuntos finitos tais que $|A| > |B|$, $|A| >
1$ e $f : A \to B$ uma função total. Então:
\[
\exists a_1. \exists a_2. a_1 \in A \land a_2 \in A \land a_1 \neq a_2
\land f(a_1) = f(a_2)
\]
\end{Theorem}
\begin{proof}
Como $f$ é total, temos que todo $x\in A$ é tal que existe $y \in B$
de forma que $f(x) = y$. Porém, como $|A| > |B|$, temos que existirão
$a_1, a_2 \in A$, $a_1 \neq a_2$ tais que $f(a_1) = f(a_2)$.
\end{proof}

% \begin{proof}
% Suponha que $A,B$ são conjuntos finitos tais que $|A| > |B|$,
% $|A| > 1$ e $f : A  \to B$. Como $|A| > 1$, o conjunto $A$ deve
% possuir pelo menos dois elementos distintos. Sejam $x,y\in A$ dois
% elementos tais que $x \neq y$.
% \end{proof}


\subsection{Funções sobrejetoras}

Dizemos que uma função é sobrejetora se sua imagem é todo o seu
contradomínio. Formalizamos esse conceito na próxima definição.

\begin{Definition}[Função sobrejetora]
Dizemos que uma função $f : A \to B$ é uma função sobrejetora se a
seguinte fórmula é verdadeira:
\[
\forall b. b\in B \to \exists a. a \in A \land f(a) = b
\]
\end{Definition}

\begin{Example}
Seja $A =\{1,2,3\}$, $B =\{4,5\}$ e  $f : A \to B$ definida como $f
=\{(1,4),(2,5),(3,5)\}$. Temos que $f$ é uma função sobrejetora, uma
vez que sua imagem é igual ao conjunto $B$ (contradomínio de
$f$). Note que apesar de ser sobrejetora, a função $f$ não é
injetora uma vez que $f(2) = 5 = f(3)$.
\end{Example}

\begin{Example}
A função $f : \mathbb{N} \to \mathbb{N}$ definida como $f(x) = x^2$
não é sobrejetora, uma vez que não existe $x \in \mathbb{N}$ tal que
$x^2 = 3$. Na verdade, esta função possui como imagem o conjunto de
quadrados perfeitos de $\mathbb{N}$.
\end{Example}

Assim como as funções injetoras, a classe das funções sobrejetoras
também é fechada sobre a operação de composição. Esse fato é expresso
pelo teorema seguinte.

\begin{Theorem}\label{sobre-compose}
Sejam $f : A \to B$ e $g : B \to C$ duas funções sobrejetoras. Então,
$g \circ f : A \to C$ é uma função sobrejetora.
\end{Theorem}

\subsection{Funções bijetoras}

Nesta seção apresentaremos uma classe importante de funções: as
funções bijetoras.

\begin{Definition}[Função bijetora]
Dizemos que $f : A \to B$ é uma função bijetora se $f$ é uma função
injetora e sobrejetora.
\end{Definition}

Caso uma função é bijetora, dizemos que está é uma correspondência de
um para um, conforme definido pelo teorema a seguir.

\begin{Theorem}
Seja $f : A \to B$ uma função bijetora. Então, $|dom(f)| = |ran(f)|$.
\end{Theorem}
\begin{proof}
Suponha que $|dom(f)| > |ran(f)|$. Como $|dom(f)| > |ran(f)|$, temos
que $f$ não pode ser injetora (pelo princípio da casa dos pombos).
Suponha que $|dom(f)| < |ran(f)|$. Desta forma, temos que
$f$ não pode ser sobrejetora. Assim, temos que se $f$ é bijetora então
$|dom(f) = |ran(f)|$.
\end{proof}

Evidentemente, temos que a classe de funções bijetoras é fechada sobre
a operação de composição de funções. Este fato é expresso pelo teorema
seguinte.

\begin{Theorem}
Sejam $f : A \to B$ e $g : B \to C$ funções bijetoras. Então, $g \circ
f : A \to C$ é uma função bijetora.
\end{Theorem}
\begin{proof}
Consequência imediata dos teoremas \ref{inj-compose} e \ref{sobre-compose}.
\end{proof}

\subsection{Função inversa}

Como funções são apenas um caso especial de relações, podemos utilizar
a noção de inversa de uma relação à funções. Porém, nem sempre a
inversa de uma função será uma função. Os próximos exemplos ilustram
estes problemas.

\begin{Example}
Sejam $A =\{1,2,3\}$, $B=\{3,4,5\}$ e $f
=\{(1,3),(2,3),(3,5)\}$. Temos que a inversa da função $f$ é
\[
f^{-1} = \{(3,1),(3,2),(5,3)\}
\]
não pode ser considerada uma função já que não existe uma única imagem
para o valor $3 \in B$.
\end{Example}
Note que este problema é devido ao fato de que a função $f : A \to B$
não era injetora. Será que isto é suficiente para garantir que a
inversa de uma função seja também uma função?  Veremos que somente o
fato de uma função ser injetora não é suficiente para que sua inversa
também seja uma função. O próximo exemplo ilustra este caso.
\begin{Example}
Sejam $A =\{1,2,3\}$, $B=\{4,5,6,7\}$ e
$f=\{(1,4),(2,5),(3,6)\}$. Desta forma temos que a inversa de  $f$ é
dada pelo seguinte conjunto:
\[
f^{-1}=\{(4,1),(5,2),(6,3)\}
\]
que não pode ser considerada uma função de $B \to A$ pois, $7 \in B$,
mas $7 \not\in dom(f^{-1})$ e, portanto, $f^{-1}$ não pode ser
considerada uma função de $B \to A$.
\end{Example}
Note que o motivo que evitou que a inversa da função $f$ do exemplo
anterior fosse considerada uma função é que a imagem desta não era
igual ao seu contradomínio, isto é, $f : A \to B$ não era uma função
sobrejetora.

Desta forma, podemos concluir que a condição necessária e suficiente
para que uma função $f : A \to B$ possua uma função inversa é que $f$
seja uma função bijetora. Este fato é expresso pelo teorema seguinte:
\begin{Theorem}
Seja $f : A \to B$ uma função bijetora. Então $f^{-1} : B \to A$.
\end{Theorem}
\begin{proof}
  Suponha que $f : A \to B$ seja uma função bijetora. Suponha $b$
  arbitrário. Suponha $b \in B$. Para mostrar que $f^{-1}$ é uma
  função devemos mostrar que existe um único valor $a \in A$ tal que
  $f(b) =a$.
  \begin{enumerate}
    \item Existência: Como $f$ é sobrejetora, temos que existe $a \in
      A$ tal que $f(a) = b$ e, portanto, $f^{-1}(b) = a$.
    \item Unicidade: Suponha que $f(a_1) = b$ e $f(a_2) = b$. Como $f$
      é injetora, temos que se $f(a_1) = f(a_2)$ então $a_1 = a_2$.
  \end{enumerate}
\end{proof}

Outra propriedade importante de funções bijetoras e suas inversas é
que a composição destas produz como resultado a função identidade. O
seguinte teorema expressa esse fato.

\begin{Theorem}\label{thm-inversa}
Seja $f : A \to B$ uma função bijetora e $f^{-1} : B \to A$ sua
inversa. Então, temos que $f^{-1} \circ f = i_{A}$ e $f \circ f^{-1}  = i_{B}$.
\end{Theorem}
\subsection{Exercícios}
\begin{enumerate}
  \item Responda o que se pede
  \begin{enumerate}
    \item Seja $A=\{1,2,3\}$, $B = \{4\}$ e $f =
      \{(1,4),(2,4),(3,4)\}$. Neste caso, $f : A \to B$? Justifique.
    \item Seja $A=\{1\}$, $B = \{2,3,4\}$ e $f =
      \{(1,2),(1,3),(1,4)\}$. Neste caso, $f : A \to B$? Justifique.
    \item Seja $C$ o conjunto de todos autom\'oveis registrados no Brasil e
      $S$ o conjunto de todas as strings de tamanho finito formadas por letras
      e n\'umeros. Ent\~ao $L = \{(c,s)\,|\,s\text{ \'e a placa do autom\'ovel
      }c\}$ \'e uma fun\c{c}\~ao?
  \end{enumerate}
  \item Sejam $f, g : \mathbb{R}\to \mathbb{R}$ fun\c{c}\~oes definidas como:
  \[
  \begin{array}{cc}
    \begin{array}{rcl}
      f(x) & = & \frac{1}{x^2 +2}
    \end{array} &
    \begin{array}{rcl}
      g(x) & = & 2x - 1
    \end{array}
  \end{array}
  \]
  Encontre f\'ormulas para $(f\circ g)(x)$ e $(g \circ f)(x)$.
  \item Suponha $f : A \to B$ e $C \subseteq A$. O conjunto $f\cap (C \times
    B)$ \'e uma rela\c{c}\~ao de $C$ em $B$, chamada de restri\c{c}\~ao de $f$
    com respeito a $C$, e \'e representado como $f\,\uparrow\,C$. Isto \'e:
    $f\,\uparrow\,C = f\cap (B\times C)$.
  \begin{itemize}
    \item Prove que $f\,\uparrow\,C : C \to B$ e que para todo $c \in C$,
      $f(c) = (f\,\uparrow\,C)(c)$.
    \item Prove que se $f$ \'e injetora, ent\~ao $f\,\uparrow\,C$ tamb\'em
      \'e.
    \item Prove que se $f$ \'e sobretora, ent\~ao $f\,\uparrow\,C$ tamb\'em
      \'e.
   \end{itemize}
  \item Suponha $f : A \to B$ e $S$ uma rela\c{c}\~ao bin\'aria sobre
    $B$. Defina $R$ como sendo:
    \[
    R = \{(x,y)\in A \times A\,|\, (f(x),f(y)) \in S\}
    \]
    \begin{enumerate}
      \item Prove que se $S$ \'e reflexiva, ent\~ao $R$ tamb\'em \'e.
      \item Prove que se $S$ \'e transitiva, ent\~ao $R$ tamb\'em \'e.
      \item Prove que se $S$ \'e sim\'etrica, ent\~ao $R$ tamb\'em \'e.
    \end{enumerate}

  \item Prove os teoremas \ref{sobre-compose} e \ref{thm-inversa}.
\end{enumerate}

\section{Notas Bibliográficas}