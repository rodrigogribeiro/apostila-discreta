\chapter{Recursividade}\label{cap10}

\epigraph{Recursive. adj. See RECURSIVE.}{Stan Kelly-Bootie --- The
  Devil's DP Dictionary}

\section{Motivação}

Tanto em matemática, quanto na ciência da computação, diversas
operações são definidas recursivamente, isto é, alguns valores
iniciais para esta operação são dados e os demais são obtidos
aplicando-se uma ou mais regras sucessivamente. Um exemplo de função
recursiva é a definição do fatorial, apresentada abaixo:

\[
\begin{array}{lcl}
0! & = & 1\\
n! & = & n \times (n - 1)!
\end{array}
\]

Como valor inicial, temos que o fatorial de $0$ é $1$ e, demais
valores são obtidos pela segunda equação da definição.

De certa forma, provas por indução possuem uma estrutura similar a
definições recursivas: apresenta-se provas de fatos elementares (casos
base) e usa-se uma regra (passo indutivo) para mostrar que o fato em
questão é válido para elementos diferentes dos considerados nos casos
base. Neste capítulo, veremos como a indução é utilizada para
demonstrar propriedades sobre definições recursivas.

\section{Funções Recursivas}

Existem diversas maneiras de se definir funções. Podemos definir uma
função usando uma expressão que caracteriza a relação entre o domíno e
sua imagem (método usualmente utilizado na matemática). Outra maneira
de se definir uma função é através do uso de composição, que permite a
definição de funções utilizando definições prévias. Esta forma de
definir funções é o mais próximo do que idealmente deve ser feito em
computação. Existe, ainda uma terceira forma de se definir uma função:
utilizando recursão. Como exemplo,
considere a seguinte função $f : \mathbb{N} \to \mathbb{N}$ definida
como
\[
\left\{
\begin{array}{lcl}
  f(0) & = & 1 \\
  f(n) & = & 2n + f(n - 1)\\
\end{array}
\right .
\]
Note que esta definição especifica um valor inicial para $f$, $f(0) =
1$ e os demais valores são obtidos a partir de valores ``anteriores''
desta função. Como exemplo, considere o cálculo de $f(5)$, apresentado
abaixo:
\[
\begin{array}{lc}
f(5) & = \\
2.5 + f(4) & = \\
10 + (2.4 + f(3)) & = \\
10 + (8 + (2.3 + f(2))) & = \\
 10 + (8 + (6 + 2.2 + f(1))) & = \\
10 + (8 + (6 + (4 + (2.1 + f(0))))) & = \\
10 + (8 + (6 + (4 + (2 + 1)))) & = \\
31
\end{array}
\]
Apesar de simples compreensão, o uso de funções recursivas possui o
inconveniente de que o cálculo desta para valores elevados do domínio
pode consumir muito tempo. Considere calcular $f(2000)$. Este cálculo
ocasionaria $2000$ chamadas recursivas. Porém, muitas vezes, podemos
encontrar uma função $g$, equivalente a $f$, sem
recursividade. Existem diversas técnicas para solucionar este tipo de
problema e apresentaremos a mais simples destas baseada em indução matemática.
Inicialmente, montamos uma pequena tabela de valores para $f$:
\[
\begin{array}{|c|c|}
  \hline
  n & f(n) \\ \hline
  0 &  1 \\
  1 &  3 \\
  2 &  7 \\
  3 & 13 \\
  4 & 21 \\
  5 & 31 \\
  6 & 43 \\ \hline
\end{array}
\]
Após pensar um pouco, podemos conjecturar que a função
\[
g(n) = n (n+1) + 1
\]
é equivalente a $f$, uma vez que esta possui os mesmos valores que
$f$, conforme tabela abaixo:
\[
\begin{array}{|c|c|c|}
  \hline
  n & f(n) & g(n)\\ \hline
  0 &  1  & 1 \\
  1 &  3  & 3 \\
  2 &  7  & 7 \\
  3 & 13 & 13 \\
  4 & 21 & 21 \\
  5 & 31 & 31 \\
  6 & 43 & 43\\ \hline
\end{array}
\]
Porém, somente construir e verificar esta tabela para alguns valores
não é suficiente para mostrar que $f(n) = n(n+1) + 1$. Para isso,
devemos provar que:
\[
\forall n. n\in\mathbb{N} \to f(n) = n(n+1) + 1
\]
que pode ser provado por indução matemática, conforme apresentado no
teorema seguinte.

\begin{Theorem}
Seja $f(n)$ uma função definida como:
\[
\left\{
\begin{array}{lcl}
  f(0) & = & 1 \\
  f(n) & = & 2n + f(n - 1)\\
\end{array}
\right .
\]
então $f(n) = n(n+1) + 1$.
\end{Theorem}
\begin{proof}
\verb| |\\
\begin{enumerate}
  \item[\ ]Caso base ($n = 0$): Temos que $f(0) = 1 = 0(0 +1) + 1$,
    conforme requerido.
  \item[\ ]Passo indutivo: Suponha $n\in\mathbb{N}$ arbitrário e que
    $f(n) = n(n+1) + 1$. Temos:
   \[
      \begin{array}{lcl}
      f(n+1) & = \\
      2 (n+1) + f(n) & = & \text{pela definição de }f(n)\\
      2(n+1) + n(n+1) + 1 & = & \text{pela hipótese de indução}\\
      (n+1)[(n+1) + 1] +1
      \end{array}
   \]
   Logo, $f(n+1) = (n+1)[(n+1) + 1] + 1$ conforme requerido.
\end{enumerate}
\end{proof}

De maneira geral, podemos obter uma fórmula fechada (isto é, sem
recursividade) para uma função recursiva $f(n)$ usando os seguintes
passos:
\begin{enumerate}
  \item Construir uma tabela contendo alguns valores da função $f(n)$.
  \item ``Adivinhar'', a partir da tabela construída no passo
    anterior, qual função não recursiva produz os mesmos resultados
    para os valores da tabela.
  \item Provar, usando indução matemática, que a fórmula fechada
    encontrada é realmente equivalente a função em questão.
\end{enumerate}
A seguir, mostraremos mais exemplo desta técnica encontrando uma
fórmula fechada para a seguinte função recursiva.

\[
\left\{
\begin{array}{lcl}
    f(0) & = & 0\\
    f(n) & = & 2f(n - 1) + 1
\end{array}
\right.
\]
Inicialmente, construiremos uma tabela contendo alguns valores de
$f(n)$:
\[
\begin{array}{|c|c|}
  \hline
  n & f(n) \\ \hline
  0 &  0 \\
  1 &  1 \\
  2 &  3 \\
  3 &  7 \\
  4 & 15 \\
  5 & 31\\
  6 & 63 \\ \hline
\end{array}
\]
Se observarmos os valores da tabela, podemos perceber que estes são
próximos de potências perfeitas de $2$, logo, podemos conjecturar que
a fórmula fechada para $f(n)$ é $2^n - 1$. Constataremos este fato
provando por indução.
\begin{Theorem}\label{thmhanoi}
Seja $f(n)$ a função definida como
\[
\left\{
\begin{array}{lcl}
    f(0) & = & 0\\
    f(n) & = & 2f(n - 1) + 1
\end{array}
\right.
\]
então $f(n) = 2^n - 1$.
\end{Theorem}
\begin{proof}
\verb| |\\
\begin{enumerate}
  \item[\ ]Caso base: Para $n = 0$, temos $f(0) = 0 = 1 - 1 = 2^0 - 1$.
  \item[\ ]Passo indutivo: Suponha $n\in\mathbb{N}$ arbitrário e que
    $f(n) = 2^n - 1$. Temos que:
\[
\begin{array}{lcl}
    f(n + 1) & = & \\
    2f(n) + 1 & = & \\
    2(2^n - 1) + 1 & = & \{\text{pela hipótese de indução}\}\\
    2^{n+1} - 2 + 1 & = &\\
    2^{n+1} - 1
\end{array}
\]
Logo, $f(n + 1) = 2^{n + 1} - 1$.
\end{enumerate}
\end{proof}
A seguir apresentamos alguns problemas clássicos e como estes podem
ser modelados utilizando funções recursivas.


\subsection{As Torres de Hanói}

As torres de Hanói é um quebra-cabeça inventado por um matemático
francês, Édouard Lucas em 1833.  Este quebra-cabeça consiste de uma
torre contendo uma quantidade $n\in\mathbb{N}$ de discos, inicialmente
empilhados em ordem decrescente de tamanho. A figura abaixo, apresenta
a configuração inicial deste quebra-cabeças:

\begin{figure}[h!]
  \centering
      \includegraphics[width=0.5\textwidth]{imagens/torredehanoi.jpg}
 \end{figure}
O objetivo deste jogo é transferir todos os discos de um pino para
outro movendo apenas um disco de cada vez e nunca colocando um disco
maior em cima de um menor.

Apesar de simples, não é óbvio que este quebra-cabeças possui
solução. Após pensar um pouco, podemos perceber que este de fato,
sempre possui solução. Porém, qual será a melhor? Isto é, é possível
solucionar este problema fazendo o menor número de movimentos?

Para chegar a resposta para esta pergunta, devemos primeiro introduzir
algumas notações. Chamaremos de $T(n)$ o número de movimentos
necessários para solucionar o quebra cabeças contendo $n$ discos.

É bastante fácil ver que $T(0) = 0$ e que $T(1) = 1$. A figura
seguinte, mostra passo a passo, a solução para $n = 2$.\\

\begin{figure}[H]
  \centering
      \includegraphics[width=0.5\textwidth]{imagens/hanoi2.jpg}
 \end{figure}

Para $n = 3$, temos: \\

\begin{figure}[h!]
  \centering
      \includegraphics[width=0.5\textwidth]{imagens/hanoi3.jpg}
 \end{figure}

Com isso, temos a seguinte tabela de valores iniciais de $T(n)$:

\[
\begin{array}{|c|c|}
  \hline
  n & T(n)\\ \hline
  0  & 0 \\
  1  & 1 \\
  2 & 3 \\
  3 & 7\\
  \hline
\end{array}
\]

Agora, que fizemos alguns experimentos com este problema, vamos mudar
nossa perspectiva: ao invés de tentar pensar em como resolver este
problema para casos específicos, vamos tentar
generalizá-lo. Observando a figura para a solução com 3 discos,
podemos perceber que o problema para $n = 3$ é resolvido da seguinte
maneira:
\begin{itemize}
  \item Mova $n - 1$ discos do pino $A$ para o pino $B$.
  \item Mova o disco $n$ do pino $A$ para o pino $C$.
  \item Mova $n - 1$ discos do pino $C$ para o pino $C$.
\end{itemize}
Como, para mover $n - 1$ discos de um pino para outro, precisamos de $T(n
-1)$ movimentos, no total precisamos de
\[
T(n - 1) + T(n - 1) + 1 = 2T(n - 1) + 1
\]
para solucionar um quebra-cabeças de tamanho $n$. Assim, temos que o
número mínimo de movimentos para a solução deste problema é dado pela
seguinte função recursiva:
\[
\left\{
\begin{array}{lcl}
T(0) & = & 0 \\
T(n) & = & 2 T(n - 1) + 1
\end{array}
\right.
\]
Mas será que esta função reflete os resultados que obtivemos
solucionando o problema? Vamos fazer os cálculos para $n = 3$:
\[
\begin{array}{lc}
T(3) & = \\
2 T(2) + 1 & = \\
2(2T(1) + 1) + 1 & =\\
2(2(2T(0) + 1) + 1) + 1 & = \\
2(2(2.0 + 1) + 1) + 1 & = \\
7
\end{array}
\]
conforme requerido. Como vimos anteriormente, funções recursivas são
usualmente ineficientes para o cálculo manual. Logo, é uma boa prática
encontrarmos uma fórmula fechada para a função em questão. Porém, já
encontramos esta fórmula no teorema \ref{thmhanoi}.


\subsection{O Problema da Pizzaria}

Suponha que em um fim de semana você tenha ido a uma pizzaria que
possuía a seguinte promoção:

\begin{quote}
``O cliente que conseguir descobrir o número máximo de pedaços que pode
ser obtido ao se fazer $n \in \mathbb{N}$ cortes em uma pizza, não a pagará.''
\end{quote}

Então, como pode-se comer uma pizza de graça? Novamente, vamos seguir
a estratégia utilizada no exemplo anterior. Primeiro, vamos chamar de
$T(n)$ o número de fatias obtidas após fazermos o $n$-ésimo corte. É
bem fácil perceber que $T(0) = 1$, visto que se não fizermos nenhum
corte, temos uma fatia (a pizza inteira). Usando um raciocínio
parecido, temos que $T(1) = 2$, visto que ao fazermos um corte, iremos
dividir a pizza em dois pedaços. Porém, quantos pedaços obtemos ao
fazer o $3^o$ corte? A intuição nos diz que devemos obter $T(3) = 6$,
porém, conforme mostrado na próxima figura, isso não é bem verdade...

\begin{figure}[H]
  \centering
      \includegraphics[width=0.5\textwidth]{imagens/plane.jpg}
 \end{figure}

Note que obtemos um número maior de pedaços fazendo com que o
$n$-ésimo corte intercepte todos os cortes anteriores. Com isso,
aumentamos o número total de fatias em $n$ pedaços, isto é, $T(3) = 4
+ 3 = 7$, em que $4 = T(2)$. Desta forma, podemos conjecturar que
$T(n)$ é a seguinte função recursiva:

\[
\left\{
\begin{array}{lcl}
  T(0) & = & 1 \\
  T(n) & = & T(n - 1) + n
\end{array}
\right.
\]

Note que ao calcularmos alguns valores de $T(n)$, podemos notar que
este nada mais é que a soma dos $n$ primeiros números naturais somados
com 1, conforme expandido abaixo:

\[
\begin{array}{lcl}
T(n) & = & T(n - 1) + n \\
       & = & (T(n - 2) + (n - 1)) + n\\
       & = & ( (T(n - 3) + (n - 2)) + (n - 1)) + n\\
       &  & \vdots\\
       & = & T(0) + 1 + 2 + ... + (n -2) + (n - 1) + n\\
       & = & 1 + 1 + 2 + ... + (n -2) + (n - 1) + n\\
       & = & 1 + \sum_{k = 1}^nk
\end{array}
\]
Pode-se mostrar por indução que $\sum_{k = 1}^n =
\frac{n(n+1)}{2}$. Logo, temos que $T(n)$ é dado por:
\[
T(n) = \frac{n(n+1)}{2} + 1
\]
Realmente esta fórmula corresponde a função $T(n)$, conforme provamos
no teorema a seguir.

\begin{Theorem}
Seja $T(n)$ a função definida como:
\[
\left\{
\begin{array}{lcl}
  T(0) & = & 1 \\
  T(n) & = & T(n - 1) + n
\end{array}
\right.
\]
então, $T(n) = \frac{n(n+1)}{2} + 1$.
\end{Theorem}
\begin{proof}
\begin{enumerate}
  \item[\ ]Caso base ($n = 0$): Temos que $T(0) = \frac{0(0 + 1)}{2} +
    1$, conforme requerido.
  \item[\ ]Passo indutivo: Suponha $n\in\mathbb{N}$ arbitrário e que
    $T(n) = \frac{n(n+1)}{2} + 1$. Temos que:
   \[
\begin{array}{lcl}
T(n + 1) & = \\
T(n) + (n + 1) & = & \{\text{pela def. de }T(n)\}\\
\frac{n(n+1)}{2} + 1 + (n + 1) & = & \{\text{pela hipótese de
  indução}\}\\
\frac{n(n + 1) + 2(n+1)}{2} & = & \\
\frac{(n + 1)(n + 2)}{2}
\end{array}
   \]
conforme requerido.
\end{enumerate}
\end{proof}

\subsection{Conjunto Potência, Recursivamente}

No capítulo \ref{cap5}, apresentamos a definição do conjunto potência
(ou conjunto das partes) de um conjunto $A$:
\[\mathcal{P}(A) =\{X\,|\,X\subseteq A\}\]
É fácil mostrar que $|\mathcal{P}(A)| = 2^n$ se $|A| = n$, usando o
princípio multiplicativo (veja no capítulo \ref{cap6}).

Porém, como provar este resultado usando indução? Pode-se argumentar que basta
utilizar indução sobre o tamanho do conjunto. Logo, no caso base, para
$n = 0$, consideramos o conjunto vazio e obtemos $\mathcal{P}(\emptyset)=\{\emptyset\}$.

No caso indutivo, devemos considerar o cálculo do conjunto potência de
um conjunto $A$ com pelo menos um elemento. Isto é:
\[
A = B \cup \{a\}\,\,\,a\not\in B
\]

Essas observações levam a seguinte função recursiva que, a partir de
um conjunto qualquer, produz o conjunto potência deste.

\[
\left\{
\begin{array}{lcl}
\mathcal{P}(\emptyset) & = & \{\emptyset\}\\
\mathcal{P}(B \cup \{a\}) & = & \mathcal{P}(B)\cup \{X \cup
\{a\}\,|\,X \in \mathcal{P}(B)\}\,\,\,\,\text{em que }a\not\in B
\end{array}
\right.
\]
Observe que o cálculo do conjunto potência inclui o elemento $a$ em
cada um dos subconjuntos de $B$. Evidentemente, como $\mathcal{P}(B)$
e $\{X \cup\{a\}\,|\,X \in \mathcal{P}(B)\}$ são disjuntos e cada um
destes possui $2^n$ elementos (pela hipótese de indução), temos que
$|\mathcal{P}(A)| = 2^{n + 1}$. A demonstração deste fato é
apresentada a seguir.

\begin{Theorem}
Para todo $A$, se $|A| = n$ então $|\mathcal{P}(A)|=2^{n}$.
\end{Theorem}
\begin{proof}
\verb| |\\
\begin{enumerate}
  \item[\ ]Caso base ($n = 0$): Neste caso, temos que $A =
    \emptyset$. Logo, $|P(\emptyset)| = |\{\emptyset\}| = 1 = 2^0$,
    conforme requerido.
  \item[\ ]Passo indutivo: Suponha $n \in\mathbb{N}$ arbitrário e que
    $|B| = n$ e $|\mathcal{P}(B)| = 2^n$. Suponha $a$ arbitrário tal
    que $a\not\in B$ e que $A = B \cup {a}$. Seja $X = \{Y \cup
    \{a\}\,|\,Y \in \mathcal{P}(B)\}$. É óbvio que $|X| = 2^n$. Como
    $\mathcal{P}(A) = \mathcal{P}(B) \cup X$, temos que
    $|\mathcal{P}(A)| = |\mathcal{P}(B)| + |X| = 2^n + 2^n = 2^{n + 1}$.
\end{enumerate}
\end{proof}


\subsection{A Sequência de Fibonnaci}

\[
\left\{
\begin{array}{lcl}
F(0) & = & 0 \\
F(1) & = & 1 \\
F(n) & = & F(n - 1) + F(n - 2)
\end{array}
\right.
\]

\section{Exercícios}
\begin{enumerate}
	\item Encontre uma f\'ormula fechada (sem recursividade) equivalente a cada uma das fun\c{c}\~oes recursivas a seguir
		  e prove que a f\'ormula encontrada \'e equivalente a fun\c{c}\~ao em quest\~ao.
	\begin{enumerate}
		\item
				$\left\{
					\begin{array}{l}
						T(0) = 0 \\
						T(n) = 2T(n - 1) + n
					\end{array}
			    \right .$

		\item
				$\left\{
					\begin{array}{l}
						T(0) = 2 \\
						T(n) = (T(n - 1))^{2}
					\end{array}
			    \right .$

		\item
				$\left\{
					\begin{array}{l}
						T(0) = 2 \\
						T(n) = 2T(n - 1) + n
					\end{array}
			    \right .$

		\item
				$\left\{
					\begin{array}{l}
						T(1) = 1 \\
						T(n) = \frac{T(n-1)}{1 + T(n - 1)}
					\end{array}
			    \right .$
               \item
                            $\left\{
					\begin{array}{l}
						T(1) = \frac{1}{4} \\
						T(2) = \frac{1}{8}\\
                                                T(n) =
                                                \frac{T(n-1)T(n-2)}{2T(n-2)
                                                - T(n-1)}
					\end{array}
			    \right .$
	\end{enumerate}
	\item Seja $F(n)$ o $n$-\'esimo termo da sequ\^encia de Fibonacci, definida como:
	\[
		\left\{
			\begin{array}{lcl}
				F(0) & = & 0\\
				F(1) & = & 1 \\
				F(n) & = & F(n - 1) + F(n - 2)
			\end{array}
		\right.
	\]
	Prove os seguintes fatos sobre a sequ\^encia de Fibonacci.
	\begin{enumerate}
		\item $\sum_{i=0}^{n}F(i) = F(n + 2) - 1$
		\item $\sum_{i=0}^{n}F(2i + 1) = F(2n + 2)$
		\item $\sum_{i=0}^{n}(F(i))^{2} = F(n)F(n+1)$
		\item Prove que para todo $n\in\mathbb{N}$, $F(n) < 2^n$.
	\end{enumerate}
\end{enumerate}

\section{Notas Bibliográficas}

Recursividade e sua relação com a indução matemática é um tema
presente em todo texto de matemática discreta. O foco do capítulo
atual foi o uso de indução matemática para demonstrar fórmulas
fechadas equivalentes a funções recursivas. Alguns dos exemplos deste
capítulo foram retirados de \cite{Graham94}.